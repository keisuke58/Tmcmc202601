% ══════════════════════════════════════════════════════════════════════════════
% Comprehensive Defense of E(DI) Constitutive Law
% ══════════════════════════════════════════════════════════════════════════════
% Insert into paper as Section X.X
% Addresses: exponent n, functional form, E bounds, DI→p mapping,
%            E_min residual stiffness, Bayesian non-identifiability
% ══════════════════════════════════════════════════════════════════════════════

\subsubsection{Constitutive model: justification and sensitivity}
\label{sec:constitutive_sensitivity}

The constitutive model
\begin{equation}
  E_{\mathrm{bio}}(DI) = E_{\max}(1-r)^n + E_{\min}\,r,
  \quad r = \mathrm{clamp}(DI / s_{DI}, 0, 1)
  \label{eq:Edi}
\end{equation}
involves three structural choices: the exponent~$n$,
the functional form (power-law), and the boundary values
$E_{\max}$, $E_{\min}$.
We justify each in turn and demonstrate that all qualitative
conclusions are robust to these choices
(Fig.~\ref{fig:n_sensitivity}, Table~\ref{tab:n_sensitivity}).

% ── (1) Exponent n ──────────────────────────────────────────────────────────

\paragraph{(1) Exponent $n = 2$: percolation analogy.}
Biofilm stiffness originates from the extracellular polymeric substance (EPS)
matrix, a cross-linked polymer network embedding bacterial
cells~\cite{Flemming2010,Koo2017}.
The elastic modulus of cross-linked networks near the gel--sol transition
obeys \emph{rigidity percolation} scaling~\cite{Sahimi1994,deGennes1979}:
\begin{equation}
  E \sim (p - p_c)^f, \qquad p > p_c,
  \label{eq:percolation}
\end{equation}
where $p$ is the cross-link density, $p_c$ the percolation threshold,
and $f$ the universal rigidity exponent.
For three-dimensional central-force networks,
$f = 2.0 \pm 0.1$~\cite{Arbabi1993};
for bond-bending models, $f \approx 1.7$~\cite{Feng1984,Kantor1984}.
Identifying the rescaled diversity measure $1 - r$ with the reduced
cross-link density $(p - p_c)/(1 - p_c)$ yields
\begin{equation}
  E_{\max}(1-r)^n
  \;=\; E_{\max}\!\left(\frac{p - p_c}{1 - p_c}\right)^{\!f},
  \qquad n = f \approx 2.
  \label{eq:reduction}
\end{equation}
This identification is \emph{phenomenological}:
$DI$ measures Shannon entropy of species composition, not polymer
cross-linking directly.
The assumption that species diversity monotonically promotes
EPS network connectivity rests on three lines of indirect evidence:

\begin{enumerate}[label=(\alph*)]
  \item \textbf{Multi-species emergent stiffness.}
    Gloag et al.~\cite{Gloag2019} showed that dual-species biofilms
    exhibit emergent mechanical properties absent in either monoculture,
    attributable to cross-linking between heterogeneous EPS types.

  \item \textbf{EPS knockout studies.}
    Disruption of single EPS pathways causes disproportionate stiffness
    loss---curli deletion ($50\%$ $E$ reduction~\cite{Hollenbeck2018}),
    $\Delta$\textit{eps} in \textit{B.~subtilis}
    ($70$--$80\%$ loss~\cite{Kesel2016}),
    \textit{gtfB}/\textit{gtfC} knockouts
    ($50$--$70\%$ EPS reduction~\cite{Bowen2011})---consistent with
    percolation-like threshold behavior where removing one cross-link
    type collapses the network.

  \item \textbf{Enzymatic disruption.}
    Selective cleavage of cross-link types
    (DNase~I targeting eDNA, dextranase targeting glucans)
    causes viscoelastic collapse~\cite{Flemming2010,Liao2014},
    confirming that the EPS network operates near
    a connectivity threshold rather than deep in the elastic regime.
\end{enumerate}

\noindent
A direct validation would require simultaneous measurement of species
composition and EPS cross-link density in the same sample.
This remains a target for future work (Section~\ref{sec:future}).

\paragraph{Sensitivity to $n$.}
The full analysis was repeated for $n \in \{1.0, 1.5, 2.0, 2.5, 3.0\}$
and continuously for $n \in [0.5, 4.0]$
using 50 unfiltered posterior samples per condition
(Fig.~\ref{fig:n_sensitivity}a,c,d).
Three findings establish robustness:
\begin{itemize}
  \item The condition ordering
    $E_{\mathrm{CS}} \geq E_{\mathrm{CH}} > E_{\mathrm{DH}} \gg E_{\mathrm{DS}}$
    is preserved for \emph{all} $n \in [0.5, 4.0]$.
  \item The pseudo Bayes factor (DI vs $\varphi_{Pg}$) exceeds
    $10^6$ for every tested~$n$
    (minimum $BF = 1.8 \times 10^6$ at $n = 3.0$, CS--DH pair),
    far above the ``decisive'' threshold of~100~\cite{Kass1995}.
  \item The CS/DS stiffness ratio varies from $6\times$ ($n=1$)
    to $75\times$ ($n=3$); at $n = 2$, the ratio of $29\times$ lies
    within the $10$--$80\times$ range measured by
    Pattem et al.~\cite{Pattem2018} for sucrose-dependent biofilm
    stiffness variation.
    The literature-compatible range of $n$ is $[1.3, 3.3]$
    (Fig.~\ref{fig:n_sensitivity}d).
\end{itemize}

% ── (2) Functional form ─────────────────────────────────────────────────────

\paragraph{(2) Why power-law?}
We compared five functional forms
(Fig.~\ref{fig:n_sensitivity}b):
power-law ($n=2$), linear ($n=1$), logistic sigmoid ($k=10$, $DI_{\mathrm{mid}}=0.5$),
exponential decay ($\lambda=3$), and piecewise linear ($DI_c = 0.5$).
Using the posterior-mean species fractions, the CS/DS stiffness
ratios are: power-law $29\times$, sigmoid $25\times$,
exponential $10\times$, linear $6\times$, piecewise $3\times$.
Only power-law, sigmoid, and exponential fall within the experimental
range of $10$--$80\times$~\cite{Pattem2018}.
The power-law is preferred because (i)~it has a direct physical
interpretation via percolation (Eq.~\ref{eq:reduction}),
(ii)~it provides the best quantitative match to the center of
the literature range, and (iii)~it reduces to a single free parameter~$n$
(vs.\ two for sigmoid, one for exponential).
We note that the sigmoid model also performs adequately;
the qualitative conclusions are form-independent.

% ── (3) E_min: residual stiffness ──────────────────────────────────────────

\paragraph{(3) $E_{\min} > 0$: residual stiffness.}
Classical percolation predicts $E = 0$ below~$p_c$
(Eq.~\ref{eq:percolation}).
We retain $E_{\min} = 10$~Pa because even a mono-dominated,
EPS-depleted biofilm retains finite stiffness from the bacterial
cell bodies themselves.
Single-cell AFM measurements report
$E_{\mathrm{cell}} \sim 10$--$100$~Pa for hydrated
bacteria~\cite{Longo2013,Formosa2015}.
The $E_{\min}$ term thus represents the lower bound where the EPS
network has percolated below~$p_c$ but the cellular scaffold persists.

% ── (4) E_max / E_min sensitivity ──────────────────────────────────────────

\paragraph{(4) Sensitivity to $E_{\max}$ and $E_{\min}$.}
Since $E_{\max}$ and $E_{\min}$ are calibrated to the literature
range ($20$--$14{,}000$~Pa) rather than fit to data,
we tested $E_{\max} \in [500, 1500]$~Pa and $E_{\min} \in [5, 50]$~Pa
($\pm 50\%$ of default).
For all $2{,}500$ tested combinations,
$E_{\mathrm{CS}} > E_{\mathrm{DS}}$ (ordering preserved),
and $99\%$ of parameter space yields CS/DS ratios within the
Pattem $10$--$80\times$ range
(Fig.~\ref{fig:n_sensitivity}f).

% ── (5) Why not estimate n? ────────────────────────────────────────────────

\paragraph{(5) Why not infer $n$ from data?}
Adding $n$ as a 21st TMCMC parameter would require
condition-resolved experimental measurements of $E_{\mathrm{bio}}$
to constrain it.
Since our TMCMC posterior is conditioned on ODE-predicted species fractions
(not elastic moduli), $n$ would be non-identifiable and
dominated by the prior.
The sensitivity analysis (Fig.~\ref{fig:n_sensitivity}c--e)
provides a stronger argument:
it demonstrates conclusion-independence over a wide range of~$n$,
which is more informative than a poorly constrained posterior.

% ── (6) Non-parametric validation ──────────────────────────────────────────

\paragraph{(6) Non-parametric validation.}
The pseudo Bayes factor uses a Gaussian approximation for the
Bhattacharyya distance.
To validate this, we computed the kernel density estimate (KDE)
overlap coefficient between condition pairs at each~$n$
(Fig.~\ref{fig:n_sensitivity}e, dashed).
The KDE overlap is exactly zero for both CS--DH and CS--DS pairs
at all tested~$n$, confirming complete distributional separation
independent of parametric assumptions.

% ── (7) CH ≥ CS is physically expected ──────────────────────────────────────

\paragraph{(7) Why $E_{\mathrm{CH}} \geq E_{\mathrm{CS}}$?}
The commensal HOBIC condition (periodic aerobic--anaerobic cycling)
promotes more uniform species composition than static culture,
yielding lower $DI$ and thus higher~$E$.
This is physically expected: environmental fluctuations prevent
any single species from dominating, maintaining high
cross-link diversity.

% ── Table ───────────────────────────────────────────────────────────────────

\begin{table}[h]
\centering
\caption{Sensitivity of $E(DI)$ predictions to exponent~$n$.
Mean $E_{\mathrm{bio}}$ [Pa] from 50 posterior samples per condition.
BF: pseudo Bayes factor (DI vs $\varphi_{Pg}$, CS--DH pair).
All $n$ yield ordering CS $>$ DH $>$ DS and decisive BF.}
\label{tab:n_sensitivity}
\begin{tabular}{ccccccc}
\toprule
$n$ & $E_{\mathrm{CS}}$ & $E_{\mathrm{CH}}$ & $E_{\mathrm{DH}}$ & $E_{\mathrm{DS}}$
    & CS/DS & BF(CS--DH) \\
\midrule
1.0 &  970 &  998 &  528 &  163 &  $6\times$  & $2.0 \times 10^{12}$ \\
1.5 &  956 &  997 &  383 &   69 & $14\times$  & $6.3 \times 10^{10}$ \\
\textbf{2.0} & \textbf{942} & \textbf{995} & \textbf{279} & \textbf{32}
    & $\mathbf{29\times}$ & $\mathbf{1.4 \times 10^{9}}$ \\
2.5 &  929 &  994 &  203 &   18 & $52\times$  & $3.8 \times 10^{7}$ \\
3.0 &  916 &  993 &  148 &   12 & $75\times$  & $1.8 \times 10^{6}$ \\
\bottomrule
\end{tabular}
\end{table}

% ══════════════════════════════════════════════════════════════════════════════
% BibTeX entries
% ══════════════════════════════════════════════════════════════════════════════

% @book{Sahimi1994,
%   author    = {Sahimi, Muhammad},
%   title     = {Applications of Percolation Theory},
%   publisher = {Taylor \& Francis},
%   year      = {1994}
% }
% @book{deGennes1979,
%   author    = {de Gennes, Pierre-Gilles},
%   title     = {Scaling Concepts in Polymer Physics},
%   publisher = {Cornell University Press},
%   year      = {1979}
% }
% @article{Arbabi1993,
%   author  = {Arbabi, S. and Sahimi, M.},
%   title   = {Mechanics of disordered solids. {I}. {P}ercolation on elastic
%              networks with central forces},
%   journal = {Phys.\ Rev.\ B},
%   volume  = {47},
%   pages   = {695--702},
%   year    = {1993}
% }
% @article{Feng1984,
%   author  = {Feng, S. and Sen, P. N.},
%   title   = {Percolation on elastic networks: new exponent and threshold},
%   journal = {Phys.\ Rev.\ Lett.},
%   volume  = {52},
%   pages   = {216--219},
%   year    = {1984}
% }
% @article{Kantor1984,
%   author  = {Kantor, Y. and Webman, I.},
%   title   = {Elastic properties of random percolating systems},
%   journal = {Phys.\ Rev.\ Lett.},
%   volume  = {52},
%   pages   = {1891--1894},
%   year    = {1984}
% }
% @article{Kass1995,
%   author  = {Kass, R. E. and Raftery, A. E.},
%   title   = {Bayes factors},
%   journal = {J.\ Am.\ Stat.\ Assoc.},
%   volume  = {90},
%   pages   = {773--795},
%   year    = {1995}
% }
% @article{Gloag2019,
%   author  = {Gloag, E. S. and others},
%   journal = {J.\ Bacteriol.},
%   year    = {2019},
%   note    = {PMC6707914}
% }
% @article{Hollenbeck2018,
%   author  = {Hollenbeck, E. C. and others},
%   journal = {Biophys.\ J.},
%   year    = {2018}
% }
% @article{Kesel2016,
%   author  = {Kesel, S. and others},
%   journal = {Appl.\ Environ.\ Microbiol.},
%   year    = {2016}
% }
% @article{Bowen2011,
%   author  = {Bowen, W. H. and Koo, H.},
%   journal = {Caries Res.},
%   volume  = {45},
%   pages   = {69--86},
%   year    = {2011}
% }
% @article{Pattem2018,
%   author  = {Pattem, J. and others},
%   journal = {Sci.\ Rep.},
%   year    = {2018},
%   note    = {PMC5890245}
% }
% @article{Longo2013,
%   author  = {Longo, G. and others},
%   title   = {Rapid detection of bacterial resistance to antibiotics
%              using AFM cantilevers as nanomechanical sensors},
%   journal = {Nat.\ Nanotechnol.},
%   volume  = {8},
%   pages   = {522--526},
%   year    = {2013}
% }
% @article{Formosa2015,
%   author  = {Formosa, C. and others},
%   title   = {Nanoscale analysis of the effects of antibiotics and CX1
%              on a \textit{Pseudomonas aeruginosa} multidrug-resistant strain},
%   journal = {Sci.\ Rep.},
%   volume  = {5},
%   pages   = {12088},
%   year    = {2015}
% }
