% ============================================================
%  openjaw_openfulljaw.tex
%  Full documentation: OpenJaw Patient-1 FEM biofilm analysis
%  Hollow crown (T23) + inter-proximal slit (T30-T31)
%  DI-mapped transversely isotropic material
%
%  Compile:  pdflatex openjaw_openfulljaw.tex
% ============================================================

\documentclass[11pt,a4paper]{article}

\usepackage[T1]{fontenc}
\usepackage[utf8]{inputenc}
\usepackage[margin=25mm]{geometry}
\usepackage{amsmath,amssymb}
\usepackage{graphicx}
\usepackage{booktabs}
\usepackage{array}
\usepackage{xcolor}
\usepackage{hyperref}
\usepackage{listings}
\usepackage{caption}
\usepackage{subcaption}
\usepackage{enumitem}
\usepackage{float}

% ---- colour scheme ----
\definecolor{codeblue}{RGB}{25,80,160}
\definecolor{codegray}{RGB}{248,248,248}
\definecolor{codecomment}{RGB}{100,120,100}
\definecolor{darkred}{RGB}{160,30,30}
\definecolor{darkgreen}{RGB}{20,120,60}

\hypersetup{
  colorlinks=true, linkcolor=codeblue,
  urlcolor=codeblue, citecolor=darkred,
  pdftitle={OpenJaw P1 Full-Jaw FEM Analysis},
  pdfauthor={IKM Hiwi}
}

\definecolor{framecolor}{RGB}{200,200,200}
\definecolor{bashbg}{RGB}{242,242,242}

\lstset{
  language=Python,
  backgroundcolor=\color{codegray},
  basicstyle=\ttfamily\footnotesize,
  keywordstyle=\color{codeblue}\bfseries,
  commentstyle=\color{codecomment}\itshape,
  stringstyle=\color{darkred},
  numbers=left, numberstyle=\tiny\color{gray},
  numbersep=5pt,
  breaklines=true,
  frame=single, rulecolor=\color{framecolor},
  captionpos=b,
  showstringspaces=false
}

\lstdefinelanguage{bash}{
  morekeywords={abaqus,python3,pdflatex,cd},
  morecomment=[l]{\#},
  morestring=[b]",
  basicstyle=\ttfamily\footnotesize,
  backgroundcolor=\color{bashbg},
  frame=single, rulecolor=\color{framecolor}
}

\pagestyle{headings}

% ============================================================
\begin{document}

\begin{titlepage}
  \vspace*{2cm}
  \begin{center}
    {\LARGE\bfseries OpenJaw Patient-1 Full-Jaw FEM Analysis}\\[0.8em]
    {\large\color{codeblue}%
      Hollow Crown Biofilm around Tooth~23\quad\&\quad%
      Inter-proximal Slit Biofilm between Teeth~30 and~31}\\[0.6em]
    {\large DI-Mapped Transversely Isotropic Material Model}\\[2em]
    \rule{0.75\textwidth}{0.5pt}\\[1.5em]
    {\large IKM Hiwi Project\enspace--\enspace Abaqus 2024\enspace--\enspace\today}\\[0.5em]
    {\small Abaqus/Standard\,+\,Python\,3\,+\,\LaTeX}
  \end{center}
  \vfill
  \tableofcontents
\end{titlepage}

% ============================================================
\section{Overview}
\label{sec:overview}

This document gives a complete account of the methodology and results
for the FEM analysis of dental biofilm modelled on real patient geometry
from the \textbf{Open-Full-Jaw dataset}~\cite{gholamalizadeh2022open} (Patient~1, mandible).
The analysis couples two upstream components:

\begin{enumerate}[nosep]
  \item A five-species oral-biofilm ODE model calibrated with TMCMC,
        which yields a spatially resolved \emph{Dysbiotic Index} (DI) field
        $\phi_\mathrm{DI}(\mathbf{x})$.
  \item Abaqus/Standard structural FEM that maps DI to the local Young's
        modulus via a transversely isotropic material model.
\end{enumerate}

Two anatomically realistic biofilm geometries are investigated:

\begin{description}[nosep,leftmargin=1.8cm]
  \item[Crown (T23)] A hollow annular ring of biofilm surrounding tooth~23
        (upper-left first premolar), extruded over the full clinical crown height.
  \item[Slit (T30--T31)] A thin inter-proximal biofilm slab in the gingival
        pocket between teeth~30 and~31 (lower-left first and second molars).
\end{description}

\begin{figure}[H]
  \centering
  \includegraphics[width=0.78\linewidth]{openjaw1.png}
  \caption{%
    Patient-1 mandible surface mesh imported into Abaqus as an orphan-mesh part
    (triangulated green surface, 476\,KB CAE).
    The mesh was generated from CBCT data in the
    Open-Full-Jaw dataset~\cite{gholamalizadeh2022open}.
    Coordinate axes: $X$ (red), $Y$ (green), $Z$ (blue, superior direction).
    Teeth~23, 30, and~31 are visible as the darker triangulated regions
    on the occlusal surface.
  }
  \label{fig:openjaw1}
\end{figure}

All Python scripts are compatible with Abaqus's embedded CPython interpreter
and require no external mesh libraries.

% ============================================================
\section{Pipeline Architecture}
\label{sec:pipeline}

The full analysis consists of five sequential stages:

\begin{center}
\fbox{\parbox{0.87\textwidth}{\small\ttfamily
\textbf{Stage 1}\enspace\texttt{stl\_bbox.py}
  \hfill STL files $\to$ \texttt{p1\_tooth\_bbox.json}\\[2pt]
\textbf{Stage 2}\enspace\texttt{export\_for\_abaqus.py}
  \hfill TMCMC results $\to$ \texttt{abaqus\_field\_dh\_3d.csv}\\[2pt]
\textbf{Stage 3}\enspace\texttt{openjaw\_p1\_auto\_import.py}
  \hfill STL $\to$ \texttt{OpenJaw\_P1\_auto.cae} (reference only)\\[2pt]
\textbf{Stage 4a}\enspace\texttt{openjaw\_p1\_full\_assembly.py --case crown}
  \hfill $\to$ \texttt{OJ\_Crown\_T23\_b050.odb}\\[2pt]
\textbf{Stage 4b}\enspace\texttt{openjaw\_p1\_full\_assembly.py --case slit}
  \hfill $\to$ \texttt{OJ\_Slit\_T3031\_b050.odb}\\[2pt]
\textbf{Stage 5}\enspace\texttt{compare\_biofilm\_abaqus.py} + \texttt{plot\_oj\_comparison.py}
  \hfill $\to$ figures
}}
\end{center}

\subsection{File inventory}

\begin{table}[H]
\centering
\caption{Key files in \texttt{Tmcmc202601/FEM/}.}
\label{tab:files}
\renewcommand{\arraystretch}{1.12}
\begin{tabular}{>{\ttfamily}lp{9cm}}
\toprule
\normalfont File & Description \\
\midrule
stl\_bbox.py                  & Pure-Python STL bounding-box + 2-D convex-hull extractor. \\
openjaw\_p1\_auto\_import.py  & Imports tooth/mandible STLs into Abaqus as orphan-mesh parts. \\
openjaw\_p1\_biofilm\_solid.py& Per-tooth filled solid biofilm + DI mapping (simple solid). \\
openjaw\_p1\_full\_assembly.py& Hollow ring (T23) + rectangular slit (T30--T31) + job run. \\
export\_for\_abaqus.py        & Exports DI field from TMCMC results to CSV. \\
compare\_biofilm\_abaqus.py   & Extracts $S_\text{Mises}$ at three depth fractions from an ODB. \\
plot\_oj\_comparison.py       & Comparison figures from stress CSV. \\
run\_aniso\_comparison.py     & Orchestrates full anisotropy-ratio sweep across all conditions.\\
\midrule
p1\_tooth\_bbox.json          & Bounding-box + cross-section polygon for T23, T30, T31. \\
abaqus\_field\_dh\_3d.csv     & 3375-point DI field from TMCMC snapshot ($t=0.05$). \\
oj\_crown\_slit\_stress.csv   & Extracted $S_\text{Mises}$ (3 depth fracs) per ODB. \\
OJ\_Crown\_T23\_b050.odb      & Abaqus result: hollow crown, $\beta=0.5$. \\
OJ\_Slit\_T3031\_b050.odb     & Abaqus result: inter-proximal slit, $\beta=0.5$. \\
OpenJaw\_P1\_crown.cae        & Abaqus CAE for crown case. \\
OpenJaw\_P1\_slit.cae         & Abaqus CAE for slit case. \\
\bottomrule
\end{tabular}
\end{table}

\subsection{Three-tooth and full-lower template models}

In addition to the legacy per-case assemblies, two generic template models
are provided for downstream analysis and rapid what-if studies:

\begin{itemize}[nosep]
  \item \textbf{Three-tooth template} (\texttt{openjaw\_p1\_auto\_import.py}):\\
        imports the mandible and teeth~23, 30, and~31 from STL, attaches a
        hollow crown biofilm around tooth~23 and an inter-proximal slit
        biofilm between teeth~30 and~31, and saves
        \texttt{OpenJaw\_P1\_auto\_withbiofilm.cae}. The script also creates
        a job \texttt{OJ\_ThreeTooth\_WithBiofilm} with three steps
        (\texttt{LOAD\_TEMPLATE}, \texttt{LOAD\_CROWN}, \texttt{LOAD\_SLIT})
        and can optionally run it to generate \texttt{.inp} and \texttt{.odb}
        files in a fully non-interactive workflow.

  \item \textbf{All-lower template} (\texttt{openjaw\_p1\_all\_lower\_import.py}):\\
        imports the mandible and all available lower teeth from the Patient-1
        STL directory, uses the same biofilm constructions around tooth~23
        and between teeth~30 and~31, and saves
        \texttt{OpenJaw\_P1\_all\_lower\_withbiofilm.cae}. The corresponding
        job \texttt{OJ\_AllLower\_WithBiofilm} uses the same step structure
        and boundary-condition pattern as the three-tooth template, but on
        the full lower dentition.
\end{itemize}

\subsubsection*{Boundary conditions in the template models}

Both templates share a simple but physically interpretable boundary-condition
scheme:

\begin{itemize}[nosep]
  \item \emph{Mandible fixation} (\texttt{LOAD\_TEMPLATE}): the inferior band
        of the mandible orphan-mesh (faces within $5\%$ of the lowest $z$)
        is clamped by a displacement boundary condition
        ($u_x=u_y=u_z=0$), representing attachment to the surrounding
        jaw tissue and skull.
  \item \emph{Crown loading} (\texttt{LOAD\_CROWN}): the lower end of the
        crown biofilm is fixed in all translational degrees of freedom,
        while a uniform normal pressure of $p=1.0\ \text{MPa}$ is applied
        to the upper end of the biofilm ring, mimicking a distributed
        occlusal contact force.
  \item \emph{Slit loading} (\texttt{LOAD\_SLIT}): the lower edge of the
        inter-proximal slit biofilm is fixed, and the upper edge is loaded
        by the same pressure $p=1.0\ \text{MPa}$ in the local normal
        direction. This models a thin gingival pocket biofilm being
        compressed between two neighbouring molars.
\end{itemize}

In both cases the biofilm is modelled with an effective Young's modulus
of $E_\text{bio}=1.0{\times}10^9\ \text{Pa}$ and Poisson ratio
$\nu_\text{bio}=0.3$, corresponding to a relatively soft, nearly
incompressible material compared with enamel and dentine
($E_\text{tooth}=1.0{\times}10^{10}\ \text{Pa}$).

\subsubsection*{Representative displacements and reaction forces}

To benchmark the mechanical response prior to full-jaw simulations,
simplified single-biofilm test models were analysed:

\begin{itemize}[nosep]
  \item \texttt{dh\_crown\_v3.odb}: isolated crown biofilm ring.
  \item \texttt{dh\_slit\_v3.odb}: isolated inter-proximal slit biofilm.
\end{itemize}

Post-processing with a small Python script (\texttt{analyze\_biofilm\_odb.py})
shows the following indicative values at the end of the loading step:

\begin{itemize}[nosep]
  \item Crown biofilm: maximum von Mises stress
        $S_\text{Mises} \approx 2.0\ \text{MPa}$, maximum displacement
        $\lVert\mathbf{u}\rVert_\text{max} \approx 0.28\ \text{mm}$, and
        total reaction force magnitude
        $\lVert\mathbf{R}\rVert \approx 7.8\times10^{5}\ \text{N}$.
  \item Slit biofilm: maximum von Mises stress
        $S_\text{Mises} \approx 1.5\ \text{MPa}$, maximum displacement
        $\lVert\mathbf{u}\rVert_\text{max} \approx 0.14\ \text{mm}$, and
        total reaction force magnitude
        $\lVert\mathbf{R}\rVert \approx 1.0\times10^{6}\ \text{N}$.
\end{itemize}

These numbers are not intended as direct clinical predictions, but they
provide useful scales for expected strains ($\varepsilon\sim10^{-3}$) and
for checking numerical consistency: the reaction forces are dominated by
the direction of the applied pressure and are of the same order as the
imposed surface traction multiplied by the loaded area.

\subsubsection*{Qualitative comparison: three-tooth vs full-lower}

When the same biofilm constructs are embedded into the three-tooth and
all-lower templates, the surrounding teeth and bone redistribute the
load away from the biofilm solids:

\begin{itemize}[nosep]
  \item In the three-tooth model the biofilm carries a larger fraction of
        the applied load, leading to slightly higher peak stresses and
        displacements, and a more localised deformation pattern.
  \item In the full-lower model the additional teeth and bone provide
        alternative load paths. The biofilm tends to show reduced peak
        stresses and smoother displacement gradients, especially in the
        slit case where the neighbouring molars are better represented.
\end{itemize}

These template models therefore bracket the expected range of mechanical
responses: the three-tooth configuration emphasises local biofilm
deformation, while the full-lower configuration captures global jaw
stiffening and more realistic load sharing across the dentition.

% ============================================================
\section{Geometry Extraction (\texttt{stl\_bbox.py})}
\label{sec:geometry}

\subsection{Algorithm}

The script reads binary or ASCII STL files and computes:
\begin{itemize}[nosep]
  \item Axis-aligned bounding box: $[\mathbf{x}_{\min},\mathbf{x}_{\max}]$,
        centroid $\mathbf{c}$, and size $\mathbf{s}$.
  \item 2-D cross-section polygon at mid-height ($z_\text{frac}=0.5$):
        convex hull of vertices in a $\pm5\%$ height slab, resampled
        to $N_\text{poly}=12$ equi-arc-length points by Graham-scan.
\end{itemize}

The JSON key for each tooth uses the file stem (no extension) to match
Abaqus part names exactly.

\begin{lstlisting}[language=bash,caption={Extracting bounding boxes and cross-section polygons.}]
python3 stl_bbox.py \
    external_tooth_models/OpenJaw_Dataset/Patient_1/Teeth/P1_Tooth_23.stl \
    external_tooth_models/OpenJaw_Dataset/Patient_1/Teeth/P1_Tooth_30.stl \
    external_tooth_models/OpenJaw_Dataset/Patient_1/Teeth/P1_Tooth_31.stl \
    --out p1_tooth_bbox.json --poly-n 12 --z-fraction 0.5
\end{lstlisting}

\subsection{Patient-1 tooth geometry (numerical output)}

All coordinates in mm, in the Open-Full-Jaw global frame~\cite{gholamalizadeh2022open}
(origin at mandibular centroid).
Figure~\ref{fig:openjaw1} shows the imported mandible mesh.

\begin{table}[H]
\centering
\caption{Bounding-box data from \texttt{p1\_tooth\_bbox.json}.}
\label{tab:bbox}
\renewcommand{\arraystretch}{1.1}
\begin{tabular}{lrrrrrr}
\toprule
Tooth & $c_x$ (mm) & $c_y$ (mm) & $c_z$ (mm) &
       $s_x$ (mm) & $s_y$ (mm) & $s_z$ (mm) \\
\midrule
P1\_Tooth\_23 & $-69.11$ & $-41.56$ & $28.28$ & $6.38$ & $9.46$ & $19.46$ \\
P1\_Tooth\_30 & $-40.24$ & $-63.90$ & $28.20$ & $13.07$ & $13.30$ & $20.06$ \\
P1\_Tooth\_31 & $-36.04$ & $-75.68$ & $30.20$ & $14.12$ & $14.47$ & $18.94$ \\
\bottomrule
\end{tabular}
\end{table}

Key derived parameters for the crown biofilm (T23):
\begin{align}
  R_\text{inner} &= \tfrac{1}{2}\min(s_x,s_y) = \tfrac{1}{2}\min(6.38,9.46)
    = 3.19\ \text{mm}, \\
  t_\text{biofilm} &= f_\text{bio}\,R_\text{inner}
    = 0.15 \times 3.19 = 0.48\ \text{mm}, \\
  z_\text{range} &= [18.56,\ 38.01]\ \text{mm},\quad s_z = 19.46\ \text{mm}.
\end{align}

Key derived parameters for the inter-proximal slit (T30--T31):
\begin{align}
  \hat{n} &= (0.332,\ {-}0.930,\ 0.000)\quad\text{(contact-normal unit vector, T30}\to\text{T31)}, \\
  \hat{t} &= (0.930,\ 0.332,\ 0.000)\quad\text{(tangent, buccal--lingual)}, \\
  z_\text{slit} &= [\max(z_{\min}^{30},z_{\min}^{31}),\ \min(z_{\max}^{30},z_{\max}^{31})]
    = [20.73,\ 38.23]\ \text{mm},\\
  h_\text{slit} &= 17.50\ \text{mm}.
\end{align}

% ============================================================
\section{DI Field Export (\texttt{export\_for\_abaqus.py})}
\label{sec:di_field}

The Dysbiotic Index (DI) field is extracted from the TMCMC biofilm simulation
at the final time snapshot (snapshot index $-1$, $t=0.05$):

\begin{lstlisting}[language=bash,caption={Exporting the DI field.}]
python3 export_for_abaqus.py \
    --results-dir _results_3d/dh_baseline \
    --snapshot-index -1 \
    --out-csv abaqus_field_dh_3d.csv
\end{lstlisting}

\textbf{Output:} 3375 rows, columns \texttt{x, y, z, phi\_pg, di, phi\_tot, r\_pg, t}.
DI range at this snapshot: $[0,\ 0.025778]$,
$\phi_\mathrm{DI,max} = 0.025778$ (used as \texttt{di\_scale}).

% ============================================================
\section{Abaqus Model Construction}
\label{sec:model}

\subsection{STL import (\texttt{openjaw\_p1\_auto\_import.py})}

Tooth STLs are imported as TRI3 (S3 shell) orphan-mesh parts for
visualisation and future contact analyses.
These parts are \emph{not} included in the structural analysis models
because Abaqus cannot assign a \texttt{HomogeneousSolidSection}
(for C3D elements) to S3 shell elements.
All positioning information needed for the biofilm solid is taken
from \texttt{p1\_tooth\_bbox.json}.

\begin{lstlisting}[language=bash,caption={Importing tooth STLs (reference only).}]
abaqus cae noGUI=openjaw_p1_auto_import.py -- \
    --stl-root external_tooth_models/OpenJaw_Dataset/Patient_1 \
    --cae-out  OpenJaw_P1_auto.cae
\end{lstlisting}

\subsection{Crown biofilm hollow ring (T23)}

The crown biofilm is a \emph{hollow annular ring} extruded along Z
over the full tooth height $s_z = 19.46$\,mm.

\begin{enumerate}[nosep,leftmargin=1.5cm]
  \item \textbf{Inner polygon}: 12-point cross-section convex hull from JSON
        (\texttt{--poly-from-json}).
  \item \textbf{Outer polygon}: each inner point offset radially outward by
        $t_\text{biofilm} = 0.48$\,mm from centroid $(c_x,c_y)$.
  \item \textbf{Sketch}: both closed loops drawn in one \texttt{ConstrainedSketch};
        Abaqus interprets the inner loop as a through-hole.
  \item \textbf{Extrusion}: \texttt{BaseSolidExtrude(depth=19.46)} mm.
  \item \textbf{Mesh}: global seed size ${\approx}0.29$\,mm $\to$ C3D8R hexahedra.
  \item \textbf{Assembly Z-offset}: translate instance by
        $z_{\min} = 18.56$\,mm to anatomical position.
\end{enumerate}

\begin{lstlisting}[language=bash,caption={Crown case command.}]
abaqus cae noGUI=openjaw_p1_full_assembly.py -- \
    --bbox-json p1_tooth_bbox.json \
    --field-csv abaqus_field_dh_3d.csv \
    --case crown --aniso-ratio 0.5 --poly-from-json \
    --crown-job OJ_Crown_T23_b050 \
    --cae-out   OpenJaw_P1_crown.cae
\end{lstlisting}

\subsection{Inter-proximal slit (T30--T31)}

The slit is a \emph{rectangular box} aligned with the T30$\to$T31
contact-normal vector:

\begin{enumerate}[nosep,leftmargin=1.5cm]
  \item \textbf{Cross-section} (XY plane):
        half-depth $d/2 = 1.5$\,mm along $\hat{n}$;
        half-width $w/2 \approx 6.54$\,mm along $\hat{t}$.
  \item \textbf{Centre}: midpoint between T30 and T31 centroids
        $(m_x, m_y) = (-38.14, -69.80)$\,mm.
  \item \textbf{Extrusion}: \texttt{BaseSolidExtrude(depth=17.50)} mm.
  \item \textbf{Mesh}: global seed size $1.0$\,mm $\to$ C3D8R hexahedra.
  \item \textbf{Assembly Z-offset}: translate by $z_\text{lo} = 20.73$\,mm.
\end{enumerate}

\begin{lstlisting}[language=bash,caption={Slit case command.}]
abaqus cae noGUI=openjaw_p1_full_assembly.py -- \
    --bbox-json p1_tooth_bbox.json \
    --field-csv abaqus_field_dh_3d.csv \
    --case slit --aniso-ratio 0.5 \
    --slit-job OJ_Slit_T3031_b050 \
    --cae-out  OpenJaw_P1_slit.cae
\end{lstlisting}

% ============================================================
\section{Material Model}
\label{sec:material}

\subsection{DI to stiffness mapping}

The stiffness in the gradient direction is mapped from DI by:
\begin{equation}
  E_\text{stiff}(\mathrm{DI}) = E_{\max}\,(1-r)^n + E_{\min}\,r,
  \qquad r = \mathrm{clip}\!\left(\frac{\mathrm{DI}}{\mathrm{DI}_\text{scale}},\,0,1\right),
  \label{eq:E_di}
\end{equation}
with parameters listed in Table~\ref{tab:matparams}.

\begin{table}[H]
\centering
\caption{Material model parameters.}
\label{tab:matparams}
\begin{tabular}{llll}
\toprule
Parameter & Symbol & Value & Unit \\
\midrule
Maximum Young's modulus & $E_{\max}$              & $10.0$ & GPa \\
Minimum Young's modulus & $E_{\min}$              & $0.5$  & GPa \\
Power-law exponent      & $n$                   & $2.0$  & --- \\
DI normalisation scale  & $\mathrm{DI}_\text{scale}$ & $0.025778$ & --- \\
Poisson's ratio         & $\nu$                 & $0.30$ & --- \\
Anisotropy ratio        & $\beta$               & $0.5$  & --- \\
Number of DI bins       & $N_\text{bin}$        & $20$   & --- \\
Applied pressure        & $p$                   & $1.0$  & MPa \\
\bottomrule
\end{tabular}
\end{table}

\subsection{Transverse isotropy}

The biofilm material is transversely isotropic with $E_1 = E_\text{stiff}$
in the gradient direction and $E_2 = E_3 = \beta\,E_\text{stiff}$
in the transverse plane:
\begin{equation}
  (E_1,E_2,E_3,\nu_{12},\nu_{13},\nu_{23},G_{12},G_{13},G_{23})
  =\bigl(E_s,E_t,E_t,\nu,\nu,\nu,G_s,G_s,G_t\bigr),
\end{equation}
\[
  G_s = \frac{E_s}{2(1+\nu)},\quad G_t = \frac{E_t}{2(1+\nu)},\quad E_t = \beta E_s.
\]
$\beta = 1$ recovers isotropy.

\subsection{Element-wise DI bin assignment}

\paragraph{Crown (radial).}
For each element centroid $(\bar x,\bar y)$:
\begin{equation}
  \rho = \sqrt{(\bar x - c_x)^2 + (\bar y - c_y)^2},\qquad
  \rho_\text{norm} = \mathrm{clip}\!\left(\frac{\rho - R_\text{inner}}{t_\text{biofilm}},0,1\right).
\end{equation}
Nearest-neighbour lookup in the field CSV (along its $x$-axis, the depth
coordinate of the biofilm model) maps $\rho_\text{norm}$ to a DI value
$\to$ bin index.

\paragraph{Slit (contact-normal projection).}
\begin{equation}
  d_n = \bigl|(\bar x - m_x)\,n_x + (\bar y - m_y)\,n_y\bigr|,\qquad
  d_\text{norm} = \mathrm{clip}\!\left(\frac{d_n}{d/2},0,1\right),
\end{equation}
followed by the same field-CSV lookup.

\subsection{Bin values ($\beta=0.5$, $\nu=0.30$)}

\begin{table}[H]
\centering\small
\caption{Young's moduli per DI bin.}
\label{tab:dibins}
\begin{tabular}{rrrr|rrrr}
\toprule
Bin & $\bar{\mathrm{DI}}$ & $E_s$ (GPa) & $E_t$ (GPa) &
Bin & $\bar{\mathrm{DI}}$ & $E_s$ (GPa) & $E_t$ (GPa) \\
\midrule
00 & 0.000644 & 9.501 & 4.751 & 10 & 0.013534 & 2.610 & 1.305 \\
01 & 0.001933 & 8.559 & 4.279 & 11 & 0.014823 & 2.217 & 1.109 \\
02 & 0.003222 & 7.677 & 3.839 & 12 & 0.016112 & 1.876 & 0.938 \\
03 & 0.004511 & 6.851 & 3.426 & 13 & 0.017401 & 1.586 & 0.793 \\
04 & 0.005800 & 6.082 & 3.041 & 14 & 0.018690 & 1.347 & 0.674 \\
05 & 0.007089 & 5.367 & 2.684 & 15 & 0.019979 & 1.157 & 0.579 \\
06 & 0.008378 & 4.709 & 2.354 & 16 & 0.021268 & 1.015 & 0.508 \\
07 & 0.009667 & 4.104 & 2.052 & 17 & 0.022557 & 0.922 & 0.461 \\
08 & 0.010956 & 3.552 & 1.776 & 18 & 0.023846 & 0.876 & 0.438 \\
09 & 0.012245 & 3.055 & 1.527 & 19 & 0.025135 & 0.877 & 0.439 \\
\bottomrule
\end{tabular}
\end{table}

% ============================================================
\section{Boundary Conditions}
\label{sec:bc}

\begin{table}[H]
\centering
\caption{Boundary conditions for both cases.}
\label{tab:bc}
\renewcommand{\arraystretch}{1.1}
\begin{tabular}{lllp{6.5cm}}
\toprule
Case & BC name & Location ($z$, mm) & Type \\
\midrule
Crown &
  \texttt{FIX\_CROWN\_BOT} & $18.56$ (bottom) &
  $u_1=u_2=u_3=0$, Initial step \\
Crown &
  \texttt{PRESS\_CROWN} & $38.02$ (top) &
  Uniform pressure $p=1\,\text{MPa}$, \texttt{LOAD\_CROWN} step \\
\midrule
Slit &
  \texttt{FIX\_SLIT\_BOT}  & $20.73$ (bottom) &
  $u_1=u_2=u_3=0$, Initial step \\
Slit &
  \texttt{PRESS\_SLIT}     & $38.23$ (top) &
  Uniform pressure $p=1\,\text{MPa}$, \texttt{LOAD\_SLIT} step \\
\bottomrule
\end{tabular}
\end{table}

Both use \texttt{Abaqus/Standard} static steps:
\texttt{maxNumInc=100}, \texttt{initialInc=0.1}, \texttt{minInc=1e-5}.

% ============================================================
\section{Results}
\label{sec:results}

\subsection{Job completion}

\begin{table}[H]
\centering
\caption{Abaqus job log (2026-02-22, JST).  Both jobs terminated \texttt{COMPLETED}.}
\label{tab:jobs}
\begin{tabular}{lllr}
\toprule
Job name & Start time & End time & ODB size \\
\midrule
\texttt{OJ\_Crown\_T23\_b050}  & 01:40:45 & 01:40:51 & 6.4 MB \\
\texttt{OJ\_Slit\_T3031\_b050} & 01:41:10 & 01:41:13 & 0.85 MB \\
\bottomrule
\end{tabular}
\end{table}

\textbf{Note on earlier failure.}
The first run of \texttt{OJ\_Crown\_T23\_b050} failed with
\emph{``3590 elements have missing property definitions''}.
Root cause: the T23 tooth orphan-mesh (TRI3/S3 shell elements)
was being assigned a \texttt{HomogeneousSolidSection} (solid-only type)
inside the analysis model, which Abaqus rejects.
Fix: removed all \texttt{\_import\_stl\_part()} calls for T23, T30, T31
from \texttt{openjaw\_p1\_full\_assembly.py}.
The biofilm C3D8R solid elements were correctly assigned throughout.

\subsection{Von Mises stress at three depth fractions}

$S_\text{Mises}$ is probed at depth fractions $f\in\{0.0,\,0.5,\,1.0\}$
by \texttt{compare\_biofilm\_abaqus.py}.
For 3-D geometries the script selects the global X-axis as the depth
direction (see limitations in Section~\ref{sec:future}).

\begin{table}[H]
\centering
\caption{%
  Von Mises stress (MPa) at depth fractions 0 (inner), 0.5 (mid), 1.0 (outer).
  Applied pressure = 1.0 MPa.  $\beta=0.5$ unless stated.}
\label{tab:smises}
\renewcommand{\arraystretch}{1.08}
\begin{tabular}{lrrr}
\toprule
ODB / geometry & $f=0.0$ (inner) & $f=0.5$ (mid) & $f=1.0$ (outer) \\
\midrule
\multicolumn{4}{l}{\textit{OpenJaw real-tooth geometries}} \\
\texttt{OJ\_Crown\_T23\_b050} (hollow crown, $\beta=0.5$) & 0.981 & 0.993 & 1.001 \\
\texttt{OJ\_Slit\_T3031\_b050} (slit, $\beta=0.5$)        & 1.082 & 0.800 & 1.082 \\
\addlinespace
\multicolumn{4}{l}{\textit{OpenJaw T23 solid biofilm --- anisotropy sweep}} \\
\texttt{oj\_t23\_b100} ($\beta=1.0$, isotropic)            & 0.958 & 0.963 & 0.966 \\
\texttt{oj\_t23\_b070} ($\beta=0.7$)                       & 0.958 & 0.963 & 0.966 \\
\texttt{oj\_t23\_b050} ($\beta=0.5$)                       & 0.958 & 0.963 & 0.966 \\
\texttt{oj\_t23\_b030} ($\beta=0.3$)                       & 0.958 & 0.963 & 0.966 \\
\addlinespace
\multicolumn{4}{l}{\textit{OpenJaw per-tooth solid, $\beta=0.5$}} \\
\texttt{oj\_t30\_b050} (T30 solid)                         & 1.011 & 1.006 & 0.995 \\
\texttt{oj\_t31\_b050} (T31 solid)                         & 1.014 & 1.008 & 0.994 \\
\addlinespace
\multicolumn{4}{l}{\textit{Idealized geometries, $\beta=0.5$}} \\
\texttt{dh\_cube\_v3} (unit cube)                          & 0.806 & 0.624 & 0.984 \\
\texttt{dh\_crown\_v3} (idealized crown)                   & 0.209 & 1.100 & 0.384 \\
\texttt{dh\_slit\_v3} (idealized slit)                     & 0.896 & 1.046 & 0.551 \\
\bottomrule
\end{tabular}
\end{table}

\subsection{Discussion}

\paragraph{Crown hollow biofilm.}
$S_\text{Mises}$ is nearly uniform across the ring thickness
($0.981\to1.001$\,MPa), close to the applied pressure.
The dominant stress component is $S_{33}\approx-1$\,MPa (axial compression),
confirming the biofilm column transmits load in the extrusion direction.
The slight radial gradient (inner $<$ outer) reflects local bending
at the inner edge.

\paragraph{Inter-proximal slit.}
The symmetric inner/outer stress (1.082\,MPa) with a depressed mid-value
(0.800\,MPa) is consistent with uniform contact-normal pressure on both
tooth-facing faces while the mid-plane is partially in shear.

\paragraph{T23 anisotropy sweep.}
The per-tooth solid biofilm shows negligible $\beta$-sensitivity
($\Delta S_\text{Mises}<0.01$\,MPa across $\beta=0.3\ldots1.0$)
because the probe nodes lie on the global X-axis (buccal--lingual),
not the loading axis (Z).
The DI bin assignments and material orientations are confirmed
correct by inspection of the INP file.

\paragraph{Idealized vs.\ real geometry.}
The idealized unit-cube crown yields 0.21--0.38\,MPa, much lower
than the real hollow crown (0.98--1.00\,MPa).
This is expected: the idealized model uses 1\,mm$^3$ normalised
dimensions, while the real geometry spans 19.46\,mm in height
with a biofilm thickness of 0.48\,mm.

% ============================================================
\section{Figures}
\label{sec:figures}

\begin{figure}[H]
  \centering
  \includegraphics[width=\linewidth]{_aniso_sweep/figures/fig_C2_geometry_comparison.png}
  \caption{%
    \textbf{C2: Geometry comparison.}
    $S_\text{Mises}$ at inner surface ($f=0$, left) and outer surface ($f=1$, right)
    for all 11 ODB variants at $\beta=0.5$, $p=1$\,MPa.
    The horizontal dashed line marks the applied pressure.
    Real-tooth hollow crown (teal) and slit (yellow-green) sit close to
    1\,MPa throughout, while the idealized crown (orange) is substantially lower.
  }
  \label{fig:geom_comparison}
\end{figure}

\begin{figure}[H]
  \centering
  \includegraphics[width=0.65\linewidth]{_aniso_sweep/figures/fig_C2_beta_sweep_t23.png}
  \caption{%
    \textbf{C2: Anisotropy sweep for T23 solid biofilm.}
    $S_\text{Mises}$ at inner (blue) and outer (orange) surfaces as a function
    of $\beta$.  Minimal variation indicates that the global-X probe direction
    does not align with the loading axis; the DI bin assignments are correct
    (verified from the ODB).
  }
  \label{fig:beta_sweep}
\end{figure}

\begin{figure}[H]
  \centering
  \includegraphics[width=0.78\linewidth]{_aniso_sweep/figures/fig_C2_depth_gradient.png}
  \caption{%
    \textbf{C2: Depth gradient profiles.}
    $S_\text{Mises}$ at $f\in\{0,0.5,1\}$ for real-tooth hollow geometries
    (solid lines) vs.\ idealized geometries (dashed lines).
    The hollow crown (teal) and slit (yellow-green) are nearly flat.
    The idealized crown (orange) and slit (blue) show stronger gradients
    due to unit-cube scaling and a different DI-mapping orientation.
  }
  \label{fig:depth_gradient}
\end{figure}

% ============================================================
\section{Known Limitations and Future Work}
\label{sec:future}

\begin{enumerate}
  \item \textbf{Probe-direction mismatch.}
        \texttt{compare\_biofilm\_abaqus.py} uses the global X-axis as
        depth for all 3-D geometries. For Z-loaded crown and slit models
        this maps to the buccal--lingual direction, not the loading direction.
        A geometry-aware probe (radial for crown, Z for slit) should be added.

  \item \textbf{Single $\beta$ value.}
        The hollow crown and slit have been run only at $\beta=0.5$.
        A sweep $\beta\in\{0.3,0.5,0.7,1.0\}$ via \texttt{run\_aniso\_comparison.py}
        is the immediate next step.

  \item \textbf{Tooth--biofilm contact.}
        The biofilm solid is positioned geometrically at the tooth location
        but shares no tied or contact constraint with the tooth surface.
        Adding a tied surface between the inner ring face and the tooth
        surface (after remeshing the tooth in C3D elements) would
        improve physical realism.

  \item \textbf{Boundary condition realism.}
        Fixed-bottom + uniform-top pressure is a first approximation.
        Patient-specific occlusal loads and a periodontal-ligament layer
        should be included in future models.

  \item \textbf{Multi-case both.}
        Running with \texttt{--case both} produces crown and slit in a
        single Abaqus session but shares one CAE database, which may cause
        step-name conflicts.  Running each case separately (as done here)
        is recommended.
\end{enumerate}

% ============================================================
\appendix
\section{CLI Reference: \texttt{openjaw\_p1\_full\_assembly.py}}
\label{app:cli}

\begin{table}[H]
\centering
\caption{All command-line arguments.}
\label{tab:cli}
\renewcommand{\arraystretch}{1.1}
\begin{tabular}{>{\ttfamily}llll}
\toprule
\normalfont Argument & Default & Type & Description \\
\midrule
--bbox-json   & \emph{required} & path  & JSON from \texttt{stl\_bbox.py} \\
--field-csv   & \emph{required} & path  & DI field CSV \\
--case        & both            & str   & \texttt{crown}, \texttt{slit}, or \texttt{both} \\
--biofilm-frac& 0.15            & float & $t_\text{bio}/R_\text{inner}$ \\
--pocket-depth& 3.0             & float & Slit depth in $\hat{n}$ direction (mm) \\
--pocket-width& 2.0             & float & Slit width in $\hat{t}$ direction (mm) \\
--aniso-ratio & 0.5             & float & $\beta = E_t/E_s$ \\
--n-bins      & 20              & int   & Number of DI material bins \\
--e-max       & 10e9            & float & $E_{\max}$ (Pa) \\
--e-min       & 0.5e9           & float & $E_{\min}$ (Pa) \\
--di-scale    & 0.025778        & float & DI normalisation ($\mathrm{DI}_\text{scale}$) \\
--di-exponent & 2.0             & float & Power-law exponent $n$ \\
--nu          & 0.30            & float & Poisson's ratio \\
--pressure-mpa& 1.0             & float & Applied pressure (MPa) \\
--poly-from-json & off          & flag  & Use 12-pt cross-section from bbox JSON \\
--crown-job   & OJ\_Crown\_T23  & str   & Abaqus job name (crown) \\
--slit-job    & OJ\_Slit\_T30T31& str   & Abaqus job name (slit) \\
--cae-out     & OpenJaw\_P1\_assembly.cae & path & Output CAE path \\
\bottomrule
\end{tabular}
\end{table}

\section{Full Reproduction Script}
\label{app:reproduce}

\begin{lstlisting}[language=bash,caption={Complete pipeline from Stage 1 to figures.}]
#!/bin/bash
# Run from: Tmcmc202601/FEM/
ABAQUS=/home/nishioka/DassaultSystemes/SIMULIA/Commands/abaqus

# Stage 1: bounding boxes + cross-section polygons
python3 stl_bbox.py \
    external_tooth_models/OpenJaw_Dataset/Patient_1/Teeth/P1_Tooth_23.stl \
    external_tooth_models/OpenJaw_Dataset/Patient_1/Teeth/P1_Tooth_30.stl \
    external_tooth_models/OpenJaw_Dataset/Patient_1/Teeth/P1_Tooth_31.stl \
    --out p1_tooth_bbox.json --poly-n 12

# Stage 2: DI field export
python3 export_for_abaqus.py \
    --results-dir _results_3d/dh_baseline \
    --snapshot-index -1 --out-csv abaqus_field_dh_3d.csv

# Stage 3: STL import to CAE (reference only)
$ABAQUS cae noGUI=openjaw_p1_auto_import.py -- \
    --stl-root external_tooth_models/OpenJaw_Dataset/Patient_1 \
    --cae-out OpenJaw_P1_auto.cae

# Stage 4a: hollow crown around T23
$ABAQUS cae noGUI=openjaw_p1_full_assembly.py -- \
    --bbox-json p1_tooth_bbox.json \
    --field-csv abaqus_field_dh_3d.csv \
    --case crown --aniso-ratio 0.5 --poly-from-json \
    --crown-job OJ_Crown_T23_b050 \
    --cae-out OpenJaw_P1_crown.cae

# Stage 4b: inter-proximal slit T30-T31
$ABAQUS cae noGUI=openjaw_p1_full_assembly.py -- \
    --bbox-json p1_tooth_bbox.json \
    --field-csv abaqus_field_dh_3d.csv \
    --case slit --aniso-ratio 0.5 \
    --slit-job OJ_Slit_T3031_b050 \
    --cae-out OpenJaw_P1_slit.cae

# Stage 5: extract S_Mises + plots
$ABAQUS python compare_biofilm_abaqus.py \
    oj_crown_slit_stress.csv \
    OJ_Crown_T23_b050.odb OJ_Slit_T3031_b050.odb \
    oj_t23_b100.odb oj_t23_b070.odb oj_t23_b050.odb oj_t23_b030.odb \
    oj_t30_b050.odb oj_t31_b050.odb dh_cube_v3.odb dh_crown_v3.odb dh_slit_v3.odb

python3 plot_oj_comparison.py
pdflatex openjaw_openfulljaw.tex
\end{lstlisting}

% ============================================================
\begin{thebibliography}{9}

\bibitem{gholamalizadeh2022open}
T.~Gholamalizadeh, F.~Moshfeghifar, Z.~Ferguson, T.~Schneider, D.~Panozzo,
S.~Darkner, M.~Makaremi, F.~Chan, P.\,L.~S{\o}ndergaard, and K.~Erleben,
``Open-Full-Jaw: An open-access dataset and pipeline for finite element models
of human jaw,''
\textit{Computer Methods and Programs in Biomedicine},
vol.~224, p.~107009, 2022.

\end{thebibliography}

\end{document}
