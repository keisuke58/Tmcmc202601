\documentclass[11pt,a4paper]{article}
\usepackage[utf8]{inputenc}
% \usepackage[T1]{fontenc}  % Disabled - missing cm-super fonts
\usepackage{amsmath,amssymb}
\usepackage{booktabs}
% \usepackage{graphicx}  % Disabled - missing epstopdf-base.sty
\usepackage{hyperref}
\usepackage{listings}
\usepackage{xcolor}
\usepackage{geometry}
\geometry{margin=2.5cm}

\definecolor{codegreen}{rgb}{0,0.6,0}
\definecolor{codegray}{rgb}{0.5,0.5,0.5}
\definecolor{codepurple}{rgb}{0.58,0,0.82}
\definecolor{backcolour}{rgb}{0.95,0.95,0.92}

\lstdefinestyle{mystyle}{
    backgroundcolor=\color{backcolour},
    commentstyle=\color{codegreen},
    keywordstyle=\color{magenta},
    numberstyle=\tiny\color{codegray},
    stringstyle=\color{codepurple},
    basicstyle=\ttfamily\footnotesize,
    breakatwhitespace=false,
    breaklines=true,
    captionpos=b,
    keepspaces=true,
    numbers=left,
    numbersep=5pt,
    showspaces=false,
    showstringspaces=false,
    showtabs=false,
    tabsize=2
}
\lstset{style=mystyle}

\title{5-Species Biofilm TMCMC Parameter Estimation\\
\large Technical Documentation}
\author{Biofilm Modeling Project}
\date{\today}

\begin{document}

\maketitle

\tableofcontents
\newpage

%==============================================================================
\section{Introduction}
%==============================================================================

This document describes the implementation of Transitional Markov Chain Monte Carlo (TMCMC) for parameter estimation in a 5-species biofilm model. The goal is to estimate 20 model parameters from experimental biofilm composition data.

\subsection{Problem Statement}

Given experimental observations of species volume fractions $\bar{\phi}_i(t)$ at discrete timepoints, estimate the parameter vector $\boldsymbol{\theta} \in \mathbb{R}^{20}$ that best explains the data under a Bayesian framework.

%==============================================================================
\section{Mathematical Model}
%==============================================================================

\subsection{State Variables}

The model tracks the following state variables for each species $i \in \{0,1,2,3,4\}$:
\begin{itemize}
    \item $\phi_i$: Volume fraction of species $i$
    \item $\psi_i$: Viability (fraction of living cells)
    \item $\gamma$: Interface parameter
\end{itemize}

The observable quantity is:
\begin{equation}
    \bar{\phi}_i = \phi_i \cdot \psi_i
\end{equation}
representing the living bacteria volume fraction.

\subsection{Parameter Vector}

The 20-dimensional parameter vector is structured as:
\begin{equation}
    \boldsymbol{\theta} = \begin{pmatrix}
        a_{11}, a_{12}, a_{22}, b_1, b_2 & \text{(M1: Species 0-1)} \\
        a_{33}, a_{34}, a_{44}, b_3, b_4 & \text{(M2: Species 2-3)} \\
        a_{13}, a_{14}, a_{23}, a_{24} & \text{(M3: Cross-terms)} \\
        a_{55}, b_5 & \text{(M4: Species 4 self)} \\
        a_{15}, a_{25}, a_{35}, a_{45} & \text{(M5: Species 4 cross)}
    \end{pmatrix}
\end{equation}

where:
\begin{itemize}
    \item $a_{ij}$: Interaction coefficient between species $i$ and $j$ (symmetric: $a_{ij} = a_{ji}$)
    \item $b_i$: Decay/death rate for species $i$
\end{itemize}

\subsection{Interaction Matrix}

The interaction matrix $\mathbf{A} \in \mathbb{R}^{5 \times 5}$ is:
\begin{equation}
    \mathbf{A} = \begin{pmatrix}
        a_{11} & a_{12} & a_{13} & a_{14} & a_{15} \\
        a_{12} & a_{22} & a_{23} & a_{24} & a_{25} \\
        a_{13} & a_{23} & a_{33} & a_{34} & a_{35} \\
        a_{14} & a_{24} & a_{34} & a_{44} & a_{45} \\
        a_{15} & a_{25} & a_{35} & a_{45} & a_{55}
    \end{pmatrix}
\end{equation}

%==============================================================================
\section{TMCMC Algorithm}
%==============================================================================

\subsection{Overview}

TMCMC constructs a sequence of intermediate distributions:
\begin{equation}
    p_j(\boldsymbol{\theta}) \propto p(\mathbf{y}|\boldsymbol{\theta})^{\beta_j} \cdot p(\boldsymbol{\theta})
\end{equation}
where $0 = \beta_0 < \beta_1 < \cdots < \beta_m = 1$ is the tempering schedule.

\subsection{Algorithm Steps}

\begin{enumerate}
    \item \textbf{Initialize}: Sample $N$ particles from prior $p(\boldsymbol{\theta})$
    \item \textbf{For each stage} $j = 1, \ldots, m$:
    \begin{enumerate}
        \item Compute importance weights: $w_k \propto p(\mathbf{y}|\boldsymbol{\theta}_k)^{\beta_j - \beta_{j-1}}$
        \item Resample particles according to weights
        \item Apply MCMC moves to diversify particles
        \item Adapt proposal covariance from current particles
    \end{enumerate}
    \item \textbf{Output}: Final particles approximate posterior $p(\boldsymbol{\theta}|\mathbf{y})$
\end{enumerate}

\subsection{Adaptive $\beta$ Selection}

The tempering parameter $\beta_j$ is chosen to maintain effective sample size:
\begin{equation}
    \text{ESS}(\beta) = \frac{\left(\sum_k w_k\right)^2}{\sum_k w_k^2} \approx \frac{N}{2}
\end{equation}

%==============================================================================
\section{Likelihood Function}
%==============================================================================

\subsection{Observation Model}

Given model predictions $\bar{\phi}_i^{\text{model}}(t_k)$ at observation times $t_k$, the likelihood is:
\begin{equation}
    p(\mathbf{y}|\boldsymbol{\theta}) = \prod_{k=1}^{K} \prod_{i=1}^{5} \mathcal{N}\left(y_{ik} \,\Big|\, \bar{\phi}_i^{\text{model}}(t_k; \boldsymbol{\theta}), \sigma^2\right)
\end{equation}

where $\sigma$ is the observation noise standard deviation.

\subsection{Log-Likelihood}

\begin{equation}
    \log p(\mathbf{y}|\boldsymbol{\theta}) = -\frac{1}{2\sigma^2} \sum_{k=1}^{K} \sum_{i=1}^{5} \left(y_{ik} - \bar{\phi}_i^{\text{model}}(t_k; \boldsymbol{\theta})\right)^2 + \text{const}
\end{equation}

%==============================================================================
\section{Experimental Data}
%==============================================================================

\subsection{Data Sources}

\begin{table}[h]
\centering
\caption{Available experimental conditions}
\begin{tabular}{llll}
\toprule
Condition & Cultivation & Timepoints (days) & Species \\
\midrule
Commensal & Static & 1, 3, 6, 10, 15, 21 & 5 \\
Commensal & HOBIC & 1, 3, 6, 10, 15, 21 & 5 \\
Dysbiotic & Static & 1, 3, 6, 10, 15, 21 & 5 \\
Dysbiotic & HOBIC & 1, 3, 6, 10, 15, 21 & 5 \\
\bottomrule
\end{tabular}
\end{table}

\subsection{Commensal Static Data Pattern}

\begin{table}[h]
\centering
\caption{Species fractions for Commensal Static condition (\%)}
\begin{tabular}{lccccc}
\toprule
Day & S. oralis & A. naeslundii & V. dispar & F. nucleatum & P. gingivalis \\
\midrule
3  & 70.7 & 8.1  & 20.2 & 0.5 & 0.5 \\
6  & 74.3 & 5.0  & 19.8 & 0.5 & 0.5 \\
10 & 66.0 & 9.4  & 23.6 & 0.5 & 0.5 \\
15 & 36.5 & 24.0 & 38.5 & 0.5 & 0.5 \\
21 & 34.7 & 34.7 & 29.7 & 0.5 & 0.5 \\
\bottomrule
\end{tabular}
\end{table}

\subsection{Key Observations}

\begin{itemize}
    \item \textbf{S. oralis}: Shows non-monotonic behavior (plateau Day 3-6, then decline)
    \item \textbf{A. naeslundii}: Initial dip (Day 3-6), then strong growth
    \item \textbf{V. dispar}: Gradual increase then slight decline
    \item \textbf{F. nucleatum \& P. gingivalis}: Remain at low levels ($\sim$0.5\%)
\end{itemize}

%==============================================================================
\section{Implementation}
%==============================================================================

\subsection{Software Structure}

\begin{lstlisting}[language=bash,caption=Project structure]
data_5species/
├── main/
│   ├── estimate_commensal_static.py  # Main estimation script
│   ├── generate_all_figures.py       # Post-processing
│   └── compare_runs.py               # Run comparison
├── core/                             # TMCMC implementation
├── visualization/                    # Plotting utilities
├── experiment_data/                  # Data files
└── _runs/                           # Output directories
\end{lstlisting}

\subsection{Running Estimation}

\begin{lstlisting}[language=bash,caption=Basic estimation command]
python main/estimate_commensal_static.py \
    --condition Commensal \
    --cultivation Static \
    --n-particles 1000 \
    --n-stages 50 \
    --n-chains 2 \
    --use-exp-init \
    --start-from-day 3 \
    --normalize-data
\end{lstlisting}

\subsection{Key Parameters}

\begin{table}[h]
\centering
\caption{Important command-line arguments}
\begin{tabular}{lll}
\toprule
Argument & Default & Description \\
\midrule
\texttt{--n-particles} & 500 & Number of TMCMC particles \\
\texttt{--n-stages} & 30 & Maximum stages \\
\texttt{--n-chains} & 2 & Parallel chains \\
\texttt{--prior-decay-max} & None & Upper bound for $b_i$ \\
\texttt{--use-exp-init} & False & Use experimental initial conditions \\
\texttt{--normalize-data} & False & Normalize to fractions \\
\bottomrule
\end{tabular}
\end{table}

%==============================================================================
\section{Results and Analysis}
%==============================================================================

\subsection{Fit Metrics}

For each run, the following metrics are computed:

\begin{equation}
    \text{RMSE} = \sqrt{\frac{1}{NK} \sum_{k=1}^{K} \sum_{i=1}^{N} \left(y_{ik} - \hat{y}_{ik}\right)^2}
\end{equation}

\begin{equation}
    \text{MAE} = \frac{1}{NK} \sum_{k=1}^{K} \sum_{i=1}^{N} \left|y_{ik} - \hat{y}_{ik}\right|
\end{equation}

\subsection{Baseline Results (500 particles, 30 stages)}

\begin{table}[h]
\centering
\caption{Fit metrics for Commensal Static baseline run}
\begin{tabular}{lcc}
\toprule
Metric & MAP Estimate & Mean Estimate \\
\midrule
Total RMSE & 0.0614 & 0.1051 \\
Total MAE & 0.0431 & 0.0705 \\
\bottomrule
\end{tabular}
\end{table}

\subsection{Per-Species RMSE}

\begin{table}[h]
\centering
\caption{Per-species RMSE (MAP estimate)}
\begin{tabular}{lc}
\toprule
Species & RMSE \\
\midrule
S. oralis (Species 1) & 0.1105 \\
A. naeslundii (Species 2) & 0.0462 \\
V. dispar (Species 3) & 0.0591 \\
F. nucleatum (Species 4) & 0.0128 \\
P. gingivalis (Species 5) & 0.0294 \\
\bottomrule
\end{tabular}
\end{table}

%==============================================================================
\section{Model Improvements}
%==============================================================================

\subsection{Tighter Decay Priors}

To address over-prediction of species decline, decay parameters can be constrained:

\begin{equation}
    b_i \sim \text{Uniform}(0, 1) \quad \text{instead of} \quad b_i \sim \text{Uniform}(0, 3)
\end{equation}

Command:
\begin{lstlisting}[language=bash]
python main/estimate_commensal_static.py \
    --prior-decay-max 1.0 \
    [other options...]
\end{lstlisting}

\subsection{Increased Sampling}

For better posterior approximation:
\begin{itemize}
    \item Increase particles: 500 $\rightarrow$ 1000
    \item Increase stages: 30 $\rightarrow$ 50
\end{itemize}

%==============================================================================
\section{Run Scripts}
%==============================================================================

\subsection{Available Scripts}

\begin{table}[h]
\centering
\caption{Ready-to-run shell scripts}
\begin{tabular}{ll}
\toprule
Script & Purpose \\
\midrule
\texttt{run\_improved\_estimation.sh} & 1000 particles, 50 stages \\
\texttt{run\_tight\_decay\_estimation.sh} & Decay bounds [0, 1] \\
\texttt{run\_hobic\_estimation.sh} & HOBIC cultivation \\
\bottomrule
\end{tabular}
\end{table}

\subsection{Usage}

\begin{lstlisting}[language=bash,caption=Running in background]
cd /path/to/data_5species
nohup bash run_improved_estimation.sh > improved.log 2>&1 &

# Monitor progress
tail -f improved.log
\end{lstlisting}

%==============================================================================
\section{Conclusion}
%==============================================================================

This implementation provides a flexible framework for Bayesian parameter estimation in the 5-species biofilm model. Key features include:

\begin{itemize}
    \item Multi-chain TMCMC with adaptive tempering
    \item Comprehensive visualization pipeline
    \item Run comparison utilities
    \item Configurable prior bounds for model refinement
\end{itemize}

Future work may include:
\begin{itemize}
    \item Time-varying parameters
    \item Model selection criteria (BIC, DIC)
    \item Observation correlation modeling
\end{itemize}

\end{document}
