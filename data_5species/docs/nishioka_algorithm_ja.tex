\documentclass[11pt,a4paper]{article}

\usepackage{amsmath,amssymb,amsfonts}
\usepackage{booktabs}
\usepackage{array}
\usepackage{colortbl}
\usepackage{xcolor}
\usepackage{tikz}
\usetikzlibrary{arrows.meta,positioning,shapes.geometric}
\usepackage{graphicx}
\usepackage{geometry}
\geometry{margin=2.2cm}
\usepackage{listings}

% Japanese support with XeLaTeX - Unified Mincho style
\usepackage{fontspec}

% Main text: Noto Serif (elegant, academic)
\setmainfont{Noto Serif CJK JP}[
    UprightFont = *-Regular,
    BoldFont = *-Bold,
    Scale = 1.0
]

% Sans-serif (for code comments only)
\setsansfont{Noto Sans CJK JP}[
    UprightFont = *-Regular,
    BoldFont = *-Bold,
    Scale = 0.95
]

% Monospace for code
\setmonofont{Noto Sans Mono CJK JP}[Scale=0.80]

% Colors
\definecolor{accent}{RGB}{0,102,153}
\definecolor{lightaccent}{RGB}{230,242,250}
\definecolor{codegreen}{rgb}{0,0.6,0}
\definecolor{codegray}{rgb}{0.5,0.5,0.5}
\definecolor{codepurple}{rgb}{0.58,0,0.82}
\definecolor{backcolour}{rgb}{0.95,0.95,0.92}

\lstdefinestyle{mystyle}{
    backgroundcolor=\color{backcolour},
    commentstyle=\color{codegreen},
    keywordstyle=\color{magenta},
    numberstyle=\tiny\color{codegray},
    stringstyle=\color{codepurple},
    basicstyle=\ttfamily\scriptsize,
    breakatwhitespace=false,
    breaklines=true,
    captionpos=b,
    keepspaces=true,
    numbers=left,
    numbersep=5pt,
    showspaces=false,
    showstringspaces=false,
    showtabs=false,
    tabsize=2
}

\lstset{style=mystyle}

% Better line breaking for Japanese
\XeTeXlinebreaklocale "ja"
\XeTeXlinebreakskip=0pt plus 1pt minus 0.1pt
\sloppy

\usepackage[hidelinks,pdfencoding=auto]{hyperref}

\title{
    \vspace{-1cm}
    {\Huge\bfseries Nishioka法}\\[0.5cm]
    {\Large 5菌種バイオフィルムモデルのための}\\[0.2cm]
    {\Large 生物学的制約によるパラメータ削減手法}\\[1cm]
    \rule{\textwidth}{0.8pt}
}
\author{
    {\large TMCMCパラメータ推定プロジェクト}\\[0.3cm]
    {\normalsize IKM Hiwi Research Group}
}
\date{{2026年2月}}

\begin{document}

\maketitle
\thispagestyle{empty}

\begin{abstract}
本文書は、5菌種バイオフィルム相互作用モデルのベイズ推定のための
パラメータ削減手法である「Nishioka法」について記述する。
実験的に決定された相互作用ネットワークからの生物学的知見を組み込むことで、
本手法はパラメータ空間を20個から15個の自由パラメータに削減し、
推定効率と生物学的解釈性を向上させる。
\end{abstract}

\tableofcontents
\newpage

%==============================================================================
\section{はじめに}
%==============================================================================

5菌種バイオフィルムモデル\cite{Heine2025PeriImplant}は、
相互作用行列$\mathbf{A}$と減衰ベクトル$\mathbf{b}$を通じて
細菌集団の動態を記述する。
推定はTMCMC\cite{ChingChen2007TMCMC}と
TSMソルバー\cite{GeislerErdoganNagelJunker2025TSM}を組み合わせて実行される。
標準的なアプローチでは全20パラメータを自由に推定するが、
以下の問題が生じうる:

\begin{itemize}
    \item 限られた実験データによる識別可能性の低下
    \item 生物学的に妥当でないパラメータ推定値
    \item 不必要なパラメータ空間の探索による計算効率の低下
\end{itemize}

Nishioka法は、菌種間相互作用が存在しないという実験的証拠に基づいて
特定の相互作用パラメータをゼロに固定することで、これらの問題に対処する。

%==============================================================================
\section{生物学的根拠}
%==============================================================================

\subsection{モデルに含まれる菌種}

本モデルは口腔バイオフィルムに一般的に見られる5つの細菌種を含む:

\begin{table}[h]
\centering
\begin{tabular}{clll}
\toprule
\rowcolor{accent!20}
\textbf{ID} & \textbf{菌種} & \textbf{略称} & \textbf{役割} \\
\midrule
\cellcolor{blue!10} 0 & \textit{Streptococcus oralis} & S.o & 初期定着菌 \\
\cellcolor{green!10} 1 & \textit{Actinomyces naeslundii} & A.n & 初期定着菌 \\
\cellcolor{yellow!10} 2 & \textit{Veillonella} spp. & Vei & 代謝ブリッジ \\
\cellcolor{purple!10} 3 & \textit{Fusobacterium nucleatum} & F.n & ブリッジ微生物 \\
\cellcolor{red!10} 4 & \textit{Porphyromonas gingivalis} & P.g & 後期定着菌(病原菌) \\
\bottomrule
\end{tabular}
\caption{5菌種バイオフィルムモデルに含まれる菌種}
\label{tab:species}
\end{table}

\subsection{相互作用ネットワーク(Figure 4C)}

実験観察\cite{Heine2025PeriImplant}に基づき、以下の相互作用ネットワークが確立された:

\begin{figure}[h]
\centering
\begin{tikzpicture}[
    node distance=2.5cm,
    species/.style={circle, draw, minimum size=1.2cm, font=\small},
    edge/.style={<->, thick, >=Stealth},
    noedge/.style={dashed, gray, thick}
]
    \node[species, fill=blue!20] (so) at (0, 3) {S.o};
    \node[species, fill=green!20] (an) at (-2.5, 1) {A.n};
    \node[species, fill=yellow!20] (vei) at (0, 1) {Vei};
    \node[species, fill=purple!20] (fn) at (2.5, 1) {F.n};
    \node[species, fill=red!20] (pg) at (0, -1.5) {P.g};

    \draw[edge, blue] (so) -- (an) node[midway, left, font=\scriptsize] {共凝集};
    \draw[edge, blue] (so) -- (vei) node[midway, right, font=\scriptsize] {乳酸};
    \draw[edge, blue] (so) -- (fn) node[midway, right, font=\scriptsize] {ギ酸};
    \draw[edge, blue] (vei) -- (pg) node[midway, left, font=\scriptsize] {pH};
    \draw[edge, blue] (fn) -- (pg) node[midway, right, font=\scriptsize] {ペプチド};

    \draw[noedge] (an) -- (vei);
    \draw[noedge] (an) -- (fn);
    \draw[noedge] (vei) -- (fn);
    \draw[noedge] (so) to[bend left=30] (pg);
    \draw[noedge] (an) to[bend right=30] (pg);
\end{tikzpicture}
\caption{Figure 4Cから導出された菌種相互作用ネットワーク。
青い実線は活性相互作用、灰色の破線は不在相互作用を示す。}
\label{fig:network}
\end{figure}

\subsection{活性相互作用}

以下の菌種ペアは直接的な生物学的相互作用を持つ:

\begin{table}[h]
\centering
\small
\begin{tabular}{lll}
\toprule
\textbf{菌種ペア} & \textbf{メカニズム} & \textbf{タイプ} \\
\midrule
S.o $\leftrightarrow$ A.n & 共凝集 & 双方向 \\
S.o $\leftrightarrow$ Vei & 乳酸産生/消費 & 双方向 \\
S.o $\leftrightarrow$ F.n & ギ酸/酢酸共生 & 双方向 \\
Vei $\leftrightarrow$ P.g & pH上昇支援 & 正のみ \\
F.n $\leftrightarrow$ P.g & 共凝集、ペプチド供給 & 双方向 \\
\bottomrule
\end{tabular}
\caption{生物学的メカニズムを持つ活性菌種相互作用}
\label{tab:active}
\end{table}

\subsection{不在相互作用(ロック対象)}

以下の菌種ペアは実験的証拠に基づき直接相互作用を持たない:

\begin{table}[h]
\centering
\small
\begin{tabular}{cclll}
\toprule
\textbf{Index} & \textbf{パラメータ} & \textbf{菌種ペア} & \textbf{行列} & \textbf{理由} \\
\midrule
6 & $a_{34}$ & Vei$\leftrightarrow$F.n & $A[2,3]$ & 代謝経路なし \\
12 & $a_{23}$ & A.n$\leftrightarrow$Vei & $A[1,2]$ & 代謝リンクなし \\
13 & $a_{24}$ & A.n$\leftrightarrow$F.n & $A[1,3]$ & 相互作用なし \\
16 & $a_{15}$ & S.o$\leftrightarrow$P.g & $A[0,4]$ & 相互作用なし \\
17 & $a_{25}$ & A.n$\leftrightarrow$P.g & $A[1,4]$ & 相互作用なし \\
\bottomrule
\end{tabular}
\caption{Nishioka法でゼロに固定される不在相互作用}
\label{tab:locked}
\end{table}

%==============================================================================
\section{数学的定式化}
%==============================================================================

\subsection{支配方程式}

5菌種バイオフィルムモデルは、細菌体積分率$\phi_i$と生存率$\psi_i$の動態を
連立常微分方程式系で記述する。菌種$i$の相互作用項は:
\begin{equation}
I_i = \sum_{j=0}^{4} A_{ij} \phi_j \psi_j
\end{equation}
ここで$A_{ij}$は菌種$j$が菌種$i$に与える影響、$\phi_j \psi_j$は生菌体積分率である。

\subsection{対称行列の仮定}

\textbf{重要な仮定}:相互作用行列$\mathbf{A}$は対称である:
\begin{equation}
A_{ij} = A_{ji} \quad \forall i, j \in \{0, 1, 2, 3, 4\}
\end{equation}

これにより非対角相互作用パラメータ数が20から10に削減される。
例えば、S.o(菌種0)とVei(菌種2)間の乳酸ハンドオーバー相互作用は
単一パラメータで表される:
\begin{equation}
A_{02} = A_{20} = \theta_{10} \quad \text{(コード内では} a_{13} \text{として格納)}
\end{equation}

\subsection{パラメータベクトルの定義}

全20パラメータベクトル$\boldsymbol{\theta}$は5ブロックに整理される:

\begin{align}
\boldsymbol{\theta} &= \underbrace{(a_{11}, a_{12}, a_{22}, b_1, b_2)}_{\text{M1: 菌種1--2}}
\oplus \underbrace{(a_{33}, a_{34}, a_{44}, b_3, b_4)}_{\text{M2: 菌種3--4}} \notag \\
&\quad \oplus \underbrace{(a_{13}, a_{14}, a_{23}, a_{24})}_{\text{M3: クロス項}}
\oplus \underbrace{(a_{55}, b_5)}_{\text{M4: 菌種5}}
\oplus \underbrace{(a_{15}, a_{25}, a_{35}, a_{45})}_{\text{M5: 菌種5クロス}}
\end{align}

\subsection{完全パラメータマッピング}

表~\ref{tab:theta_mapping}は、パラメータインデックスと行列要素の対応を示す。

\begin{table}[h]
\centering
\scriptsize
\begin{tabular}{cclllc}
\toprule
\textbf{Idx} & \textbf{名前} & \textbf{行列} & \textbf{菌種ペア} & \textbf{役割} & \textbf{状態} \\
\midrule
0 & $a_{11}$ & $A[0,0]$ & S.o自己 & 自己調節 & 自由 \\
1 & $a_{12}$ & $A[0,1]$ & S.o$\leftrightarrow$A.n & 共凝集 & 自由 \\
2 & $a_{22}$ & $A[1,1]$ & A.n自己 & 自己調節 & 自由 \\
3 & $b_1$ & $b[0]$ & S.o & 減衰率 & 自由 \\
4 & $b_2$ & $b[1]$ & A.n & 減衰率 & 自由 \\
\midrule
5 & $a_{33}$ & $A[2,2]$ & Vei自己 & 自己調節 & 自由 \\
\rowcolor{red!15}
6 & $a_{34}$ & $A[2,3]$ & Vei$\leftrightarrow$F.n & なし & \textbf{ロック} \\
7 & $a_{44}$ & $A[3,3]$ & F.n自己 & 自己調節 & 自由 \\
8 & $b_3$ & $b[2]$ & Vei & 減衰率 & 自由 \\
9 & $b_4$ & $b[3]$ & F.n & 減衰率 & 自由 \\
\midrule
10 & $a_{13}$ & $A[0,2]$ & S.o$\leftrightarrow$Vei & \textbf{乳酸} & 自由 \\
11 & $a_{14}$ & $A[0,3]$ & S.o$\leftrightarrow$F.n & ギ酸 & 自由 \\
\rowcolor{red!15}
12 & $a_{23}$ & $A[1,2]$ & A.n$\leftrightarrow$Vei & なし & \textbf{ロック} \\
\rowcolor{red!15}
13 & $a_{24}$ & $A[1,3]$ & A.n$\leftrightarrow$F.n & なし & \textbf{ロック} \\
\midrule
14 & $a_{55}$ & $A[4,4]$ & P.g自己 & 自己調節 & 自由 \\
15 & $b_5$ & $b[4]$ & P.g & 減衰率 & 自由 \\
\midrule
\rowcolor{red!15}
16 & $a_{15}$ & $A[0,4]$ & S.o$\leftrightarrow$P.g & なし & \textbf{ロック} \\
\rowcolor{red!15}
17 & $a_{25}$ & $A[1,4]$ & A.n$\leftrightarrow$P.g & なし & \textbf{ロック} \\
18 & $a_{35}$ & $A[2,4]$ & Vei$\leftrightarrow$P.g & \textbf{pH} & 自由$^*$ \\
19 & $a_{45}$ & $A[3,4]$ & F.n$\leftrightarrow$P.g & 共凝集 & 自由 \\
\bottomrule
\end{tabular}
\caption{パラメータマッピング。赤い行はロック対象。$^*$条件により境界が異なる。}
\label{tab:theta_mapping}
\end{table}

\subsection{ロックパラメータインデックス}

Nishioka法では以下のインデックスをゼロに固定する:
\begin{equation}
\mathcal{L} = \{6, 12, 13, 16, 17\}, \quad \theta_k = 0 \text{ for } k \in \mathcal{L}
\end{equation}

\subsection{事前分布の境界}
\label{sec:prior_bounds}

基本事前分布(Commensal/Dysbiotic Static条件):
\begin{equation}
\theta_k \sim \begin{cases}
\text{Uniform}(0, 0) & k \in \mathcal{L} \text{ (ロック)} \\
\text{Uniform}(0, 1) & k = 18 \text{ (Vei→P.g)} \\
\text{Uniform}(-1, 1) & \text{その他}
\end{cases}
\end{equation}

\textbf{Dysbiotic HOBIC}(サージ再現)では:$\theta_{18} \sim \text{Uniform}(-3, -1)$

\subsection{有効パラメータ空間}

自由パラメータの有効数:$n_{\text{free}} = 20 - 5 = 15$

%==============================================================================
\section{実験条件とパラメータ推定}
\label{sec:conditions}
%==============================================================================

パラメータ推定戦略は4つの実験条件に適応する。

\begin{table}[h]
\centering
\small
\begin{tabular}{llccc}
\toprule
\textbf{条件} & \textbf{培養} & \textbf{ロック数} & \textbf{推定数} & \textbf{制約} \\
\midrule
Commensal & Static & 9 & 11 & ゼロ相互作用 \\
Dysbiotic & Static & 5 & 15 & 病原菌推定 \\
Commensal & HOBIC & 8 & 12 & ゼロ相互作用 \\
Dysbiotic & HOBIC & 0 & 20 & \textbf{全解除} \\
\bottomrule
\end{tabular}
\caption{各実験条件のパラメータ推定数}
\label{tab:conditions}
\end{table}

\subsection{詳細なロックロジック}

\begin{enumerate}
    \item \textbf{Commensal Static}:安定したコメンサル状態を再現。
    標準ロック(5個)に加え、後期定着菌の増殖率と相互作用を
    ゼロに固定($N_{locked}=9$)。

    \item \textbf{Dysbiotic Static}:病原菌リッチ状態への移行。
    構造的ロックのみ維持($N_{locked}=5$)。

    \item \textbf{Commensal HOBIC}:HOBICフロー環境に適応。
    病原菌相互作用はロック($N_{locked}=8$)。

    \item \textbf{Dysbiotic HOBIC}(サージモデル):
    \textbf{全ロック解除}($N_{locked}=0$)。サージ現象の再現に必要。
\end{enumerate}

%==============================================================================
\section{実装}
%==============================================================================

\subsection{コアモジュール}

\begin{lstlisting}[language=Python]
LOCKED_INDICES = [6, 12, 13, 16, 17]

def get_nishioka_bounds():
    bounds = [(-1.0, 1.0)] * 20
    for idx in LOCKED_INDICES:
        bounds[idx] = (0.0, 0.0)
    bounds[18] = (0.0, 1.0)  # Vei -> P.g
    return bounds, LOCKED_INDICES
\end{lstlisting}

\subsection{推定スクリプト}

\begin{lstlisting}[language=Python]
from core.nishioka_model import get_nishioka_bounds

nishioka_bounds, LOCKED_INDICES = get_nishioka_bounds()
for idx in LOCKED_INDICES:
    theta_base[idx] = 0.0
active_indices = [i for i in range(20) if i not in LOCKED_INDICES]
\end{lstlisting}

%==============================================================================
\section{比較:標準手法 vs Nishioka法}
%==============================================================================

\begin{table}[h]
\centering
\begin{tabular}{lcc}
\toprule
\textbf{項目} & \textbf{標準} & \textbf{Nishioka法} \\
\midrule
自由パラメータ & 20 & 15 \\
ロックパラメータ & 0 & 5 \\
生物学的制約 & なし & Fig 4Cネットワーク \\
計算コスト & 高い & 低い \\
識別可能性 & 問題あり得る & 改善 \\
\bottomrule
\end{tabular}
\caption{標準手法とNishioka法の比較}
\label{tab:comparison}
\end{table}

%==============================================================================
\section{利点と限界}
%==============================================================================

\subsection{利点}

\begin{enumerate}
    \item パラメータ空間の削減によりMCMCサンプリング効率が向上
    \item 推定値が既知の相互作用ネットワークを尊重
    \item 限られたデータからの識別可能性が向上
    \item パラメータ固定が暗黙の正則化として機能
\end{enumerate}

\subsection{限界}

\begin{enumerate}
    \item 相互作用ネットワークの正確な事前知識が必要
    \item 予期しない相互作用を発見できない
    \item Figure 4Cが不完全な場合、バイアスが生じる可能性
\end{enumerate}

%==============================================================================
\section{使用方法}
%==============================================================================

\begin{lstlisting}[language=Bash]
nohup python main/estimate_reduced_nishioka.py \
    --condition Commensal --cultivation Static \
    --n-particles 2000 --n-stages 30 --n-chains 2 \
    --output-dir _runs/nishioka_v1 > nishioka.log 2>&1 &
\end{lstlisting}

\subsection{出力ファイル}

\begin{table}[h]
\centering
\small
\begin{tabular}{ll}
\toprule
\textbf{ファイル} & \textbf{説明} \\
\midrule
\texttt{config.json} & 実行設定 \\
\texttt{posterior\_samples.csv} & 事後サンプル \\
\texttt{theta\_MAP.json} & MAP推定 \\
\texttt{theta\_MEAN.json} & 事後平均推定 \\
\texttt{fit\_metrics.json} & RMSE、MAE \\
\bottomrule
\end{tabular}
\caption{出力ファイル一覧}
\label{tab:output}
\end{table}

%==============================================================================
\section{結論}
%==============================================================================

Nishioka法は、複雑なバイオフィルムモデルにおけるパラメータ推定に対して
生物学的に根拠のあるアプローチを提供する。
実験的相互作用データを活用してパラメータ空間を制約することで、
計算複雑性の削減と生物学的解釈性の向上を同時に達成する。

%==============================================================================
\bibliographystyle{plain}
\bibliography{related_work}
%==============================================================================

\vfill
\noindent\rule{\textwidth}{0.4pt}
\small{最終更新:2026年2月5日}

\end{document}
