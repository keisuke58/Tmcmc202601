\documentclass[aspectratio=169]{beamer}

\usetheme{Madrid}
\usecolortheme{default}
\setbeamertemplate{navigation symbols}{}

\usepackage{fontspec}
\usepackage{xeCJK}
\IfFontExistsTF{Noto Sans CJK JP}{\setmainfont{Noto Sans CJK JP}}{}
\IfFontExistsTF{Noto Sans CJK JP}{\setsansfont{Noto Sans CJK JP}}{}
\setCJKmainfont{Noto Sans CJK JP}
\setCJKsansfont{Noto Sans CJK JP}
\setCJKmonofont{Noto Sans CJK JP}
% Some minimal xeCJK setups fail to bind the CJK monospaced family; define explicitly.
\setCJKfamilyfont{CJKtt}{Noto Sans CJK JP}
\renewcommand{\CJKttdefault}{CJKtt}
% Monospace: use modern, stylish monospace font (Source Code Pro preferred).
\IfFontExistsTF{Source Code Pro}{
  \setmonofont{Source Code Pro}[
    Scale=0.95,
    BoldFont=Source Code Pro Semibold
  ]
}{
  \IfFontExistsTF{DejaVu Sans Mono}{
    \setmonofont{DejaVu Sans Mono}
  }{}
}
\usepackage{amsmath,amssymb}
\usepackage{graphicx}
\usepackage{booktabs}
\usepackage{tikz}
\usetikzlibrary{arrows.meta,positioning,shapes.geometric,fit}

% --- Optional: best-run figures (auto-picked) ---
\IfFileExists{best_run_id.tex}{% Auto-generated by docs/auto_pick_best_run.py
\def\BestRunId{\detokenize{m1_check_np100_ns15}}
}{\def\BestRunId{\detokenize{m1_check_np100_ns15}}}
\newcommand{\BestRunFig}[1]{../tmcmc/_runs/\BestRunId/figures/#1}
\newcommand{\BestRunAsset}[1]{../tmcmc/_runs/\BestRunId/report_assets/#1}

\title{TMCMC$\times$TSM-ROM(線形化管理 + 解析微分/JIT)}
\subtitle{実装アーキテクチャ/実行フロー/性能・精度の勘所}
\author{Keisuke Nishioka\\IKM\_Hiwi / tmcmc}
\date{\today}

\begin{document}

\begin{frame}
  \titlepage
\end{frame}

\begin{frame}{この資料のゴール}
\begin{itemize}
  \item 入口 \texttt{case2\_tmcmc\_linearization.py} から辿れる実行系を俯瞰
  \item どこが重いか(性能)/何が推定を支配するか(精度)を一枚で説明
  \item 監査可能な再現性ログ(config/likelihood\_meta/diagnostics)を示す
\end{itemize}
\end{frame}

\begin{frame}{要点(論文化/発表向け)}
\begin{itemize}
  \item \textbf{手法}: TMCMC(ESS-tempering)+ TSM-ROM(線形化点$\theta_0$更新)
  \item \textbf{狙い}: 初期は線形化OFFで頑健に探索,後半は線形化ON+$\theta_0$更新でMAP近傍を高速・高精度に
  \item \textbf{監査}: $\beta$=1到達,尤度定義の明文化(\texttt{likelihood\_meta}),診断CSVの保存
\end{itemize}
\end{frame}

\begin{frame}{実行系の全体像(外枠)}
\begin{block}{最短の説明}
\texttt{run\_pipeline.py} が run\_dir を作り,
\texttt{case2\_tmcmc\_linearization.py} を実行した後に \texttt{make\_report.py} で \texttt{REPORT.md} を生成する.
\end{block}
\begin{itemize}
  \item 進捗ログ: \texttt{subprocess.log / pipeline.log}
  \item 監査ログ: \texttt{config.json}, \texttt{likelihood\_meta\_*.json}
\end{itemize}
\end{frame}

\begin{frame}{主要モジュール(実行に効く)}
\begin{itemize}
  \item 入口/制御: \texttt{case2\_tmcmc\_linearization.py}
  \item 設定: \texttt{config.py}
  \item TSM(線形化/JIT): \texttt{demo\_analytical\_tsm\_with\_linearization\_jit.py}
  \item 物理モデル(Newton+時間積分): \texttt{improved1207\_paper\_jit.py}
  \item 解析微分(paper mode): \texttt{paper\_analytical\_derivatives.py}
  \item 診断/レポート: \texttt{mcmc\_diagnostics.py}, \texttt{make\_report.py}
\end{itemize}
\end{frame}

\begin{frame}{モジュール相関(図)}
\centering
\begin{tikzpicture}[
  node/.style={draw,rounded corners,align=left,inner sep=5pt,font=\small},
  arrow/.style={-Latex,thick},
  scale=0.95, every node/.style={transform shape}
]
  \node[node] (case2) {\textbf{case2}\\TMCMC制御\\($\beta$/ESS/変異)};
  \node[node,below=7mm of case2] (eval) {\textbf{Evaluator}\\尤度評価\\(TSM呼び出し)};
  \node[node,below=7mm of eval] (tsm) {\textbf{TSM}\\線形化ON/OFF\\$\theta_0$更新};
  \node[node,below=7mm of tsm] (solver) {\textbf{Newton solver}\\時間積分 + Newton};
  \node[node,right=15mm of tsm] (deriv) {\textbf{Analytical}\\$\partial G/\partial\theta$\\(paper一致)};
  \node[node,right=15mm of case2] (report) {\textbf{Report}\\csv/png/md};

  \draw[arrow] (case2) -- (eval);
  \draw[arrow] (eval) -- (tsm);
  \draw[arrow] (tsm) -- (solver);
  \draw[arrow] (tsm) -- (deriv);
  \draw[arrow] (case2) -- (report);
\end{tikzpicture}
\end{frame}

\begin{frame}{TMCMC($\beta$ tempering)の要点}
\begin{itemize}
  \item prior($\beta$=0)→ posterior($\beta$=1)へ段階的遷移
  \item ステージごとに \(\Delta\beta\) を ESS 目標に合わせて調整(上下限あり)
  \item 重み更新→ESS→リサンプル→mutation(MCMC)で混合を維持
\end{itemize}
\begin{alertblock}{重要チェック}
\textbf{$\beta$=1 に到達していること}(ログに ``beta reached 1.0'' が出るか)
\end{alertblock}
\end{frame}

\begin{frame}{TSM-ROM(線形化管理)の要点}
\begin{block}{線形化近似}
\begin{equation}
x(\theta) \approx x(\theta_0) + \frac{\partial x}{\partial\theta}\Big|_{\theta_0}(\theta-\theta_0)
\end{equation}
\end{block}
\begin{itemize}
  \item 探索初期(小$\beta$): 線形化OFFで非線形性を取り込む
  \item 後半(大$\beta$): 線形化ONで高速化(MAP近傍で精度良)
  \item \texttt{update\_linearization\_point(theta0)} でキャッシュ無効化+再計算
\end{itemize}
\end{frame}

\begin{frame}{いつ線形化するか(概念)}
\centering
\begin{tikzpicture}[
  font=\small,
  arrow/.style={-Latex,thick},
  box/.style={draw,rounded corners,align=center,inner sep=6pt},
]
  \node[box] (early) {探索初期\\small $\beta$\\\textbf{線形化OFF}\\頑健に探索};
  \node[box,right=18mm of early] (late) {後半\\large $\beta$\\\textbf{線形化ON}\\MAP近傍で高速};
  \node[box,below=8mm of late] (upd) {$\theta_0$更新イベント\\(キャッシュ無効化, $x^{(0)},x^{(1)}$再計算)};
  \draw[arrow] (early) -- node[above]{閾値を満たしたら有効化} (late);
  \draw[arrow] (late) -- (upd);
\end{tikzpicture}
\vspace{2mm}
\begin{itemize}
  \item ROM error と $\|\Delta\theta_0\|$ で更新の健全性を監視する.
\end{itemize}
\end{frame}

\begin{frame}{物理モデル(Newton + 時間積分)}
\begin{itemize}
  \item \texttt{run\_deterministic}: 時間方向に進めて各ステップを Newton で解く
  \item \texttt{compute\_Q\_vector}, \texttt{compute\_Jacobian\_matrix}: 残差とヤコビアン
  \item \texttt{alpha\_schedule}: \texttt{switch\_time/switch\_step/switch\_frac} をサポート(M3\_val)
\end{itemize}
\end{frame}

\begin{frame}{解析微分(paper mode)}
\begin{itemize}
  \item \texttt{paper\_analytical\_derivatives.py}: \(\partial G/\partial\theta\) を paper-consistent に実装
  \item complex-step 参照で検証できる(verify 関数あり)
  \item complex-step を成立させる前提: \(\theta\to(A,b)\) が複素dtypeを保持
\end{itemize}
\end{frame}

\begin{frame}{再現性(監査ログ)}
\begin{itemize}
  \item \texttt{config.json}: seed / mode / TMCMC設定 / モデル設定
  \item \texttt{likelihood\_meta\_*.json}: 尤度定義(var\_total 等の内訳)
  \item \texttt{diagnostics\_tables/*.csv}: $\beta$, 受理率, ROM error, $\theta_0$履歴
  \item \texttt{REPORT.md}: しきい値による PASS/WARN/FAIL を自動整理
\end{itemize}
\vspace{3mm}
\small
\textbf{1コマンドパイプライン:}
\begin{quote}\small
\texttt{python tmcmc/run\_pipeline.py --mode paper --seed 123 --run-id paper\_M1\_seed123\_fixed --models M1 --lock-paper-conditions --use-paper-analytical}
\end{quote}
\textbf{論文条件固定:} \texttt{sigma\_obs=0.01}, \texttt{cov\_rel=0.005}
\end{frame}

\begin{frame}{性能(どこが支配的か)}
\begin{itemize}
  \item \textbf{最大}: \texttt{BiofilmTSM\_Analytical.solve\_tsm()}
  \item \textbf{最大}: \texttt{BiofilmNewtonSolver.run\_deterministic()}(内部で Q/J + Newton)
  \item \textbf{大}: 感度 \(x^{(1)}\) 生成(線形化が効かない時期に重い)
  \item \textbf{中}: TMCMC枠($\beta$更新/リサンプル/mutation)
  \item \textbf{小}: 可視化・I/O(条件次第で増える)
\end{itemize}
\end{frame}

\begin{frame}{精度(何が推定を支配するか)}
\begin{itemize}
  \item \textbf{最大}: 尤度定義(\(\sigma_{\mathrm{obs}}\),Var(\(\bar\phi\bar\psi\)) と Cov の扱い)
  \item \textbf{大}: TSM妥当性(ROM error,線形化点管理,解析微分)
  \item \textbf{大}: 線形化ONのタイミングと更新規則(早すぎると壊れる/遅いと遅い)
  \item \textbf{中}: 数値安定化(dt, Newton許容, クリップ/ペナルティ)
  \item \textbf{中}: TMCMC設定(粒子数/ステージ/mutation steps)
\end{itemize}
\end{frame}

\begin{frame}{図のネタ(そのまま発表に使える)}
\begin{itemize}
  \item $\beta$スケジュール(チェーン別)+ ``beta reached 1.0'' の確認
  \item ROM error の更新イベント系列(pre/post)
  \item $\theta_0$更新のステップノルム($\|\Delta\theta_0\|$)
  \item Posterior(M1/M2/M3/M3\_val)と MAP/MEAN fit
  \item Cost--Accuracy tradeoff(FOM評価回数 or wall-time vs MAP error)
\end{itemize}
\end{frame}

\begin{frame}{実例(auto-picked best run)}
\small Best run id: \texttt{\BestRunId}(存在する場合のみ埋め込み)
\vspace{2mm}
\begin{columns}[T,onlytextwidth]
  \column{0.5\textwidth}
  \IfFileExists{\BestRunFig{posterior_M1.png}}{
    \includegraphics[width=\linewidth]{\BestRunFig{posterior_M1.png}}
  }{\fbox{\parbox{\linewidth}{Missing: \texttt{posterior\_M1.png}}}}
  \column{0.5\textwidth}
  \IfFileExists{\BestRunFig{TSM_simulation_M1_MAP_fit_with_data.png}}{
    \includegraphics[width=\linewidth]{\BestRunFig{TSM_simulation_M1_MAP_fit_with_data.png}}
  }{\fbox{\parbox{\linewidth}{Missing: \texttt{MAP\_fit\_with\_data.png}}}}
\end{columns}
\end{frame}

\begin{frame}{診断例(ROM妥当性)}
\begin{columns}[T,onlytextwidth]
  \column{0.5\textwidth}
  \IfFileExists{\BestRunAsset{M1_rom_error.png}}{
    \includegraphics[width=\linewidth]{\BestRunAsset{M1_rom_error.png}}
  }{\fbox{\parbox{\linewidth}{Missing: \texttt{M1\_rom\_error.png}}}}
  \column{0.5\textwidth}
  \IfFileExists{\BestRunAsset{M1_theta0_step_norm.png}}{
    \includegraphics[width=\linewidth]{\BestRunAsset{M1_theta0_step_norm.png}}
  }{\fbox{\parbox{\linewidth}{Missing: \texttt{M1\_theta0\_step\_norm.png}}}}
\end{columns}
\end{frame}

\begin{frame}{まとめ}
\begin{itemize}
  \item 外枠は \texttt{run\_pipeline} → \texttt{case2} → \texttt{make\_report}
  \item 中核コストは \texttt{solve\_tsm} と Newton 時間積分
  \item \textbf{$\beta$=1 到達}と\textbf{尤度定義の明文化}が監査・再現性の肝
\end{itemize}
\end{frame}

\begin{frame}{PASSチェック(結果を主張する前に)}
\begin{alertblock}{最低限の合格条件}
\begin{itemize}
  \item 全チェーンで $\beta$=1 到達(posterior 到達)
  \item \texttt{likelihood\_meta\_*.json} を保存(分散モデルが明示)
  \item solverログに NaN/Inf なし,診断CSVが出力されている
  \item ROM error と $\|\Delta\theta_0\|$ が安定(更新が破綻していない)
\end{itemize}
\end{alertblock}
\end{frame}

\begin{frame}{参考文献}
\footnotesize
\begin{thebibliography}{99}
\bibitem{Fritsch2025BayesianMicrofilms}
L. Fritsch,H. Geisler,J. Grashorn,F. Klempt,M. Soleimani,M. Broggi,P. Junker,M. Beer.
\textit{Bayesian updating of bacterial microfilms under hybrid uncertainties with a novel surrogate model}.
\texttt{tmcmc/Bayesian updating of bacterial microfilms under hybrid uncertainties with a novel surrogate model - Kopie.pdf}.

\bibitem{Klempt2025ContinuumBiofilm}
F. Klempt,H. Geisler,M. Soleimani,P. Junker.
\textit{A continuum multi-species biofilm model with a novel interaction scheme}.arXiv:2509.01274v1,2025.
\texttt{tmcmc/biofilm\_simulation.pdf}.

\bibitem{JunkerBalzani2021ExtendedHamilton}
P. Junker,D. Balzani.
\textit{An extended Hamilton principle as unifying theory for coupled problems and dissipative microstructure evolution}.
Continuum Mechanics and Thermodynamics,2021.DOI: \texttt{10.1007/s00161-021-01017-z}.
\texttt{tmcmc/hamiltonian.pdf}.

\bibitem{Heine2025PeriImplant}
N. Heine et al.
\textit{Influence of species composition and cultivation condition on peri-implant biofilm dysbiosis in vitro}.
Front. Oral Health,2025.DOI: \texttt{10.3389/froh.2025.1649419}.
\texttt{tmcmc/Influence of species composition and cultivation condition on peri-implant biofilm dysbiosis in vitro.pdf}.
\end{thebibliography}
\end{frame}

\end{document}
