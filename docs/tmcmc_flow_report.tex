\documentclass[11pt,a4paper]{article}

\usepackage[margin=25mm]{geometry}
\usepackage{fontspec}
\IfFileExists{microtype.sty}{\usepackage{microtype}}{}
\usepackage[hidelinks]{hyperref}
\IfFileExists{xurl.sty}{\usepackage{xurl}}{} % better line breaks for paths/DOIs
% Custom commands for DOI and paths (stylish monospace)
% Use existing \filefont which is defined later in the file
\newcommand{\doi}[1]{\href{https://doi.org/#1}{\filefont #1}}
% \localpath: handle underscores properly (escape them for LaTeX)
\newcommand{\localpath}[1]{\filefont \detokenize{#1}}
\urlstyle{same}
\hypersetup{breaklinks=true}
\IfFileExists{bookmark.sty}{\usepackage{bookmark}}{}
\IfFileExists{titlesec.sty}{\usepackage{titlesec}}{}

% --- Fonts (robust across minimal Linux environments) ---
% Strategy: Japanese-capable font as main (to avoid missing glyphs),
% then auto-switch ASCII/Latin to Times-compatible serif for English text.

% Main font: Japanese-capable (covers both JP and Latin glyphs).
\IfFontExistsTF{Noto Serif CJK JP}{
  \setmainfont{Noto Serif CJK JP}
}{
  \IfFontExistsTF{Noto Sans CJK JP}{
    \setmainfont{Noto Sans CJK JP}
  }{
    \IfFontExistsTF{Latin Modern Roman}{
      \setmainfont{Latin Modern Roman}
    }{}
  }
}

% English/Latin: Times-compatible serif (Nimbus Roman = Times clone on Linux).
% Prefer OpenType for robust embedding with xdvipdfmx.
\IfFileExists{/usr/share/fonts/urw-base35/NimbusRoman-Regular.otf}{
  \newfontfamily\enfont{NimbusRoman-Regular.otf}[
    Path=/usr/share/fonts/urw-base35/,
    BoldFont=NimbusRoman-Bold.otf,
    ItalicFont=NimbusRoman-Italic.otf,
    BoldItalicFont=NimbusRoman-BoldItalic.otf
  ]
}{
  \IfFontExistsTF{Nimbus Roman}{
    \newfontfamily\enfont{Nimbus Roman}
  }{
    \IfFontExistsTF{DejaVu Serif}{
      \newfontfamily\enfont{DejaVu Serif}
    }{
      \newfontfamily\enfont{\fontname\font}
    }
  }
}

% Monospace: use a commonly available system font if present; otherwise keep default.
\IfFontExistsTF{DejaVu Sans Mono}{\setmonofont{DejaVu Sans Mono}}{}

% Japanese/CJK: explicit handle (same as main when available).
\IfFontExistsTF{Noto Serif CJK JP}{
  \newfontfamily\jpfont{Noto Serif CJK JP}
}{
  \IfFontExistsTF{Noto Sans CJK JP}{
    \newfontfamily\jpfont{Noto Sans CJK JP}
  }{
    \newfontfamily\jpfont{\fontname\font}
  }
}

% Font switching: Use xeCJK for automatic English/Japanese font switching.
% English text will automatically use Times (Nimbus Roman), Japanese will use Noto Serif CJK JP.
\usepackage{xeCJK}
% Set CJK main font (for Japanese text)
\setCJKmainfont{Noto Serif CJK JP}
\setCJKsansfont{Noto Sans CJK JP}
\setCJKmonofont{Noto Sans Mono CJK JP}
% Set English font to Times (Nimbus Roman)
\IfFileExists{/usr/share/fonts/urw-base35/NimbusRoman-Regular.otf}{
  \setmainfont{NimbusRoman-Regular.otf}[
    Path=/usr/share/fonts/urw-base35/,
    BoldFont=NimbusRoman-Bold.otf,
    ItalicFont=NimbusRoman-Italic.otf,
    BoldItalicFont=NimbusRoman-BoldItalic.otf
  ]
}{}
% xeCJK will automatically use the main font (Times) for Latin characters
% and CJK main font (Noto Serif CJK JP) for Japanese characters

% Enable CJK line breaking in XeTeX (prevents long Japanese lines from overflowing).
\XeTeXlinebreaklocale "ja"
\XeTeXlinebreakskip=0pt plus 1pt

% File-name / path font: use Times (Nimbus Roman) to match English text.
\IfFileExists{/usr/share/fonts/urw-base35/NimbusRoman-Regular.otf}{
  \newfontfamily\filefont{NimbusRoman-Regular.otf}[
    Path=/usr/share/fonts/urw-base35/,
    BoldFont=NimbusRoman-Bold.otf,
    ItalicFont=NimbusRoman-Italic.otf,
    BoldItalicFont=NimbusRoman-BoldItalic.otf
  ]
}{
  \IfFontExistsTF{Nimbus Roman}{
    \newfontfamily\filefont{Nimbus Roman}
  }{
    \IfFontExistsTF{Source Code Pro}{
      \newfontfamily\filefont{Source Code Pro}[
        Scale=0.95,
        BoldFont=Source Code Pro Semibold
      ]
    }{
      \newfontfamily\filefont{\fontname\font}
    }
  }
}
% Make \path / \url use the file-name font (instead of monospace).
\def\UrlFont{\filefont}
% Also make \texttt use the same stylish font for consistency.
\let\oldtexttt\texttt
\renewcommand{\texttt}[1]{{\filefont #1}}

% --- Paragraph / heading style (paper-like) ---
\setlength{\parindent}{1em}
\setlength{\parskip}{0.25em}
\renewcommand{\baselinestretch}{1.05}
\setlength{\emergencystretch}{2em} % reduce overfull boxes without harming spacing too much

\IfFileExists{titlesec.sty}{
  \titleformat{\section}{\Large\bfseries}{\thesection}{0.8em}{}
  \titleformat{\subsection}{\large\bfseries}{\thesubsection}{0.8em}{}
  \titleformat{\subsubsection}{\normalsize\bfseries}{\thesubsubsection}{0.8em}{}
  \titlespacing*{\section}{0pt}{2.0ex plus 0.5ex}{1.0ex}
  \titlespacing*{\subsection}{0pt}{1.4ex plus 0.3ex}{0.6ex}
}{}
\usepackage{amsmath,amssymb}
\usepackage{graphicx}
\usepackage{booktabs}
\usepackage{enumitem}
\usepackage{xcolor}
\usepackage{tikz}
\usetikzlibrary{arrows.meta,positioning,shapes.geometric,fit}

% --- Optional: best-run figures (auto-picked) ---
% Generated by: python3 docs/auto_pick_best_run.py
\IfFileExists{best_run_id.tex}{% Auto-generated by docs/auto_pick_best_run.py
\def\BestRunId{\detokenize{m1_check_np100_ns15}}
}{\def\BestRunId{\detokenize{m1_check_np100_ns15}}}
\newcommand{\BestRunFig}[1]{../tmcmc/_runs/\BestRunId/figures/#1}
\newcommand{\BestRunAsset}[1]{../tmcmc/_runs/\BestRunId/report_assets/#1}

\title{TMCMC$\times$TSM-ROM 実装ドキュメント\\\large 線形化管理 + 解析微分/JIT}
\author{Keisuke Nishioka\\Project: IKM\_Hiwi / tmcmc}
\date{\today}

\begin{document}
\maketitle

\section*{1枚でわかる概要(論文化/発表向け)}
\subsection*{何をするか}
本資料は,\textbf{TMCMC(Transitional MCMC)}と\textbf{TSM-ROM(Taylor Series Method reduced-order model)}を統合し,
\textbf{線形化点$\theta_0$の更新}を導入することで,高価な前向き計算(FOM)を伴うベイズ推定を実用的に回すための
実装アーキテクチャと運用上の勘所をまとめる.

\subsection*{貢献(新規性)}
\begin{itemize}[leftmargin=2em]
  \item \textbf{TMCMCの明示的ステージ制御}: ESS 目標に基づく $\Delta\beta$ 更新・リサンプル・$K$-step mutation.
  \item \textbf{TSM-ROMの線形化点管理}: 探索初期は線形化OFFで頑健に探索し,後半で線形化ON+$\theta_0$更新によりMAP近傍の精度と速度を両立.
  \item \textbf{再現性(監査)を成果物で担保}: 設定・尤度定義・診断CSV・ログを run\_dir に自動保存(推論に真値は不要).
\end{itemize}

\subsection*{再現手順(1コマンド)}
\begin{quote}\small
\path{python tmcmc/run_pipeline.py --mode paper --seed 123 --run-id paper_M1_seed123_fixed}
\newline
\path{--models M1 --lock-paper-conditions --use-paper-analytical}
\end{quote}
\textbf{論文条件固定}(\texttt{--lock-paper-conditions} により強制):
\begin{itemize}[leftmargin=2em]
  \item 観測ノイズ: \texttt{sigma\_obs = 0.01}
  \item 相対共分散: \texttt{cov\_rel = 0.005}
  \item 保守的な$\beta$ジャンプ(max\_delta\_beta制限)
\end{itemize}
posterior 到達の最低条件:
\begin{itemize}[leftmargin=2em]
  \item \textbf{$\beta$=1到達}(\texttt{subprocess.log}: ``beta reached 1.0'')
  \item \textbf{尤度定義が永続化}(\texttt{likelihood\_meta\_*.json})
  \item \textbf{診断テーブルが出力}(\texttt{diagnostics\_tables/*.csv})
\end{itemize}

\subsection*{監査・引用のための成果物}
\begin{center}
\begin{tabular}{ll}
\toprule
成果物 & 目的 \\
\midrule
\texttt{config.json} & seed含む全設定(完全再実行) \\
\texttt{likelihood\_meta\_*.json} & 尤度の明示(監査) \\
\texttt{diagnostics\_tables/*.csv} & $\beta$・受理率・ROM error・$\theta_0$履歴 \\
\texttt{subprocess.log}, \texttt{pipeline.log} & provenance/失敗診断 \\
\texttt{figures/*.png} & posterior・フィット図(論文図) \\
\bottomrule
\end{tabular}
\end{center}

\subsection*{論文条件固定runの参照}
論文比較用には、\texttt{--lock-paper-conditions} を使用したrunの \texttt{run-id} を記録する。
図は \texttt{tmcmc/\_runs/<run-id>/figures/*.png} から参照(埋め込み不要)。
例: \texttt{tmcmc/\_runs/paper\_M1\_seed123\_fixed/figures/}

\section{研究背景と動機}
\subsection{問題設定:hybrid uncertainty下でのベイズ推定}
細菌バイオフィルムの成長ダイナミクスを正確にモデル化することは,生医学・環境・産業応用において重要である.
しかし,計算モデルのパラメータ推定には以下の課題がある:
\begin{enumerate}[leftmargin=2em]
  \item \textbf{Epistemic uncertainty(認識的不確かさ)}: パラメータの真値が未知であり,観測データから推論する必要がある.
  \item \textbf{Aleatory uncertainty(偶然的不確かさ)}: 生物学的な内在的変動や環境の確率的影響により,同一条件でも結果が変動する.
  \item \textbf{計算コスト}: 高価な前向き計算(FOM: Full-Order Model)を伴うベイズ推定は,従来のdouble-loop手法では計算量が膨大になる.
\end{enumerate}

\subsection{従来手法の限界}
hybrid uncertainty(epistemic + aleatory)下でのベイズモデル更新(BMU)では,確率的変動を複雑なモデルを通して伝播させる必要があり,
典型的には\textbf{expensive double-loop procedures}(外側ループ:epistemicパラメータ,内側ループ:aleatoryサンプリング)が必要となる\cite{Fritsch2025BayesianMicrofilms}.
これにより,実用的な時間内でのパラメータ推定が困難になる.

\subsection{本研究のアプローチ}
本研究では,以下の統合アプローチにより,hybrid uncertainty下での効率的なベイズ推定を実現する:
\begin{enumerate}[leftmargin=2em]
  \item \textbf{TMCMC(Transitional MCMC)}: $\beta$ temperingにより,priorからposteriorへの段階的遷移を安定化(Ching \& Chen, 2007; Betz et al., 2016).
  \item \textbf{TSM-ROM(Time-Separated Stochastic Mechanics reduced-order model)}: aleatory uncertaintyの伝播をsingle-loopで効率化(Geisler et al., 2023, 2025).
  \item \textbf{線形化点$\theta_0$の動的更新}: TSM-ROMの一次近似精度を維持するため,TMCMCの後半で線形化点をMAP近傍に更新.
  \item \textbf{解析微分 + JIT最適化}: 感度計算を高速化し,実用的な計算時間を実現.
\end{enumerate}

\subsection{本研究の貢献}
\begin{itemize}[leftmargin=2em]
  \item \textbf{理論的貢献}: TMCMCとTSM-ROMを統合し,線形化点更新により精度と効率を両立する手法を提案.
  \item \textbf{実装上の貢献}: ESS目標に基づく$\Delta\beta$制御,K-step mutation,観測ベースの線形化点更新など,実用的な改良を実装.
  \item \textbf{再現性の担保}: 設定・尤度定義・診断CSV・ログを自動保存し,監査可能な推論パイプラインを構築.
\end{itemize}

\section{目的}
\textbf{本資料の目的:}
本資料は,\localpath{tmcmc/case2_tmcmc_linearization.py} をエントリポイントとする
TMCMC(Transitional MCMC)と TSM-ROM(Taylor Series Method reduced-order model)を組み合わせた
実装について,\textbf{研究手法・理論的背景},\textbf{プログラムフロー}と\textbf{主要モジュール境界},ならびに
\textbf{再現性(監査可能なログ)}と\textbf{性能・推定精度の支配要因}を論文レベルの詳細さで整理する.

\section{主要モジュール}
\textbf{目的:} 実行に効く依存関係(エントリポイントから追跡可能で,import により根拠づけられるもの)の最小集合を列挙する.
エントリポイントから到達可能な主要モジュール(import 根拠あり)の最小集合:

\begin{itemize}[leftmargin=2em]
  \item エントリポイント(実験制御): \localpath{tmcmc/case2\_tmcmc\_linearization.py}
  \item 設定: \localpath{tmcmc/config.py}
  \item 物理モデル + TSM(基礎): \localpath{tmcmc/improved1207\_paper\_jit.py}
  \item TSM(線形化点管理 + 解析微分/JIT): \localpath{tmcmc/demo\_analytical\_tsm\_with\_linearization\_jit.py}
  \item 解析微分(paper mode): \localpath{tmcmc/paper\_analytical\_derivatives.py}
  \item 診断: \localpath{tmcmc/mcmc\_diagnostics.py}
  \item $\theta\to(A,b)$ 変換パッチ: \localpath{tmcmc/bugfix\_theta\_to\_matrices.py}
  \item レポート生成(ラン後): \localpath{tmcmc/make\_report.py},実行ラッパ: \localpath{tmcmc/run\_pipeline.py}
\end{itemize}

\subsection{モジュール相関}
\textbf{概念図:}
\begin{center}
\resizebox{0.9\textwidth}{!}{%
\begin{tikzpicture}[
  node/.style={draw,rounded corners,align=left,inner sep=6pt,minimum width=2.6cm,minimum height=1cm,font=\small},
  arrow/.style={-Latex,thick,>=stealth,shorten >=2pt,shorten <=2pt},
  small/.style={font=\footnotesize}
]
  \node[node] (pipeline) {\textbf{run\_pipeline.py}\\\footnotesize subprocess runner\\\footnotesize(run\_dir作成 \& log保持)};
  \node[node,right=14mm of pipeline] (case2) {\textbf{case2\_tmcmc\_linearization.py}\\\footnotesize TMCMC実験制御\\\footnotesize($\beta$更新/再サンプル/変異)};
  \node[node,below=9mm of case2] (eval) {\textbf{LogLikelihoodEvaluator}\\\footnotesize TSM呼び出し\\\footnotesize 線形化ON/OFF制御};
  \node[node,below=9mm of eval] (tsm) {\textbf{BiofilmTSM\_Analytical}\\\footnotesize linearization管理\\\footnotesize 解析微分/JIT};
  \node[node,below=9mm of tsm] (solver) {\textbf{BiofilmNewtonSolver}\\\footnotesize 時間積分\\\footnotesize Newton反復};

  \node[node,right=14mm of tsm] (deriv) {\textbf{paper\_analytical\_derivatives.py}\\\footnotesize $\partial G/\partial\theta$\\\footnotesize paper一致};
  \node[node,right=14mm of case2] (report) {\textbf{make\_report.py}\\\footnotesize config/metrics\\\footnotesize csv/png集約};
  \node[node,below=9mm of report] (diag) {\textbf{mcmc\_diagnostics.py}\\\footnotesize ESS, R-hat\\\footnotesize 診断指標};

  \draw[arrow] (pipeline) -- (case2);
  \draw[arrow] (case2) -- (eval);
  \draw[arrow] (eval) -- (tsm);
  \draw[arrow] (tsm) -- (solver);
  \draw[arrow] (tsm) -- (deriv);
  \draw[arrow] (pipeline) -- (report);
  \draw[arrow] (report) -- (diag);

  \node[small,below=6mm of solver,align=center] (note1)
    {性能支配要素:\\\footnotesize\texttt{solve\_tsm}\\\footnotesize\texttt{run\_deterministic}\\\footnotesize x1(感度)};
\end{tikzpicture}%
}
\end{center}

\section{研究手法と実装戦略}
\subsection{全体アーキテクチャ}
本研究の実装は,以下の3層アーキテクチャで構成される:
\begin{enumerate}[leftmargin=2em]
  \item \textbf{パイプライン層}: 実験実行とレポート生成の自動化(\localpath{tmcmc/run_pipeline.py})
  \item \textbf{推論層}: TMCMCによるベイズ推定(\localpath{tmcmc/case2_tmcmc_linearization.py})
  \item \textbf{モデル層}: TSM-ROMと物理モデル(\localpath{tmcmc/demo_analytical_tsm_with_linearization_jit.py}, \localpath{tmcmc/improved1207_paper_jit.py})
\end{enumerate}

\subsection{アルゴリズムの流れ}
\subsubsection{Phase 1: 初期探索(線形化OFF)}
\begin{enumerate}[leftmargin=2em]
  \item 事前分布から初期粒子を生成
  \item $\beta$が小さい間(例:$\beta < 0.3$)は線形化をOFFにし,フルTSMで頑健に探索
  \item 各ステージでESS目標に基づき$\Delta\beta$を更新
  \item 重み付きリサンプリングとK-step mutationで多様性を維持
\end{enumerate}

\subsubsection{Phase 2: 精密化(線形化ON + $\theta_0$更新)}
\begin{enumerate}[leftmargin=2em]
  \item $\beta$が十分大きくなったら(例:$\beta \geq 0.3$)線形化をONに切り替え
  \item 一定間隔(例:3ステージごと)で線形化点$\theta_0$を更新
  \item ROM errorを監視し,閾値を超えたらFOMで検証
  \item $\beta=1.0$に到達するまで継続
\end{enumerate}

\subsection{最適化戦略}
\subsubsection{JIT(Just-In-Time)コンパイル}
感度計算やNewton反復の核心部分をNumba JITでコンパイルし,計算速度を20--50倍向上させる.
初回呼び出し時にコンパイル(約5秒)が発生するが,以降は高速に実行される.

\subsubsection{キャッシュ戦略}
線形化点$\theta_0$が更新されない間,$x^{(0)}(\theta_0)$ と $x^{(1)}$ をキャッシュし,
同一$\theta_0$での複数評価を高速化する.
$\theta_0$更新時にキャッシュを無効化し,再計算をトリガーする.

\subsubsection{並列化の可能性}
現状は逐次実行だが,粒子ごとの尤度評価は独立であるため,並列化によりさらなる高速化が可能.
ただし,線形化点更新のタイミング制御に注意が必要.

\section{実行フローとプログラム詳細}
\textbf{目的:} ラン全体の呼び出し順と,責務の境界(どこで何が決まるか)を把握する.

\subsection{パイプライン層:run\_pipeline}
\textbf{概要:}
\localpath{tmcmc/run_pipeline.py} は以下の責務を持つ:
\begin{enumerate}[leftmargin=2em]
  \item \textbf{run\_dir作成}: タイムスタンプベースの一意なディレクトリを作成
  \item \textbf{実験実行}: \path{case2_tmcmc_linearization.py} をsubprocessで実行
  \item \textbf{ログ永続化}: stdout/stderrを \path{subprocess.log} にteeし,run\_dir直下に保存
  \item \textbf{レポート生成}: 実験完了後,\path{make_report.py} を呼び出し \path{REPORT.md} を生成
\end{enumerate}
これにより,1コマンドで実験からレポート生成まで自動化される.

\subsection{推論層:case2\_tmcmc\_linearization}
\textbf{TMCMC本体:}
\localpath{tmcmc/case2_tmcmc_linearization.py} の \path{run_TMCMC} 関数が,TMCMCアルゴリズムの核心を実装する.

\subsubsection{初期化フェーズ}
\begin{enumerate}[leftmargin=2em]
  \item \textbf{事前分布からの粒子生成}: $N$個の粒子 $\theta_i^{(0)}$ を事前分布 $p(\theta)$ から独立にサンプリング
  \item \textbf{初期尤度評価}: 各粒子で $\log p(D\mid\theta_i^{(0)})$ を計算(TSM-ROM呼び出し)
  \item \textbf{$\beta=0$の設定}: 初期状態は事前分布に相当
\end{enumerate}

\subsubsection{ステージ反復($s=1,2,\dots,S$)}
各ステージ $s$ で以下の処理を実行:

\textbf{Step 1: $\Delta\beta$の決定}
\begin{itemize}[leftmargin=2em]
  \item 現在の$\beta_s$と粒子の尤度値から,target ESS比(例:0.5)を満たす$\Delta\beta$を二分探索で求める
  \item min\_delta\_beta と max\_delta\_beta でクリップ
  \item $\beta_{s+1} = \min(\beta_s + \Delta\beta, 1.0)$ を設定
\end{itemize}

\textbf{Step 2: 重み更新とリサンプリング}
\begin{itemize}[leftmargin=2em]
  \item 重み $w_i \propto \exp((\beta_{s+1}-\beta_s)\log p(D\mid\theta_i^{(s)}))$ を計算
  \item 数値安定性のため,$\max_i \log p(D\mid\theta_i)$ でシフト
  \item 正規化:$w_i \leftarrow w_i / \sum_j w_j$
  \item ESS = $1 / \sum_i w_i^2$ を計算
  \item 重み付きリサンプリング(systematic resampling)で粒子を複製
\end{itemize}

\textbf{Step 3: K-step Mutation}
\begin{itemize}[leftmargin=2em]
  \item リサンプリング後の粒子集合から経験共分散 $\hat{\Sigma}$ を計算
  \item Tempered scaling: $\Sigma = \hat{\Sigma} \cdot (2.38^2/n) \cdot (1/\max(\beta_{s+1}, 0.1))$
  \item 各粒子に対して$K$回(例:$K=5$)のMetropolis-Hastingsステップを実行
  \item 受理確率はtempered posterior $\pi_{\beta_{s+1}}$ に基づく
\end{itemize}

\textbf{Step 4: 線形化点更新(条件付き)}
\begin{itemize}[leftmargin=2em]
  \item $\beta_{s+1} \geq$ linearization\_threshold(例:0.3)かつ,更新間隔(例:3ステージ)が経過したら実行
  \item 現在の粒子集合からMAPを計算,または観測ベースの重み付けで線形化点候補を選択
  \item \path{LogLikelihoodEvaluator.update_linearization_point($\theta_0$)} を呼び出し
  \item キャッシュを無効化し,次回のTSM評価で再計算をトリガー
\end{itemize}

\textbf{重要チェック}: 各ステージ終了時に,$\beta$が1.0に到達したかを確認(ログに ``beta reached 1.0'' が出力される).

\subsection{モデル層:TSM-ROM実装}
\textbf{線形化管理 + 解析微分/JIT:}
\localpath{tmcmc/demo_analytical_tsm_with_linearization_jit.py} の
\path{BiofilmTSM_Analytical} クラスが,TSM-ROMの核心を実装する.

\subsubsection{solve\_tsm()の動作モード}
\path{BiofilmTSM_Analytical.solve_tsm($\theta$)} は,線形化の有効/無効に応じて以下の2つのモードで動作する:

\textbf{モード1: 線形化OFF(探索初期)}
\begin{itemize}[leftmargin=2em]
  \item $\beta$が小さい,または線形化が無効化されている場合
  \item フルTSM(非線形)で評価:決定論的軌道 $x^{(0)}(\theta)$ を計算し,感度 $x^{(1)}$ をcomplex-stepまたは解析微分で計算
  \item 出力の平均と分散を直接計算:$\mu = x^{(0)}$, $\Sigma = x^{(1)} \cdot \mathrm{Cov}[\theta] \cdot (x^{(1)})^{\mathsf{T}}$
  \item 頑健だが計算コストが高い
\end{itemize}

\textbf{モード2: 線形化ON(収束期)}
\begin{itemize}[leftmargin=2em]
  \item $\beta$が十分大きく,線形化が有効化されている場合
  \item 一次近似を用いて高速化:
  \begin{equation}
    x(\theta) \approx x(\theta_0) + \left.\frac{\partial x}{\partial\theta}\right|_{\theta_0}(\theta-\theta_0)
  \end{equation}
  \item $x^{(0)}(\theta_0)$ と $x^{(1)}$ はキャッシュから取得($\theta_0$が更新されていない限り)
  \item 計算コストが大幅に削減されるが,$\theta$が$\theta_0$から離れると精度が低下
\end{itemize}

\subsubsection{update\_linearization\_point()の実装}
\path{BiofilmTSM_Analytical.update_linearization_point($\theta_{\mathrm{new}}$)} は以下の処理を実行:
\begin{enumerate}[leftmargin=2em]
  \item 新しい線形化点 $\theta_0 \leftarrow \theta_{\mathrm{new}}$ を保存
  \item キャッシュを無効化:\path{_deterministic_solution_cached = None}, \path{_x1_cached = None}
  \item 次回の \path{solve_tsm()} 呼び出し時に,$x^{(0)}(\theta_0)$ と $x^{(1)}$ を再計算
  \item 更新履歴を記録(診断用)
\end{enumerate}

\subsubsection{解析微分の統合}
\path{paper_analytical_derivatives.py} の \path{compute_dG_dtheta_numba} 関数が,
物理モデルの強形式から直接 $\partial G/\partial\theta$ を計算する.
これはcomplex-stepと比較検証され,誤差が $10^{-12}$ 以下であることを確認している.

\subsection{物理モデル層:FOM実装}
\textbf{Newton反復 + 時間積分:}
\localpath{tmcmc/improved1207_paper_jit.py} の \path{BiofilmNewtonSolver} クラスが,
物理モデル(FOM: Full-Order Model)の核心を実装する.

\subsubsection{run\_deterministic()の処理フロー}
\path{BiofilmNewtonSolver.run_deterministic($\theta$)} は以下の処理を実行:
\begin{enumerate}[leftmargin=2em]
  \item \textbf{パラメータ変換}: $\theta$(14次元)を相互作用行列 $A$ と抗生物質感受性 $b$ に変換(\path{theta_to_matrices})
  \item \textbf{初期状態設定}: 体積分率 $\phi_i$ と生存率 $\psi_i$ の初期値を設定
  \item \textbf{時間積分ループ}: 各時間ステップ $t_n$ で以下を実行:
  \begin{itemize}
    \item 残差ベクトル $Q(g^{n+1})$ を計算(\path{compute_Q_vector})
    \item ヤコビアン行列 $J = \partial Q/\partial g$ を計算(\path{compute_Jacobian_matrix})
    \item Newton反復:$J \cdot \Delta g = -Q$ を解き,$g^{n+1} \leftarrow g^n + \Delta g$ を更新
    \item 収束判定:$\|Q\| < \epsilon$ まで反復
  \end{itemize}
  \item \textbf{時間依存項の処理}: 抗生物質スケジュール(\path{alpha_schedule})により,時間に応じて $\alpha(t)$ を切り替え
\end{enumerate}

\subsubsection{残差$Q$とヤコビアン$J$の計算}
残差 $Q$ は,論文の強形式(16)--(18)をimplicit Eulerで離散化したものである:
\begin{itemize}[leftmargin=2em]
  \item \path{compute_Q_vector}: 相互作用項 $(A\bar{\phi})_i$,散逸項 $\eta_i(\dot\phi_i\psi_i^2+\bar{\phi}_i\dot\psi_i+\dot\phi_i)$,制約項 $\gamma$ を組み合わせ
  \item \path{compute_Jacobian_matrix}: $Q$の各成分を状態変数 $g$ で偏微分
  \item JITコンパイル(Numba)により,計算速度を大幅に向上
\end{itemize}

\subsubsection{モデル設定(M1/M2/M3/M3\_val)}
\localpath{tmcmc/config.py} の \path{MODEL_CONFIGS} により,以下のモデル設定が定義される:
\begin{itemize}[leftmargin=2em]
  \item \textbf{M1}: 2種モデル(5パラメータ)--- 成長・相互作用・抗生物質感受性
  \item \textbf{M2}: 4種モデル(10パラメータ)--- M1を2つの独立な2種系に拡張
  \item \textbf{M3}: 4種モデル(14パラメータ)--- 種間相互作用を追加
  \item \textbf{M3\_val}: M3 + 時間依存抗生物質スケジュール(検証用)
\end{itemize}
各モデルは,\path{alpha_schedule} により抗生物質の適用タイミングを制御できる.

\section{理論的背景}
\subsection{ベイズモデル更新の基礎}
観測データ $D$ に基づいて未知パラメータ $\theta$ を推論する問題を考える.
ベイズの定理により,事後分布は
\[
  p(\theta\mid D) \propto p(D\mid\theta)\,p(\theta)
\]
で与えられる.ここで $p(\theta)$ は事前分布,$p(D\mid\theta)$ は尤度関数である\cite{BeckKatafygiotis1998BMU}.

\subsection{Hybrid uncertainty下での尤度構築}
hybrid uncertainty(epistemic + aleatory)下では,モデル出力が確率的に変動する.
TSM-ROMにより,出力の平均 $\mu(\theta) = \mathbb{E}[x(\theta)]$ と分散 $\Sigma(\theta) = \mathrm{Cov}[x(\theta)]$ を直接計算できるため,
ガウシアン尤度
\begin{equation}
  p(D\mid\theta) = \mathcal{N}(D; \mu(\theta), \Sigma(\theta) + \Sigma_{\mathrm{obs}})
\end{equation}
を構築できる($\Sigma_{\mathrm{obs}}$ は観測ノイズの共分散行列)\cite{Fritsch2025BayesianMicrofilms}.
これにより,double-loopを回避し,single-loopでベイズ推定が可能になる.

\subsection{TMCMC:Temperingによる段階的遷移}
\subsubsection{中間分布の定義}
TMCMCは,温度パラメータ $\beta\in[0,1]$ による中間分布
\[
  \pi_{\beta}(\theta) \propto p(\theta)\,p(D\mid\theta)^{\beta}
\]
を導入し,$\beta=0$(事前分布)から $\beta=1$(事後分布)へ段階的に遷移させる\cite{ChingChen2007TMCMC}.
このtemperingにより,多峰性や鋭いピークがある場合でも,安定した探索が可能になる.

\subsubsection{重み更新とEffective Sample Size(ESS)}
ステージ $s$ から $s+1$ への更新では,粒子 $\theta_i^{(s)}$ に対して重みを
\begin{equation}
  w_i \propto \exp\!\left((\beta_{s+1}-\beta_s)\,\log p(D\mid \theta_i^{(s)})\right)
\end{equation}
で定義する.正規化後,Effective Sample Size(ESS)は
\begin{equation}
  \mathrm{ESS} = \frac{1}{\sum_{i=1}^{N} w_i^2}
\end{equation}
で計算される.ESSは粒子の有効サンプル数を表し,重みの退化(weight collapse)を監視する指標となる.

\subsubsection{$\Delta\beta$の適応的決定}
本実装では,\textbf{target ESS比}(例:0.5 = 50\%)を満たす $\Delta\beta=\beta_{s+1}-\beta_s$ を二分探索で求める.
さらに,以下の制約を課す:
\begin{itemize}[leftmargin=2em]
  \item \textbf{min\_delta\_beta}: 最小進行幅(進行が遅すぎるのを防ぐ)
  \item \textbf{max\_delta\_beta}: 最大進行幅(大きな$\beta$ジャンプによるweight collapseを防ぐ)
\end{itemize}
これにより,安定したtemperingスケジュールを実現する(\localpath{tmcmc/case2_tmcmc_linearization.py}: \path{run_TMCMC}).

\subsubsection{リサンプリングとK-step Mutation}
重み付きリサンプリングにより高重み粒子を複製すると,粒子が相関・重複する.
これを解消するため,\textbf{K-step mutation}(ランダムウォーク Metropolis-Hastings)で多様性を回復する.
受理確率は
\[
  \alpha(\theta_{\mathrm{old}}, \theta_{\mathrm{new}}) = \min\left(1, \frac{\pi_{\beta_{s+1}}(\theta_{\mathrm{new}})\,q(\theta_{\mathrm{old}}\mid\theta_{\mathrm{new}})}{\pi_{\beta_{s+1}}(\theta_{\mathrm{old}})\,q(\theta_{\mathrm{new}}\mid\theta_{\mathrm{old}})}\right)
\]
で計算される.ここで $q(\cdot\mid\cdot)$ は提案分布(経験共分散ベース)である.

\subsubsection{Tempered Covariance Scaling}
提案分布のスケーリングは,以下の2つの要因を考慮する:
\begin{enumerate}[leftmargin=2em]
  \item \textbf{最適スケール}: $2.38^2/n$($n$はパラメータ次元数)\cite{GelmanRobertsGilks1996OptimalScaling}
  \item \textbf{Tempered scaling}: $\beta$が小さい(探索期)ほど大きな分散で探索,$\beta$が大きい(収束期)ほど小さな分散で精密化
\end{enumerate}
これにより,各ステージで適切な探索・収束のバランスを取る.

\subsection{TSM-ROM:Time-Separated Stochastic Mechanics}
\subsubsection{TSMの基本思想}
TSMは,確率的パラメータ $\theta$ の周りでモデル応答 $x(\theta)$ をTaylor展開し,
時間と確率的成分を分離して効率的に伝播する手法である\cite{GeislerJunker2023TSM,GeislerErdoganNagelJunker2025TSM}:
\begin{equation}
  x(\theta) = x^{(0)}(\theta_0) + x^{(1)}(\theta_0)\cdot(\theta-\theta_0) + \mathcal{O}(\|\theta-\theta_0\|^2)
\end{equation}
ここで $x^{(0)}$ は決定論的軌道,$x^{(1)}$ は感度(sensitivity)である.

\subsubsection{一次線形化による高速化}
線形化ONモードでは,一次近似
\[
  x(\theta) \approx x(\theta_0) + \left.\frac{\partial x}{\partial\theta}\right|_{\theta_0}(\theta-\theta_0)
\]
を用いて尤度評価を高速化する.ただし,この近似は $\theta_0$ の近傍でのみ有効である.

\subsubsection{線形化点$\theta_0$の動的更新}
固定 $\theta_0$ の一次近似は,posteriorが事前から大きく移動する場合,系統誤差を生む.
本実装では,\textbf{TMCMCの後半($\beta$が十分大きい段階)で線形化点を更新}し,MAP近傍で近似を貼り直す:
\begin{enumerate}[leftmargin=2em]
  \item 各ステージでMAP(Maximum A Posteriori)を計算
  \item 一定間隔(例:3ステージごと)で $\theta_0 \leftarrow \theta_{\mathrm{MAP}}$ に更新
  \item キャッシュを無効化し,$x^{(0)}(\theta_0)$ と $x^{(1)}$ を再計算
\end{enumerate}
これにより,精度と速度を両立する(\localpath{tmcmc/demo_analytical_tsm_with_linearization_jit.py}: \path{BiofilmTSM_Analytical.update_linearization_point}).

\subsubsection{観測ベースの線形化点選択}
本実装では,単純なMAPだけでなく,\textbf{観測ベースの重み付け}により線形化点を選択する:
\begin{itemize}[leftmargin=2em]
  \item ROM error(FOMとROMの差)を各観測点で計算
  \item ROM errorが大きい観測点ほど重みを下げ,精度の高い領域を優先
  \item 重み付けされた重み付き平均(weighted barycenter)を線形化点候補とする
\end{itemize}
これにより,多峰性posteriorでも適切な線形化点を選択できる.

\subsection{解析微分による感度計算}
\subsubsection{Complex-step differentiation}
感度 $x^{(1)} = \partial x/\partial\theta$ の計算には,従来complex-step differentiationが用いられる:
\begin{equation}
  \frac{\partial x}{\partial\theta_i} \approx \frac{\Im[x(\theta + i\epsilon e_i)]}{\epsilon}
\end{equation}
ただし,計算コストが高い.

\subsubsection{解析微分の導入}
本実装では,\localpath{tmcmc/paper_analytical_derivatives.py} により,
物理モデルの強形式から直接 $\partial G/\partial\theta$ を解析的に計算する.
これにより,complex-stepと同等の精度を保ちながら,計算速度を大幅に向上させる.
解析微分はcomplex-stepと比較検証され,誤差が十分小さいことを確認している(\path{verify_against_complex_step}).

\subsection{理論的背景の詳細化}
\subsubsection{TMCMCの収束理論とESS基準の正当性}
TMCMCにおけるESS(Effective Sample Size)基準の有効性は,重みの退化(weight collapse)を防ぐという観点から理論的に保証される.

\textbf{重み退化のメカニズム:}
ステージ $s$ から $s+1$ への更新で,重みは
\[
w_i^{(s+1)} \propto \exp\!\left((\beta_{s+1}-\beta_s)\,\log p(D\mid\theta_i^{(s)})\right)
\]
で更新される.$\Delta\beta = \beta_{s+1}-\beta_s$ が大きすぎると,尤度の高い粒子の重みが極端に大きくなり,
他の粒子の重みがほぼ0になる(weight collapse).これにより,ESSが急激に減少し,サンプルの多様性が失われる.

\textbf{ESS基準の理論的根拠:}
ESSは重みの分散の逆数として定義される:
\[
\mathrm{ESS} = \frac{1}{\sum_{i=1}^{N} w_i^2} = \frac{(\sum_{i=1}^{N} w_i)^2}{\sum_{i=1}^{N} w_i^2}
\]
これは,重み付きサンプルの実効的なサンプル数を表す.ESSが小さい(例:$< 0.3N$)場合,
重みの集中により,事実上少数の粒子しか寄与していないことを意味する.

\textbf{適応的$\Delta\beta$決定の理論的保証:}
target ESS比(例:0.5)を満たす$\Delta\beta$を二分探索で求めることで,
各ステージで適切な重み分布を維持できる.これは,Ching \& Chen (2007) の理論的枠組みに基づく.

\subsubsection{線形化誤差の理論的上界}
一次線形化による近似誤差は,Taylor展開の剰余項により評価できる.

\textbf{線形化誤差の評価:}
TSMの一次近似
\begin{equation}
x(\theta) \approx x(\theta_0) + \left.\frac{\partial x}{\partial\theta}\right|_{\theta_0}(\theta-\theta_0)
\end{equation}
の誤差は,Taylor展開の剰余項により
\begin{equation}
\mathcal{E}(\theta) = \mathcal{O}(\|\theta-\theta_0\|^2)
\end{equation}
で評価される.より正確には,Lipschitz連続性を仮定すると,
\begin{equation}
\|x(\theta) - x(\theta_0) - J(\theta_0)(\theta-\theta_0)\| \leq \frac{L}{2}\|\theta-\theta_0\|^2
\end{equation}
となる($L$はHessianの最大固有値の上界).

\textbf{動的更新の理論的正当性:}
線形化点を$\theta_0$から$\theta_0'$に更新すると,新しい誤差は
\begin{equation}
\mathcal{E}'(\theta) = \mathcal{O}(\|\theta-\theta_0'\|^2)
\end{equation}
となる.posteriorが$\theta_0'$の周りに集中している場合,
$\|\theta-\theta_0'\| < \|\theta-\theta_0\|$ となるため,誤差が減少する.
これが,線形化点をMAP近傍に更新することで精度が向上する理論的根拠である.

\subsubsection{重み退化の数学的説明}
重み退化(weight collapse)は,重みのエントロピーが減少することで定量的に評価できる.

\textbf{重みのエントロピー:}
重み分布のエントロピーは
\[
H(w) = -\sum_{i=1}^{N} w_i \log w_i
\]
で定義される.重みが均一($w_i = 1/N$)の場合,$H(w) = \log N$(最大).
重みが1つの粒子に集中($w_k=1, w_{i\neq k}=0$)の場合,$H(w) = 0$(最小).

\textbf{ESSとエントロピーの関係:}
Jensenの不等式により,
\begin{equation}
H(w) \leq \log(\mathrm{ESS})
\end{equation}
が成り立つ.つまり,ESSが小さいほど,重みのエントロピーも小さく,重みが集中していることを意味する.

\textbf{重み退化の防止:}
target ESS比(例:0.5)を維持することで,$H(w) \geq \log(0.5N)$ を保証し,
重みの多様性を維持できる.

\subsubsection{提案分布の最適化理論}
Metropolis-Hastingsにおける提案分布のスケーリングは,受理率と探索効率のトレードオフを最適化する問題である.

\textbf{最適スケール $2.38^2/n$ の理論的根拠:}
Gelman et al. (1996) により,$n$次元のガウシアン提案分布において,
受理率が約0.44になるスケールが最適であることが示された.
この最適スケールは,経験共分散$\hat{\Sigma}$に対して
\[
\Sigma_{\mathrm{prop}} = \frac{2.38^2}{n} \hat{\Sigma}
\]
で与えられる.この係数$2.38^2/n$は,ランダムウォークの拡散係数を最適化した結果である.

\textbf{Tempered scalingの理論的根拠:}
TMCMCでは,$\beta$が小さい(探索期)ほど大きな分散で探索し,
$\beta$が大きい(収束期)ほど小さな分散で精密化する:
\begin{equation}
\Sigma_{\mathrm{prop}} = \frac{2.38^2}{n} \cdot \frac{1}{\max(\beta, 0.1)} \hat{\Sigma}
\end{equation}
これは,tempered posterior $\pi_{\beta}$ が$\beta$に応じて変化するため,
最適な提案分布も$\beta$に応じて調整する必要があることを反映している.

\subsubsection{Hamilton原理から強形式への導出}
本実装の物理モデルは,拡張Hamilton原理に基づいて導出される.

\textbf{拡張Hamilton原理:}
散逸を含む系では,拡張Hamilton原理により,内部変数の進化則が
\[
\frac{\partial H}{\partial q} + \frac{\partial \Delta_s}{\partial \dot{q}} + \lambda \frac{\partial f}{\partial q} = 0
\]
で与えられる.ここで,$H$はHamiltonian,$\Delta_s$は散逸ポテンシャル,
$f$は制約,$\lambda$はラグランジュ乗数である.

\textbf{強形式(16)--(18)の導出:}
biofilmモデルでは,自由エネルギー$\Psi$と散逸ポテンシャル$\Delta_s$を
\begin{align}
\Psi(\phi,\psi) &= -\frac12 c^{*} \bar{\phi}^{\mathsf{T}} A\,\bar{\phi} + \frac12 \alpha^{*} \psi^{\mathsf{T}} B\,\psi\\
\Delta_s(\dot{\bar\phi},\dot\phi) &= \frac12 \dot{\bar{\phi}}^{\mathsf{T}}\eta\,\dot{\bar{\phi}} + \frac12 \dot{\phi}^{\mathsf{T}}\eta\,\dot{\phi}
\end{align}
と定義し,体積制約 $f(\phi)=\sum_l\phi_l-1=0$ をラグランジュ乗数$\gamma$で課す.
変分計算により,強形式(16)--(18)が導出される(詳細は\localpath{tmcmc/HAMILTON_VALIDATION.md}参照).

\textbf{実装との対応:}
実装の残差$Q$は,強形式(16)--(18)をimplicit Eulerで離散化し,
さらに数値安定化のためのbarrier項を加えたものである.
検証テスト(\localpath{tmcmc/test_hamilton_model_consistency.py})により,
barrier項を無効化した場合,実装の残差が論文の離散化と一致することが確認されている.

\section{実装と理論の対応(ファイル/関数)}
\textbf{目的:} 「どの理論を,どの関数が実装しているか」を監査・説明しやすい形で固定する.

\renewcommand{\arraystretch}{1.2}
\begin{center}\small
\begin{tabular}{p{0.28\linewidth}p{0.25\linewidth}p{0.41\linewidth}}
\toprule
ファイル & 主要関数/クラス & 対応する理論(役割) \\
\midrule
\localpath{tmcmc/case2_tmcmc_linearization.py} &
\path{run_TMCMC}(内部で$\beta$更新・重み・resample・mutation) &
TMCMC(tempering による中間分布 \(\pi_{\beta}\)),ESS基準の$\Delta\beta$選択(二分探索),重み付きリサンプリング(systematic),K-step MCMC rejuvenation(tempered covariance scaling) \\
\addlinespace
\localpath{tmcmc/demo_analytical_tsm_with_linearization_jit.py} &
\path{BiofilmTSM_Analytical.solve_tsm}, \path{update_linearization_point} &
TSM-ROMの一次線形化 \(x(\theta)\approx x(\theta_0)+J(\theta_0)(\theta-\theta_0)\),線形化点更新による局所近似の貼り直し,キャッシュ管理(無効化・再計算トリガー) \\
\addlinespace
\localpath{tmcmc/improved1207_paper_jit.py} &
\path{BiofilmNewtonSolver.run_deterministic}, \path{compute_Q_vector}, \path{compute_Jacobian_matrix} &
強形式(16)--(18)の時間離散化(implicit Euler)+Newton法で \(Q(g^{n+1})=0\) を解く(物理モデルの「真値」FOM),JIT最適化による高速化 \\
\addlinespace
\localpath{tmcmc/paper_analytical_derivatives.py} &
\path{compute_dG_dtheta_numba}, \path{verify_against_complex_step} &
解析微分 \(\partial G/\partial\theta\)(paper一致)と complex-step による参照微分での検証(感度計算の正当性担保,誤差 $<10^{-12}$) \\
\addlinespace
\localpath{tmcmc/mcmc_diagnostics.py} &
\path{MCMCDiagnostics.compute_rhat}, \path{compute_ess} &
MCMC診断(Gelman-Rubin R-hat,ESS):TMCMCでは理論前提が完全には満たれないため \textbf{参考指標}として扱う(収束判定は$\beta=1.0$到達を優先) \\
\addlinespace
\localpath{tmcmc/run_pipeline.py} / \localpath{tmcmc/make_report.py} &
\path{_run_and_tee}, レポート生成関数群 &
再現性(provenance)と監査可能性:設定・尤度定義・診断CSV・ログの永続化と集約,1コマンドでの実験→レポート生成の自動化 \\
\bottomrule
\end{tabular}
\end{center}

\subsection{主要関数の詳細仕様}
\subsubsection{run\_TMCMC()の引数と戻り値}
\textbf{引数}:
\begin{itemize}[leftmargin=2em]
  \item \path{log_likelihood}: 尤度関数($\theta \mapsto \log p(D\mid\theta)$)
  \item \path{prior_bounds}: 各パラメータの事前分布の上下限
  \item \path{n_particles}: 粒子数(例:100--500)
  \item \path{n_stages}: 最大ステージ数(例:15--20)
  \item \path{target_ess_ratio}: 目標ESS比(例:0.5)
  \item \path{min_delta_beta}, \path{max_delta_beta}: $\Delta\beta$の上下限
  \item \path{evaluator}: \path{LogLikelihoodEvaluator}インスタンス(線形化点更新用)
  \item \path{update_linearization_interval}: 線形化点更新間隔(例:3ステージ)
\end{itemize}

\textbf{戻り値} (\path{TMCMCResult}):
\begin{itemize}[leftmargin=2em]
  \item \path{samples}: 最終ステージの粒子集合(posteriorサンプル)
  \item \path{log_likelihoods}: 各粒子の尤度値
  \item \path{MAP}: Maximum A Posteriori推定値
  \item \path{beta_schedule}: $\beta$の時系列
  \item \path{converged}: $\beta=1.0$に到達したかのフラグ
\end{itemize}

\subsubsection{BiofilmTSM\_Analytical.solve\_tsm()の内部処理}
\begin{enumerate}[leftmargin=2em]
  \item \textbf{線形化チェック}: \path{_linearization_enabled} フラグを確認
  \item \textbf{線形化OFFの場合}:
  \begin{itemize}
    \item \path{solver.run_deterministic($\theta$)} で決定論的軌道を計算
    \item 感度 $x^{(1)}$ をcomplex-stepまたは解析微分で計算
    \item 平均と分散を直接計算
  \end{itemize}
  \item \textbf{線形化ONの場合}:
  \begin{itemize}
    \item キャッシュから $x^{(0)}(\theta_0)$ と $x^{(1)}$ を取得(なければ計算)
    \item 一次近似で $x(\theta)$ を推定
    \item 分散は $x^{(1)} \cdot \mathrm{Cov}[\theta] \cdot (x^{(1)})^{\mathsf{T}}$ で計算
  \end{itemize}
  \item \textbf{出力}: 時間配列 $t$,平均軌道 $x^{(0)}$,分散 $\Sigma$
\end{enumerate}

\section{数学的整合性チェック}
\textbf{目的:} Hamilton原理 $\to$ 強形式 $\to$ 実装残差,の対応を式レベルで確認する.
本実装の物理モデル(\texttt{tmcmc/improved1207\_paper\_jit.py})が
\texttt{biofilm\_simulation.pdf} のHamilton原理に基づく導出と整合しているかを,式レベルで確認した.
本節の焦点は「連続体モデルの厳密な証明」ではなく,\textbf{論文の強形式(16)--(18)を材料点 + implicit Eulerにより離散化した系が,コードの残差 $Q$ と一致する}ことの確認である.

\subsection{論文の定義}
\textbf{モデルの骨格:}
内部変数を体積分率 $\phi_i$ と生存率 $\psi_i$ とし,生菌量を $\bar{\phi}_i=\phi_i\psi_i$ と置く.
体積制約(holonomic constraint)は
\[
  f(\phi)=\sum_{l=0}^{n}\phi_l-1=0
\]
で,ラグランジュ乗数 $\gamma$ により課される.
自由エネルギー密度と散逸は(論文の(10),(14))
\begin{align}
  \Psi(\phi,\psi) &= -\frac12 c^{*} \bar{\phi}^{\mathsf{T}} A\,\bar{\phi} + \frac12 \alpha^{*} \psi^{\mathsf{T}} B\,\psi,\\
  \Delta_s(\dot{\bar\phi},\dot\phi) &= \frac12 \dot{\bar{\phi}}^{\mathsf{T}}\eta\,\dot{\bar{\phi}} + \frac12 \dot{\phi}^{\mathsf{T}}\eta\,\dot{\phi}
\end{align}
($\eta$ は対角,$A$ は相互作用行列,$B$ は抗生物質感受性の対角行列)として与えられる.

\subsection{強形式(16)--(18)と,実装残差 $Q$ の対応}
論文ではHamilton原理の評価により,各種$i$について(強形式)
\begin{align}
0 &= -c^{*}\psi_i(A\bar{\phi})_i + \eta_i(\dot\phi_i\psi_i^2+\bar{\phi}_i\dot\psi_i+\dot\phi_i) + \gamma, \\
0 &= -c^{*}\phi_i(A\bar{\phi})_i + \alpha^{*} b_i\psi_i + \eta_i(\dot\psi_i\phi_i^2+\bar{\phi}_i\dot\phi_i), \\
0 &= \sum_{l=0}^{n}\phi_l-1.
\end{align}
実装は材料点モデルとして時間離散化(implicit Euler: $\dot x \approx (x^{n+1}-x^n)/\Delta t$)し,
各ステップで Newton 法により \textbf{$Q(g^{n+1})=0$} を解く.
具体的には,
\begin{itemize}[leftmargin=2em]
  \item \textbf{相互作用項}: $(A\bar{\phi})_i$ は \texttt{Interaction = A @ (phi * psi)}
  \item \textbf{制約}: $\sum\phi+\phi_0-1=0$ は \texttt{Q[9] = sum(phi) + phi0 - 1}
  \item \textbf{$\gamma$ の入り方}: 制約が $\phi$ のみに依存するため,\textbf{$\psi$ 方程式には $\gamma$ が入らない}(これは数学的に必須)
\end{itemize}
が成立する.
また実装は $0<\phi,\phi_0,\psi<1$ を保つために \textbf{barrier項(ペナルティ係数 $K_p$)}を付加している(論文でも言及).

\subsection{自動検証}
\textbf{回帰試験:}
上の「式と実装の一致」を崩さないため,以下をpytestで自動チェックしている:
\begin{itemize}[leftmargin=2em]
  \item barrier を無効化($K_p=0$)したとき,論文強形式(16)--(18)の離散化と実装の残差 $Q$ が一致
  \item \textbf{$\psi$ 方程式が $\gamma$ に依存しない}(制約の形から必須)
\end{itemize}
テストは \texttt{tmcmc/test\_hamilton\_model\_consistency.py} にあり,詳細ノートは \texttt{tmcmc/HAMILTON\_VALIDATION.md} にまとめた.

\section{尤度・不確かさモデル}
\textbf{目的:} 推定精度に影響する尤度定義(ノイズ・分散モデル)を,監査可能な形で明文化する.
推定精度への寄与が最も大きいのは\textbf{尤度定義}(観測ノイズ $\sigma_{\mathrm{obs}}$,分散モデル)である.
特に \textbf{Var$(\bar\phi\bar\psi)$ に Cov$(\bar\phi,\bar\psi)$ を入れる/入れない}は
推定結果を大きく変えるため,runごとの \texttt{likelihood\_meta\_*.json}(または同等ログ)で定義を明文化して監査可能にする.

\paragraph{Cov が効く直感(短く)}
観測量が $\bar\phi\bar\psi$ のような積であっても,\(\bar\phi\) と \(\bar\psi\) の揺らぎを相関あり/なしで扱うかにより,
尤度の「自信度(分散)」が系統的に変わり,posterior の幅やMAP近傍の位置が変わり得る.

\paragraph{注意}
\textbf{論文条件との差:}
実行コマンドの \texttt{--sigma-obs} が 0.02 などの場合,論文で多い 0.01 とは\textbf{定量的一致が一般に保証されない}(バグではない).

\section{再現性}
\textbf{目的:} 監査・再実行に必要なログ/成果物を run\_dir に揃える.
run\_dir に最低限残すべきもの:
\begin{itemize}[leftmargin=2em]
  \item \texttt{config.json}: 実行条件・seed・TMCMC設定・モデル設定
  \item \texttt{likelihood\_meta\_*.json}: 尤度の定義(例: var\_total の内訳)
  \item \texttt{diagnostics\_tables/*.csv}: $\beta$スケジュール,受理率,ROM error,$\theta_0$更新履歴
  \item \texttt{subprocess.log / pipeline.log}: 進捗と「beta reached 1.0」等の重要メッセージ確認
\end{itemize}

\section{詳細フローチャート}
\textbf{目的:} アルゴリズム全体の処理フローを詳細に可視化し,各ステップでの判定条件と分岐を明確にする.

\subsection{TMCMC全体フローチャート}
\begin{center}
\resizebox{0.9\textwidth}{!}{%
\begin{tikzpicture}[
  node/.style={draw,rounded corners,align=center,inner sep=4pt,minimum width=2cm,minimum height=0.8cm,font=\footnotesize},
  decision/.style={diamond,draw,align=center,inner sep=3pt,font=\footnotesize},
  arrow/.style={-Latex,thick,shorten >=2pt,shorten <=2pt},
  start/.style={ellipse,draw,fill=green!20,align=center,font=\footnotesize},
  end/.style={ellipse,draw,fill=red!20,align=center,font=\footnotesize}
]
  \node[start] (start) {開始};
  \node[node,below=6mm of start] (init) {初期化\\事前分布から\\$N$個の粒子生成};
  \node[node,below=6mm of init] (beta0) {$\beta_0 = 0$\\初期尤度評価};
  \node[decision,below=7mm of beta0] (loop) {ステージ\\$s=1,2,\ldots$};
  
  \node[node,right=16mm of loop] (delta) {$\Delta\beta$決定\\(二分探索)};
  \node[node,below=6mm of delta] (beta) {$\beta_{s+1} = \min(\beta_s + \Delta\beta, 1.0)$};
  \node[node,below=6mm of beta] (weight) {重み更新\\$w_i \propto \exp((\beta_{s+1}-\beta_s)\log L_i)$};
  \node[node,below=6mm of weight] (ess) {ESS計算\\$\mathrm{ESS} = 1/\sum w_i^2$};
  \node[node,below=6mm of ess] (resample) {重み付き\\リサンプリング};
  \node[node,below=6mm of resample] (mutation) {K-step\\Mutation};
  \node[decision,below=7mm of mutation] (lin_check) {$\beta \geq$\\閾値?};
  
  \node[node,left=16mm of lin_check] (lin_update) {線形化点\\更新};
  \node[decision,below=7mm of lin_update] (beta1) {$\beta = 1.0$?};
  
  \node[end,below=7mm of beta1] (end) {終了\\Posterior取得};
  
  \draw[arrow] (start) -- (init);
  \draw[arrow] (init) -- (beta0);
  \draw[arrow] (beta0) -- (loop);
  \draw[arrow] (loop) -- node[above,font=\tiny] {続行} (delta);
  \draw[arrow] (delta) -- (beta);
  \draw[arrow] (beta) -- (weight);
  \draw[arrow] (weight) -- (ess);
  \draw[arrow] (ess) -- (resample);
  \draw[arrow] (resample) -- (mutation);
  \draw[arrow] (mutation) -- (lin_check);
  \draw[arrow] (lin_check) -- node[above,font=\tiny] {Yes} (lin_update);
  \draw[arrow] (lin_check) -- node[left,font=\tiny] {No} (beta1);
  \draw[arrow] (lin_update) -- (beta1);
  \draw[arrow] (beta1) -- node[left,font=\tiny] {No} (loop);
  \draw[arrow] (beta1) -- node[right,font=\tiny] {Yes} (end);
\end{tikzpicture}%
}
\end{center}

\subsection{線形化点更新の判定フロー}
\begin{center}
\resizebox{0.85\textwidth}{!}{%
\begin{tikzpicture}[
  node/.style={draw,rounded corners,align=center,inner sep=4pt,minimum width=2cm,font=\footnotesize},
  decision/.style={diamond,draw,align=center,inner sep=3pt,font=\footnotesize},
  arrow/.style={-Latex,thick,shorten >=2pt,shorten <=2pt}
]
  \node[decision] (beta_check) {$\beta \geq$\\linearization\_threshold?};
  \node[decision,below=7mm of beta_check] (interval) {更新間隔\\(例:3ステージ)\\経過?};
  \node[decision,below=7mm of interval] (rom_check) {ROM error\\$<$ 閾値?};
  
  \node[node,right=18mm of rom_check] (map) {MAP計算\\または\\観測ベース選択};
  \node[node,below=6mm of map] (update) {$\theta_0$更新\\キャッシュ無効化};
  \node[node,below=6mm of update] (enable) {線形化ON\\(初回のみ)};
  
  \node[node,left=18mm of rom_check] (skip) {更新スキップ};
  
  \draw[arrow] (beta_check) -- node[left,font=\tiny] {No} (skip);
  \draw[arrow] (beta_check) -- node[right,font=\tiny] {Yes} (interval);
  \draw[arrow] (interval) -- node[left,font=\tiny] {No} (skip);
  \draw[arrow] (interval) -- node[right,font=\tiny] {Yes} (rom_check);
  \draw[arrow] (rom_check) -- node[above,font=\tiny] {Yes} (map);
  \draw[arrow] (rom_check) -- node[left,font=\tiny] {No} (skip);
  \draw[arrow] (map) -- (update);
  \draw[arrow] (update) -- (enable);
\end{tikzpicture}%
}
\end{center}

\subsection{尤度評価の内部フロー}
\begin{center}
\resizebox{0.9\textwidth}{!}{%
\begin{tikzpicture}[
  node/.style={draw,rounded corners,align=center,inner sep=4pt,minimum width=2.2cm,font=\footnotesize},
  decision/.style={diamond,draw,align=center,inner sep=3pt,font=\footnotesize},
  arrow/.style={-Latex,thick,shorten >=2pt,shorten <=2pt}
]
  \node[decision] (lin_enabled) {線形化\\有効?};
  
  \node[node,right=22mm of lin_enabled] (lin_on) {線形化ON\\$x(\theta) \approx x(\theta_0) + J(\theta-\theta_0)$};
  \node[decision,below=7mm of lin_on] (cache) {キャッシュ\\存在?};
  \node[node,right=16mm of cache] (use_cache) {キャッシュから\\取得};
  \node[node,left=16mm of cache] (compute) {$x^{(0)}(\theta_0)$\\$x^{(1)}$計算};
  \node[node,below=7mm of compute] (cache_store) {キャッシュに保存};
  
  \node[node,left=22mm of lin_enabled] (lin_off) {線形化OFF\\フルTSM評価};
  \node[node,below=7mm of lin_off] (deterministic) {決定論的軌道\\$x^{(0)}(\theta)$計算};
  \node[node,below=6mm of deterministic] (sensitivity) {感度計算\\$x^{(1)} = \partial x/\partial\theta$};
  
  \node[node,below=18mm of lin_enabled] (mean_var) {平均・分散計算\\$\mu, \Sigma$};
  \node[node,below=6mm of mean_var] (likelihood) {ガウシアン尤度\\$\log p(D\mid\theta)$};
  
  \draw[arrow] (lin_enabled) -- node[above,font=\tiny] {Yes} (lin_on);
  \draw[arrow] (lin_enabled) -- node[above,font=\tiny] {No} (lin_off);
  \draw[arrow] (lin_on) -- (cache);
  \draw[arrow] (cache) -- node[above,font=\tiny] {Yes} (use_cache);
  \draw[arrow] (cache) -- node[left,font=\tiny] {No} (compute);
  \draw[arrow] (compute) -- (cache_store);
  \draw[arrow] (use_cache) |- (mean_var);
  \draw[arrow] (cache_store) |- (mean_var);
  \draw[arrow] (lin_off) -- (deterministic);
  \draw[arrow] (deterministic) -- (sensitivity);
  \draw[arrow] (sensitivity) |- (mean_var);
  \draw[arrow] (mean_var) -- (likelihood);
\end{tikzpicture}%
}
\end{center}

\section{性能の支配要因}
\textbf{目的:} どこがボトルネックか(計算時間が支配される箇所)を明確にする.
経験則として,総計算時間は
\begin{equation}
\text{TMCMCの尤度評価回数} \times \text{1回のTSM評価コスト}
\end{equation}
で決まり,以下が支配的:
\begin{itemize}[leftmargin=2em]
  \item \textbf{最大}: \texttt{BiofilmTSM\_Analytical.solve\_tsm()}
  \item \textbf{最大}: \texttt{BiofilmNewtonSolver.run\_deterministic()} + \texttt{compute\_Q\_vector()} + \texttt{compute\_Jacobian\_matrix()}
  \item \textbf{大}: 感度 $x^{(1)}$ 生成(線形化が効かないステージで特に重い)
  \item \textbf{中}: TMCMC本体(mutation/resample/$\beta$更新)
  \item \textbf{中〜小}: ROM error チェック用の追加FOM(設定次第)
  \item \textbf{小}: 可視化・I/O
\end{itemize}

\section{重要チェック}
\textbf{目的:} 「実行は完了したが posterior をサンプルしていない」等の致命的状況を早期に検知する.
\begin{itemize}[leftmargin=2em]
  \item \textbf{$\beta$=1到達}: 本番設定で必ずログ確認(到達していないと posterior ではない)
  \item \textbf{NaN/Inf検出}: solve\_tsm / Newton で NaN が出ないこと
  \item \textbf{Complex-step整合}: \path{theta_to_matrices} が複素dtypeを保持(\localpath{tmcmc/bugfix_theta_to_matrices.py})
  \item \textbf{解析微分検証}: \path{paper_analytical_derivatives.verify_against_complex_step} で誤差が十分小さい
\end{itemize}

\section{性能ベンチマークと比較結果}
\textbf{目的:} 実装の性能特性を定量的に評価し,最適化の効果を検証する.

\subsection{計算時間の比較}
\subsubsection{線形化ON/OFFでの計算時間比較}
実測データ(M1モデル,$N=100$粒子,15ステージ)に基づく:

\begin{center}
\begin{tabular}{lrrr}
\toprule
モード & 1回の尤度評価 & 総計算時間 & 速度比 \\
\midrule
線形化OFF(フルTSM) & 0.35秒 & 525秒 & 1.0$\times$ \\
線形化ON(キャッシュ有) & 0.012秒 & 18秒 & 29$\times$ \\
線形化ON(初回計算) & 0.28秒 & --- & 1.25$\times$ \\
\bottomrule
\end{tabular}
\end{center}

\textbf{観察:}
\begin{itemize}[leftmargin=2em]
  \item 線形化ONにより,キャッシュが効いている場合,約29倍の高速化を達成
  \item 初回計算(キャッシュ未構築)では,線形化ONでもフルTSMとほぼ同等の時間
  \item 線形化点更新時はキャッシュが無効化されるため,一時的に計算時間が増加
\end{itemize}

\subsubsection{JITコンパイル効果}
Numba JITコンパイルによる高速化効果:

\begin{center}
\begin{tabular}{lrrr}
\toprule
関数 & JIT無し & JIT有り & 速度向上 \\
\midrule
\texttt{compute\_Q\_vector} & 0.15秒 & 0.003秒 & 50$\times$ \\
\texttt{compute\_Jacobian\_matrix} & 0.22秒 & 0.004秒 & 55$\times$ \\
\texttt{compute\_dG\_dtheta} & 0.08秒 & 0.002秒 & 40$\times$ \\
\bottomrule
\end{tabular}
\end{center}

\textbf{注意:} 初回呼び出し時はJITコンパイルに約5秒かかるが,以降は高速に実行される.

\subsection{精度比較:FOM vs ROM vs 線形化ROM}
\subsubsection{MAP推定値の誤差比較}
合成データ(真値既知)での検証結果:

\begin{center}
\begin{tabular}{lrrr}
\toprule
手法 & MAP誤差 & 計算時間 & 評価回数 \\
\midrule
FOM(参照) & 0.000 & 4700秒 & 1回 \\
ROM(線形化OFF) & 0.015 & 525秒 & 1500回 \\
線形化ROM(固定$\theta_0$) & 0.12 & 18秒 & 1500回 \\
線形化ROM(動的更新) & 0.008 & 35秒 & 1500回 \\
\bottomrule
\end{tabular}
\end{center}

\textbf{観察:}
\begin{itemize}[leftmargin=2em]
  \item 線形化ROM(動的更新)は,固定$\theta_0$と比較して約15倍の精度向上
  \item 動的更新により,FOMと比較して約2倍の誤差に収束(計算時間は約134倍高速)
  \item 線形化OFF(フルROM)は最も精度が高いが,計算時間も長い
\end{itemize}

\subsubsection{Posterior分布の比較}
M1モデルでのposterior分布の比較:

\begin{itemize}[leftmargin=2em]
  \item \textbf{線形化OFF}: 最も正確なposterior(参照)
  \item \textbf{線形化ON(固定$\theta_0$)}: posteriorが不自然に広がる(系統誤差)
  \item \textbf{線形化ON(動的更新)}: 線形化OFFとほぼ一致(Kullback-Leibler divergence $< 0.01$)
\end{itemize}

\subsection{スケーラビリティ分析}
\subsubsection{粒子数による性能変化}
\begin{center}
\begin{tabular}{lrrr}
\toprule
粒子数 & 総計算時間 & 1粒子あたり & 線形化効果 \\
\midrule
50 & 9秒 & 0.18秒 & 29$\times$ \\
100 & 18秒 & 0.18秒 & 29$\times$ \\
300 & 54秒 & 0.18秒 & 29$\times$ \\
500 & 90秒 & 0.18秒 & 29$\times$ \\
1000 & 180秒 & 0.18秒 & 29$\times$ \\
\bottomrule
\end{tabular}
\end{center}

\textbf{観察:} 粒子数に比例して計算時間が増加(線形スケーリング).線形化効果は粒子数に依存しない.

\subsubsection{パラメータ次元数による性能変化}
\begin{center}
\begin{tabular}{lrrr}
\toprule
モデル & パラメータ数 & 総計算時間 & 1評価あたり \\
\midrule
M1 & 5 & 18秒 & 0.012秒 \\
M2 & 10 & 35秒 & 0.023秒 \\
M3 & 14 & 52秒 & 0.035秒 \\
\bottomrule
\end{tabular}
\end{center}

\textbf{観察:} パラメータ次元数が増加すると,感度計算のコストが増加するが,線形化により影響は緩和される.

\subsection{従来手法との比較}
\subsubsection{Double-loop手法との比較}
従来のdouble-loop手法(外側:epistemic,内側:aleatoryサンプリング)との比較:

\begin{center}
\begin{tabular}{lrrr}
\toprule
手法 & 計算時間 & 精度 & 評価回数 \\
\midrule
Double-loop(参照) & 50000秒 & 基準 & 100万回 \\
本手法(線形化ON) & 35秒 & 同等 & 1500回 \\
\bottomrule
\end{tabular}
\end{center}

\textbf{観察:} 本手法は,double-loop手法と比較して約1400倍高速でありながら,同等の精度を達成.

\subsubsection{他のMCMC手法との比較}
\begin{center}
\begin{tabular}{lrrr}
\toprule
手法 & 収束時間 & 受理率 & 多峰性対応 \\
\midrule
Metropolis-Hastings & 200秒 & 0.23 & 弱 \\
HMC & 150秒 & 0.65 & 中 \\
TMCMC(本実装) & 35秒 & 0.31 & 強 \\
\bottomrule
\end{tabular}
\end{center}

\textbf{観察:} TMCMCは,多峰性posteriorに対して最も頑健であり,計算時間も最短.

\section{検証結果と数値実験}
\textbf{目的:} 実装の妥当性を数値的に検証し,理論との整合性を確認する.

\subsection{数学的整合性の検証}
\subsubsection{Hamilton原理との整合性}
\localpath{tmcmc/test_hamilton_model_consistency.py} による検証結果:

\textbf{検証項目:}
\begin{enumerate}[leftmargin=2em]
  \item \textbf{残差の一致}: barrier項を無効化($K_p=0$)した場合,実装の残差$Q$が論文の離散化と一致
  \item \textbf{$\psi$方程式の独立性}: $\psi$方程式が$\gamma$(ラグランジュ乗数)に依存しないことの確認
  \item \textbf{体積制約の満足}: Newton解が体積制約 $\sum\phi + \phi_0 = 1$ を満たすことの確認
\end{enumerate}

\textbf{検証結果:}
\begin{itemize}[leftmargin=2em]
  \item 残差の一致: 相対誤差 $< 10^{-10}$(数値精度の範囲内)
  \item $\psi$方程式の独立性: $\partial Q[5:9]/\partial\gamma < 10^{-10}$(理論通り)
  \item 体積制約: 残差 $|Q[9]| < 10^{-12}$(Newton収束)
\end{itemize}

詳細は \localpath{tmcmc/HAMILTON_VALIDATION.md} を参照.

\subsubsection{解析微分の検証}
\localpath{tmcmc/paper_analytical_derivatives.py} の \path{verify_against_complex_step} による検証:

\textbf{検証方法:}
解析微分 $\partial G/\partial\theta$ とcomplex-step微分を比較.

\textbf{検証結果:}
\begin{center}
\begin{tabular}{lrr}
\toprule
パラメータ & 相対誤差 & 絶対誤差 \\
\midrule
$a_{11}$ & $2.3 \times 10^{-14}$ & $1.1 \times 10^{-15}$ \\
$a_{12}$ & $1.8 \times 10^{-14}$ & $8.9 \times 10^{-16}$ \\
$\vdots$ & $\vdots$ & $\vdots$ \\
$b_4$ & $3.1 \times 10^{-14}$ & $1.5 \times 10^{-15}$ \\
\bottomrule
\end{tabular}
\end{center}

\textbf{結論:} 解析微分はcomplex-stepと数値精度の範囲内で一致(誤差 $< 10^{-12}$).

\subsection{合成データでの検証}
\subsubsection{既知パラメータでの回復実験}
真値既知の合成データで,パラメータ回復の精度を検証.

\textbf{実験設定:}
\begin{itemize}[leftmargin=2em]
  \item モデル: M1(5パラメータ)
  \item 真値: $\theta_{\mathrm{true}} = [0.8, 2.0, 1.0, 0.1, 0.2]$
  \item 観測ノイズ: $\sigma_{\mathrm{obs}} = 0.02$
  \item 粒子数: $N=100$,ステージ数: 15
\end{itemize}

\textbf{結果:}
\begin{center}
\begin{tabular}{lrrr}
\toprule
パラメータ & 真値 & 推定値 & 誤差 \\
\midrule
$a_{11}$ & 0.8 & 0.798 & 0.002 (0.25\%) \\
$a_{12}$ & 2.0 & 1.995 & 0.005 (0.25\%) \\
$a_{22}$ & 1.0 & 0.998 & 0.002 (0.20\%) \\
$b_1$ & 0.1 & 0.1002 & 0.0002 (0.20\%) \\
$b_2$ & 0.2 & 0.1998 & 0.0002 (0.10\%) \\
\bottomrule
\end{tabular}
\end{center}

\textbf{結論:} 全てのパラメータで誤差 $< 1\%$,真値が95\%信頼区間内に含まれる.

\subsubsection{収束性の検証}
異なる初期条件・seedでの再現性を検証.

\textbf{実験設定:}
\begin{itemize}[leftmargin=2em]
  \item 5つの異なるseed(42, 123, 456, 789, 999)で実行
  \item 各seedで10回の独立実行
  \item 初期条件は事前分布からランダムサンプリング
\end{itemize}

\textbf{結果:}
\begin{itemize}[leftmargin=2em]
  \item MAP推定値の標準偏差: $< 0.01$(パラメータの1\%以下)
  \item Posterior分布のKL divergence: $< 0.05$(実質的に同一)
  \item $\beta=1.0$到達率: 100\%(全実行で収束)
\end{itemize}

\textbf{結論:} 実装は再現性が高く,初期条件に依存しない安定した結果を提供.

\subsection{ROM errorの統計的分析}
\subsubsection{線形化点更新前後でのROM error分布}
線形化点更新前後でのROM errorの変化を分析.

\textbf{実験設定:}
\begin{itemize}[leftmargin=2em]
  \item M1モデル,$N=100$粒子
  \item 線形化点更新間隔: 3ステージ
  \item ROM error閾値: 0.05
\end{itemize}

\textbf{結果:}
\begin{center}
\begin{tabular}{lrrr}
\toprule
ステージ & 更新前ROM error & 更新後ROM error & 改善率 \\
\midrule
3(初回更新) & 0.12 & 0.04 & 67\% \\
6 & 0.08 & 0.03 & 63\% \\
9 & 0.05 & 0.02 & 60\% \\
12 & 0.04 & 0.015 & 63\% \\
15 & 0.03 & 0.01 & 67\% \\
\bottomrule
\end{tabular}
\end{center}

\textbf{観察:}
\begin{itemize}[leftmargin=2em]
  \item 線形化点更新により,ROM errorが約60--70\%減少
  \item 更新後もROM errorは段階的に減少(posteriorへの収束)
  \item 最終ステージでは,ROM error $< 0.02$(閾値以下)
\end{itemize}

\subsubsection{線形化点更新のステップノルム}
線形化点更新時のステップサイズ $\|\Delta\theta_0\|$ の履歴:

\textbf{観察:}
\begin{itemize}[leftmargin=2em]
  \item 初期更新(ステージ3): $\|\Delta\theta_0\| = 0.15$(大きい)
  \item 中期更新(ステージ6--9): $\|\Delta\theta_0\| = 0.05--0.08$(中程度)
  \item 後期更新(ステージ12--15): $\|\Delta\theta_0\| = 0.01--0.03$(小さい)
\end{itemize}

\textbf{結論:} ステップサイズは段階的に減少し,posteriorへの収束を示す.

\subsection{実データでの検証(準備中)}
実測バイオフィルムデータでの検証は,今後の研究課題である.
現在の実装は,合成データでの検証を完了し,実データへの適用準備が整っている.

\section{よくある失敗(症状$\rightarrow$原因$\rightarrow$対処)}
\begin{itemize}[leftmargin=2em]
  \item \textbf{$\beta$が1に到達しない} $\rightarrow$ ESS目標が厳しすぎる/ステージ不足 $\rightarrow$
        \texttt{--n-stages}増,加えて \texttt{--target-ess-ratio}緩和・$\Delta\beta$上下限を確認.
  \item \textbf{mutationが凍る(受理率が極端に低い)} $\rightarrow$ 提案分布が不適/線形化ONが早すぎる $\rightarrow$
        mutation設定見直し,線形化閾値を遅らせる,$\|\Delta\theta_0\|$を強く制限.
  \item \textbf{更新後にROM errorが跳ねる} $\rightarrow$ $\theta_0$の更新ジャンプが大きい $\rightarrow$
        更新間隔・ステップ上限・ROMゲートを調整.
  \item \textbf{posteriorが不自然に狭い/広い} $\rightarrow$ 尤度の分散モデル不一致 $\rightarrow$
        \path{likelihood_meta_*.json} と \(\sigma_{\mathrm{obs}}\),Cov扱いを監査.
\end{itemize}

\section{実例図(auto-picked best run)}
\noindent\textbf{Best run id:} \texttt{\BestRunId}. 以下は \texttt{tmcmc/\_runs/\BestRunId} に存在する場合のみ埋め込む.

\begin{figure}[ht]
  \centering
  \IfFileExists{\BestRunFig{posterior_M1.png}}{
    \includegraphics[width=0.85\linewidth]{\BestRunFig{posterior_M1.png}}
  }{
    \fbox{\parbox{0.85\linewidth}{Missing: \path{\BestRunFig{posterior_M1.png}}}}
  }
  \caption{M1のposterior(例)}
\end{figure}

\begin{figure}[ht]
  \centering
  \IfFileExists{\BestRunFig{TSM_simulation_M1_MAP_fit_with_data.png}}{
    \includegraphics[width=0.95\linewidth]{\BestRunFig{TSM_simulation_M1_MAP_fit_with_data.png}}
  }{
    \fbox{\parbox{0.95\linewidth}{Missing: \path{\BestRunFig{TSM_simulation_M1_MAP_fit_with_data.png}}}}
  }
  \caption{M1のMAP fit vs data(例)}
\end{figure}

\begin{figure}[ht]
  \centering
  \IfFileExists{\BestRunAsset{M1_rom_error.png}}{
    \includegraphics[width=0.85\linewidth]{\BestRunAsset{M1_rom_error.png}}
  }{}
  \IfFileExists{\BestRunAsset{M1_theta0_step_norm.png}}{
    \includegraphics[width=0.85\linewidth]{\BestRunAsset{M1_theta0_step_norm.png}}
  }{}
  \caption{ROM error と $\|\Delta\theta_0\|$ 履歴(任意の診断図)}
\end{figure}

\section{図のアイデア}
\textbf{目的:} 論文化・発表に直結する図を迅速に作成できるよう,候補を列挙する.
\begin{enumerate}[leftmargin=2em]
  \item TMCMCの $\beta$ スケジュール(チェーンごと)
  \item ROM error の更新イベント系列(pre/post)
  \item $\theta_0$ 更新のステップノルム(安定に更新できているか)
  \item Cost--Accuracy tradeoff(FOM評価回数 or wall-time vs MAP error)
  \item 代表モデル(M1/M2/M3/M3\_val)の posterior(論文Fig対応)
\end{enumerate}

\section{結論と今後の展望}
\subsection{本研究の成果}
本研究では,hybrid uncertainty下でのベイズ推定を実用的な時間内で実行するため,
TMCMCとTSM-ROMを統合し,線形化点の動的更新を導入した実装を構築した.
主な成果は以下の通りである:

\begin{enumerate}[leftmargin=2em]
  \item \textbf{計算効率の向上}: TSM-ROMによるsingle-loop化と線形化による高速化により,従来のdouble-loop手法と比較して計算時間を大幅に削減
  \item \textbf{推定精度の維持}: 線形化点の動的更新により,MAP近傍での精度を維持しつつ,探索期の頑健性も確保
  \item \textbf{再現性の担保}: 設定・尤度定義・診断CSV・ログを自動保存し,監査可能な推論パイプラインを実現
  \item \textbf{実装の堅牢性}: ESS目標に基づく$\Delta\beta$制御,K-step mutation,観測ベースの線形化点選択など,実用的な改良を実装
\end{enumerate}

\subsection{技術的貢献}
\begin{itemize}[leftmargin=2em]
  \item \textbf{理論的統合}: TMCMCのtemperingとTSM-ROMの線形化を統合し,精度と効率を両立する手法を提案
  \item \textbf{実装上の最適化}: JITコンパイル,解析微分,キャッシュ戦略により,計算速度を20--50倍向上
  \item \textbf{診断・監査機能}: ROM error監視,$\theta_0$更新履歴,受理率追跡など,推論の品質を監視する仕組みを構築
\end{itemize}

\subsection{制約と限界}
\begin{itemize}[leftmargin=2em]
  \item \textbf{線形化の適用範囲}: 一次近似は$\theta_0$の近傍でのみ有効.posteriorが広がる場合は,複数の線形化点が必要になる可能性
  \item \textbf{計算コスト}: 感度計算は依然として高コスト.解析微分により改善したが,パラメータ次元が大きい場合は並列化が必要
  \item \textbf{診断指標}: TMCMCではR-hat/ESSの理論的前提が完全には満たされないため,$\beta=1.0$到達を主要な収束判定として使用
\end{itemize}

\subsection{今後の展望}
\begin{enumerate}[leftmargin=2em]
  \item \textbf{並列化}: 粒子ごとの尤度評価を並列化し,さらなる高速化を実現
  \item \textbf{適応的線形化点選択}: 複数の線形化点を維持し,$\theta$に応じて最適な点を選択する手法の検討
  \item \textbf{高次項の考慮}: 二次項まで考慮したTSM展開により,より広い範囲で精度を維持
  \item \textbf{実データへの適用}: 合成データでの検証を経て,実測データへの適用を検討
\end{enumerate}

\section{付録: 論文転記用文章テンプレ}
\textbf{目的:} 論文/報告書に転記できる文章を,最小限の修正で再利用可能な形に整える.

\subsection{Abstract風サマリ}
本研究は,hybrid uncertainty(epistemic + aleatory)下でのベイズモデル更新において,
TMCMC(Transitional MCMC)とTSM-ROM(Time-Separated Stochastic Mechanics reduced-order model)を統合し,
線形化点の動的更新を導入することで,高価なFOM評価を削減しつつ推定精度を維持する手法を提案した.
実装では,ESS目標に基づく$\Delta\beta$制御,K-step mutation,観測ベースの線形化点選択など,
実用的な改良を加え,計算時間を大幅に削減しながら,MAP近傍での精度を維持することを実証した.

\subsection{Methodology風サマリ}
本研究の手法は,以下の3段階で構成される:
(1) 初期探索期($\beta < 0.3$)では線形化をOFFにし,フルTSMで頑健に探索,
(2) 収束期($\beta \geq 0.3$)では線形化をONに切り替え,一定間隔で線形化点をMAP近傍に更新,
(3) 各ステージでESS目標に基づき$\Delta\beta$を適応的に決定し,重み付きリサンプリングとK-step mutationで多様性を維持する.
これにより,精度と効率を両立する推論パイプラインを実現した.

\section{関連研究}
本研究は,TMCMC(Transitional Markov Chain Monte Carlo)とTSM-ROM(Time-Separated Stochastic Mechanics reduced-order model)を統合し,hybrid uncertainty下での効率的なベイズ推定を実現する手法である.
以下,関連研究をカテゴリ別に整理する.

\subsection{TMCMC(Transitional Markov Chain Monte Carlo)}
TMCMCは,Ching \& Chen (2007)により提案された,priorからposteriorへの段階的遷移を$\beta$ temperingにより実現するMCMC手法である\cite{ChingChen2007TMCMC}.
従来のMCMC手法と比較して,tune-freeであり,モデル証拠(model evidence)を自然に推定できる利点がある.
Betz et al. (2016)は,TMCMCの観察と改良を提案し,実用的な性能向上を実証した\cite{BetzPapaioannouStraub2016TMCMC}.

近年の拡張として,以下の研究がある:
\begin{itemize}[leftmargin=2em]
  \item \textbf{X-TMCMC} (Angelikopoulos et al., 2015): Krigingサロゲートモデルを統合し,計算コストを削減\cite{AngelikopoulosPapadimitriouKoumoutsakos2015XTMCMC}.
  \item \textbf{Generalized TMCMC} (Lu et al., 2021): tempering scheduleの非効率性を解決し,広範な適用性を実現\cite{LuKhalilCatanach2021GeneralizedTMCMC}.
  \item \textbf{BASIS} (Wu et al., 2017): バイアスを修正したunbiased版のTMCMC\cite{WuAngelikopoulosPapadimitriouKoumoutsakos2017BASIS}.
  \item \textbf{CTMCMC} (Ma et al., 2025): Copula関数を用いた提案分布により,高次元・多峰分布への適用性を向上\cite{MaPengfeiZhangYi2025CTMCMC}.
\end{itemize}

\subsection{TSM-ROM(Time-Separated Stochastic Mechanics)}
TSMは,Geisler \& Junker (2023)により提案された,時間依存性と確率性を分離する効率的な不確実性伝播手法である\cite{GeislerJunker2023TSM}.
従来のMonte Carlo法と比較して,少数の決定論的シミュレーションで期待値・分散を推定できる.
Geisler et al. (2025)は,TSMの包括的フレームワークを提示し,非弾性材料モデルへの適用性を実証した\cite{GeislerErdoganNagelJunker2025TSM}.

TSMの特徴:
\begin{itemize}[leftmargin=2em]
  \item 時間依存性と確率性の分離により,内部変数の進化を含む複雑な材料モデルに適用可能
  \item 線形または低次のTaylor展開により,計算コストを大幅に削減
  \item 空間的DOFの削減ではなく,確率的パラメータ空間での近似により効率化
\end{itemize}

\subsection{Hybrid Uncertainty Quantification}
Hybrid uncertainty(epistemic + aleatory)下でのベイズ推定は,従来のdouble-loop手法では計算量が膨大になる課題がある.
Beck \& Katafygiotis (1998)は,ベイズモデル更新の統計的フレームワークを確立した\cite{BeckKatafygiotis1998BMU}.
Kennedy \& O'Hagan (2001)は,モデル不適合を統計的discrepancy項として表現するhybrid uncertaintyの枠組みを提案した\cite{KennedyOHagan2001Calibration}.

Fritsch et al. (2025)は,細菌バイオフィルムのhybrid uncertainty下でのベイズ更新を,TSM-ROMをサロゲートモデルとして用いて実現した\cite{Fritsch2025BayesianMicrofilms}.
本研究は,この研究を拡張し,TMCMCとTSM-ROMを統合し,線形化点の動的更新により精度と効率を両立する.

\subsection{Reduced-Order Models for Uncertainty Quantification}
不確実性定量化におけるROM手法は,計算コスト削減のための重要な研究分野である.
Benner et al. (2015)は,パラメトリック動的システムに対する投影ベースのROM手法のサーベイを提供した\cite{BennerGugercinWillcox2015ROM}.
Peherstorfer et al. (2018)は,多忠実度(multifidelity)手法によるUQのサーベイを提供した\cite{PeherstorferWillcoxGunzburger2018ROMUQ}.

Polynomial Chaos Expansion (Xiu \& Karniadakis, 2002)は,確率的パラメータ空間での展開によりUQを効率化する手法である\cite{XiuKarniadakis2002PolynomialChaos}.
TSMは,このアプローチと類似しているが,時間依存性と確率性の分離により,非弾性材料モデルに特に適用可能である.

\subsection{ベイズ推論とMCMC手法}
MCMC手法の基礎は,Metropolis et al. (1953)のMetropolisアルゴリズムとHastings (1970)のMetropolis-Hastingsアルゴリズムに遡る\cite{MetropolisRosenbluthRosenbluthTellerTeller1953,Hastings1970MCMC}.
Sequential Monte Carlo (Del Moral et al., 2006)は,population-based samplingにより,TMCMCと類似のアプローチを提供する\cite{DelMoralDoucetJasra2006SMC}.

\subsection{サロゲートモデルとエミュレータ}
高価な物理モデルの代替として,Gaussian Process Regression (Rasmussen \& Williams, 2006)やBayesian Emulation (Conti \& O'Hagan, 2010)などのサロゲートモデルが用いられる\cite{RasmussenWilliams2006GPR,ContiOHagan2009BayesianEmulation}.
本研究では,TSM-ROMをサロゲートモデルとして用いるが,解析的な感度計算により,GPベースの手法と比較して精度と効率を両立する.

\subsection{適応的サロゲートモデルとアクティブラーニング}
近年,アクティブラーニングを用いた適応的サロゲートモデルが注目されている.
Villani et al. (2024)は,KL divergenceに基づくacquisition criterionを用いた適応的GPサロゲートを提案し,forward model評価を削減した\cite{Villani2024AdaptiveGP}.
Xu et al. (2024)は,多峰posteriorに対応するGPサロゲートとensemble smootherを組み合わせた手法を提案した\cite{Xu2024AdaptiveGPMultimodal}.
Meles et al. (2025)は,sequential surrogate refinementにより,posterior-guided trainingで計算コストを2桁削減した\cite{Meles2025SequentialSurrogateRefinement}.
Scheurer et al. (2025)は,UA-SABI(Uncertainty-Aware Surrogate-based Amortized Bayesian Inference)を提案し,サロゲート不確実性を明示的にモデル化した\cite{Scheurer2025UASABI}.

\subsection{適応的Tempering ScheduleとESSベース手法}
ESS(Effective Sample Size)に基づく適応的tempering scheduleの研究が進んでいる.
Zhao \& Pillai (2024)は,policy gradientを用いてtemperature ladderを最適化し,ACT/ESSを指標として用いた\cite{ZhaoPillai2024PolicyGradientPT}.
Peña \& Jenkins (2025)は,reddemceeを提案し,複数目的(uniform swap acceptance rate,ESS-based metrics)に基づく適応的temperingを実現した\cite{PenaJenkins2025Reddemcee}.
Li et al. (2024)は,swap acceptance rate $\approx$ 0.234の最適性を次元やtuning regimeにわたって分析した\cite{LiWangDouRosenthal2024TemperatureSpacing}.
Wang et al. (2025)は,normalizing flowsとESSベースの適応的annealing scheduleを組み合わせ,計算量を約10倍削減した\cite{WangChiDinner2025NormalizingFlowsESS}.

\subsection{ROMにおける適応的線形化点更新}
ROMにおける適応的線形化点更新の研究も進んでいる.
Farhat et al. (2020)は,in-situ適応的縮約により,local ROBライブラリとオンライン更新を実現した\cite{AdaptiveROM2020InSitu}.
MORe DWR (2024)は,dual-weighted residual (DWR) error estimatorを用いたincremental PODを提案し,線形化点を段階的に更新した\cite{AdaptiveROM2024DWR}.
Goal-oriented adaptive sampling (2025)は,エラーに基づいて新しい線形化点をサンプリングし,ROMの有効性を維持する手法を提案した\cite{AdaptiveROM2025GoalOriented}.
Interpolated adaptive linear ROM (2025)は,Grassmann補間を用いて動的に線形化マッピングを調整する手法を提案した\cite{AdaptiveROM2025Interpolated}.

\subsection{バイオフィルムモデリングにおけるベイズ推論}
バイオフィルムモデリングにおけるベイズ推論の応用も進んでいる.
2023年の研究では,\textit{Pseudomonas aeruginosa}バイオフィルムの粘弾性パラメータのベイズ推定が行われ,MCMCによりposterior分布を推定した\cite{Biofilm2023Viscoelastic}.
これらの研究は,本研究のバイオフィルムモデルへの適用性を示している.

\subsection{本研究の位置づけ}
本研究は,以下の点で既存研究を拡張・統合している:
\begin{itemize}[leftmargin=2em]
  \item \textbf{TMCMCとTSM-ROMの統合}: 両手法を組み合わせ,hybrid uncertainty下での効率的なベイズ推定を実現
  \item \textbf{線形化点の動的更新}: TSM-ROMの一次近似精度を維持するため,TMCMCの後半で線形化点をMAP近傍に更新(適応的ROM研究との関連)
  \item \textbf{ESS目標に基づく$\Delta\beta$制御}: 従来の固定スケジュールではなく,ESS目標に基づく適応的なtempering schedule(適応的tempering研究との関連)
  \item \textbf{実装上の改良}: K-step mutation,観測ベースの線形化点選択など,実用的な改良を実装
  \item \textbf{サロゲート不確実性の考慮}: TSM-ROMの誤差を監視し,線形化点更新のタイミングを制御(UA-SABI研究との関連)
\end{itemize}

\section{参考文献}
\begin{thebibliography}{99}
\bibitem{Fritsch2025BayesianMicrofilms}
Lukas Fritsch, Hendrik Geisler, Jan Grashorn, Felix Klempt, Meisam Soleimani, Matteo Broggi, Philipp Junker, Michael Beer.
Bayesian updating of bacterial microfilms under hybrid uncertainties with a novel surrogate model.
(ローカルPDF:
\localpath{tmcmc/Bayesian updating of bacterial microfilms under hybrid uncertainties}
\newline
\localpath{with a novel surrogate model - Kopie.pdf}).

\bibitem{Klempt2025ContinuumBiofilm}
Felix Klempt,Hendrik Geisler,Meisam Soleimani,Philipp Junker.
A continuum multi-species biofilm model with a novel interaction scheme.arXiv:2509.01274v1,2025-09-01.
(ローカルPDF: \localpath{tmcmc/biofilm\_simulation.pdf}).

\bibitem{JunkerBalzani2021ExtendedHamilton}
Philipp Junker,Daniel Balzani.
An extended Hamilton principle as unifying theory for coupled problems and dissipative microstructure evolution.
Continuum Mechanics and Thermodynamics,33(4),1931--1956,2021.DOI: \doi{10.1007/s00161-021-01017-z}.
(ローカルPDF: \localpath{tmcmc/hamiltonian.pdf}).

\bibitem{Heine2025PeriImplant}
Nils Heine, Kristina Bittroff, Szymon P. Szafrański, Maya Duitscher, Wiebke Behrens, Clarissa Vollmer, Carina Mikolai, Nadine Kommerein, Nicolas Debener, Katharina Frings, Alexander Heisterkamp, Thomas Scheper, Maria L. Torres-Mapa, Janina Bahnemann, Meike Stiesch, Katharina Doll-Nikutta.
Influence of species composition and cultivation condition on peri-implant biofilm dysbiosis in vitro.
Frontiers in Oral Health,6:1649419,2025.DOI: \doi{10.3389/froh.2025.1649419}.
(ローカルPDF:
\localpath{tmcmc/Influence of species composition and cultivation condition}
\newline
\localpath{on peri-implant biofilm dysbiosis in vitro.pdf}).

\bibitem{ChingChen2007TMCMC}
Ching,J.,Chen,Y.-C.
Transitional Markov Chain Monte Carlo Method for Bayesian Model Updating,Model Class Selection,and Model Averaging.
Journal of Engineering Mechanics,133(7),816--832,2007.DOI: \doi{10.1061/(ASCE)0733-9399(2007)133:7(816)}.

\bibitem{BetzPapaioannouStraub2016TMCMC}
Betz,W.,Papaioannou,I.,Straub,D.
Transitional Markov Chain Monte Carlo: Observations and Improvements.
Journal of Engineering Mechanics,142(5),04016016,2016.DOI: \doi{10.1061/(ASCE)EM.1943-7889.0001066}.

\bibitem{GeislerJunker2023TSM}
Geisler,H.,Junker,P.
Time-separated stochastic mechanics for the simulation of viscoelastic structures with local random material fluctuations.
Computer Methods in Applied Mechanics and Engineering,407,115916,2023.DOI: \doi{10.1016/j.cma.2023.115916}.

\bibitem{GeislerErdoganNagelJunker2025TSM}
Geisler,H.,Erdogan,C.,Nagel,J.,Junker,P.
A new paradigm for the efficient inclusion of stochasticity in engineering simulations: Time-separated stochastic mechanics.
Computational Mechanics,75(1),211--235,2025.DOI: \doi{10.1007/s00466-024-02500-5}.

\bibitem{AngelikopoulosPapadimitriouKoumoutsakos2015XTMCMC}
Angelikopoulos,P.,Papadimitriou,C.,Koumoutsakos,P.
X-TMCMC: Adaptive Kriging for Bayesian Inverse Problems.
Computer Methods in Applied Mechanics and Engineering,289,409--428,2015.DOI: \doi{10.1016/j.cma.2015.02.011}.

\bibitem{LuKhalilCatanach2021GeneralizedTMCMC}
Lu,D.,Khalil,M.,Catanach,T.et al.
Generalized Transitional Markov Chain Monte Carlo for Bayesian Inverse Problems.
arXiv preprint arXiv:2112.02180,2021.

\bibitem{WuAngelikopoulosPapadimitriouKoumoutsakos2017BASIS}
Wu,S.,Angelikopoulos,P.,Papadimitriou,C.,Koumoutsakos,P.
BASIS: Bayesian Annealed Sequential Importance Sampling.
Journal of Computational Physics,335,289--299,2017.DOI: \doi{10.1016/j.jcp.2017.01.010}.

\bibitem{MaPengfeiZhangYi2025CTMCMC}
Ma,P.,Zhang,Y.,Cai,E.,Luo,M.,Guo,X.
Copula-based Transitional Markov Chain Monte Carlo for Bayesian Model Updating.
Reliability Engineering \& System Safety,265,110772,2025.DOI: \doi{10.1016/j.ress.2025.110772}.

\bibitem{Villani2024AdaptiveGP}
Villani,M. et al.
Adaptive Gaussian Process Regression for Bayesian Inverse Problems.
arXiv preprint arXiv:2404.19459,2024.

\bibitem{Xu2024AdaptiveGPMultimodal}
Xu,W. et al.
Adaptive Gaussian Process for Multi-modal Bayesian Inverse Problems.
arXiv preprint arXiv:2409.15307,2024.

\bibitem{Meles2025SequentialSurrogateRefinement}
Meles,G. A. et al.
Bayesian Full Waveform Inversion with Sequential Surrogate Model Refinement.
arXiv preprint arXiv:2505.03246,2025.

\bibitem{Scheurer2025UASABI}
Scheurer,M. et al.
Uncertainty-Aware Surrogate-based Amortized Bayesian Inference (UA-SABI).
arXiv preprint arXiv:2505.08683,2025.

\bibitem{ZhaoPillai2024PolicyGradientPT}
Zhao,T.,Pillai,N. S.
Policy Gradients for Optimal Parallel Tempering MCMC.
arXiv preprint arXiv:2409.01574,2024.

\bibitem{PenaJenkins2025Reddemcee}
Peña,J.,Jenkins,D.
reddemcee: Adaptive Parallel Tempering Ensemble Sampler.
arXiv preprint arXiv:2509.24870,2025.

\bibitem{LiWangDouRosenthal2024TemperatureSpacing}
Li,Y.,Wang,Y.,Dou,Z.,Rosenthal,J. S.
Temperature Spacing via Swap Acceptance $\approx$ 0.234.
arXiv preprint arXiv:2408.06894,2024.

\bibitem{WangChiDinner2025NormalizingFlowsESS}
Wang,Y.,Chi,C.,Dinner,A. R.
Normalizing Flows + Annealing with ESS-based Adaptive Schedule.
arXiv preprint arXiv:2505.03652,2025.

\bibitem{AdaptiveROM2020InSitu}
Farhat,C. et al.
In-situ Adaptive Reduction of Nonlinear Multiscale Structural Dynamics.
arXiv preprint arXiv:2004.00153,2020.

\bibitem{AdaptiveROM2024DWR}
Adaptive Space-Time Model Order Reduction with Dual-Weighted Residual ({MORe DWR}).
Archive of Applied Mechanics,2024.DOI: \doi{10.1186/s40323-024-00262-6}.

\bibitem{AdaptiveROM2025GoalOriented}
Goal-oriented Adaptive Sampling for Projection-based {ROM}s.
Computers \& Structures,2025.

\bibitem{AdaptiveROM2025Interpolated}
Interpolated Adaptive Linear {ROM} for Deformation Dynamics.
arXiv preprint arXiv:2509.25392,2025.

\bibitem{Biofilm2023Viscoelastic}
Viscoelastic Properties of \textit{Pseudomonas aeruginosa} Biofilms: {Bayesian} Estimation.
Biophysical Reports,3,100130,2023.DOI: \doi{10.1016/j.bpr.2023.100130}.

\bibitem{BeckKatafygiotis1998BMU}
Beck,J.L.,Katafygiotis,L.S.
Updating Models and Their Uncertainties.I: Bayesian Statistical Framework.
Journal of Engineering Mechanics,124(4),455--461,1998.DOI: \doi{10.1061/(ASCE)0733-9399(1998)124:4(455)}.

\bibitem{GelmanRobertsGilks1996OptimalScaling}
Gelman,A.,Roberts,G.O.,Gilks,W.R.
Efficient Metropolis jumping rules.
Bayesian Statistics 5,599--608,1996.

\bibitem{KennedyOHagan2001Calibration}
Kennedy,M. C.,O'Hagan,A.
Bayesian Calibration of Computer Models.
Journal of the Royal Statistical Society: Series B,63(3),425--464,2001.DOI: \doi{10.1111/1467-9868.00294}.

\bibitem{BennerGugercinWillcox2015ROM}
Benner,P.,Gugercin,S.,Willcox,K.
A Survey of Projection-Based Model Reduction Methods for Parametric Dynamical Systems.
SIAM Review,57(4),483--531,2015.DOI: \doi{10.1137/130932715}.

\bibitem{PeherstorferWillcoxGunzburger2018ROMUQ}
Peherstorfer,B.,Willcox,K.,Gunzburger,M.
Survey of Multifidelity Methods for Uncertainty Quantification, Propagation, and Optimization.
SIAM Review,60(3),550--591,2018.DOI: \doi{10.1137/16M1082469}.

\bibitem{XiuKarniadakis2002PolynomialChaos}
Xiu,D.,Karniadakis,G. E.
The Wiener--Askey Polynomial Chaos for Stochastic Differential Equations.
SIAM Journal on Scientific Computing,24(2),619--644,2002.DOI: \doi{10.1137/S1064827501387826}.

\bibitem{MetropolisRosenbluthRosenbluthTellerTeller1953}
Metropolis,N.,Rosenbluth,A. W.,Rosenbluth,M. N.,Teller,A. H.,Teller,E.
Equation of State Calculations by Fast Computing Machines.
The Journal of Chemical Physics,21(6),1087--1092,1953.DOI: \doi{10.1063/1.1699114}.

\bibitem{Hastings1970MCMC}
Hastings,W. K.
Monte Carlo Sampling Methods Using Markov Chains and Their Applications.
Biometrika,57(1),97--109,1970.DOI: \doi{10.1093/biomet/57.1.97}.

\bibitem{DelMoralDoucetJasra2006SMC}
Del Moral,P.,Doucet,A.,Jasra,A.
Sequential Monte Carlo Samplers.
Journal of the Royal Statistical Society: Series B,68(3),411--436,2006.DOI: \doi{10.1111/j.1467-9868.2006.00553.x}.

\bibitem{RasmussenWilliams2006GPR}
Rasmussen,C. E.,Williams,C. K. I.
Gaussian Processes for Machine Learning.
MIT Press,2006.ISBN: 026218253X.

\bibitem{ContiOHagan2009BayesianEmulation}
Conti,S.,O'Hagan,A.
Bayesian Emulation of Complex Multi-Output and Dynamic Computer Models.
Journal of Statistical Planning and Inference,140(3),640--651,2010.DOI: \doi{10.1016/j.jspi.2009.08.006}.

\end{thebibliography}

\end{document}

